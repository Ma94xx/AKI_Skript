\documentclass[12pt, german]{article}
\usepackage[ngerman]{babel}
\usepackage[T1]{fontenc}  
\usepackage[utf8]{inputenc}
\usepackage{amsmath}
\usepackage{dsfont}
\usepackage{array}
\usepackage{amssymb}
\usepackage{enumitem}
\usepackage{upgreek}
\usepackage{graphicx}
\usepackage{pdfpages}
\usepackage{listings}  
\usepackage{mathtools}
\usepackage{listings}
\usepackage{endnotes}
\usepackage{color}
\usepackage{tasks}
\usepackage{mathtools}
\usepackage{forest}
\usetikzlibrary{backgrounds,fit}
\usepackage[outline]{contour}
\usetikzlibrary{shapes.multipart}
\usetikzlibrary{shapes,fit}
\usetikzlibrary{fit}
\usepackage{wrapfig}
\usepackage{hyperref}
\usepackage{tikz-cd}
\usepackage[bottom]{footmisc}


\hypersetup{
	colorlinks,
	citecolor=black,
	filecolor=black,
	linkcolor=black,
	urlcolor=black
}

%Für Binärbaum
\usepackage{tikz}
\tikzset{
	treenode/.style = {align=center, inner sep=0pt, text centered, font=\upshape},
	arn_b/.style = {treenode, circle, black, font=\upshape, draw=black, fill=okmama, text width=1.5em, thick},
	arn_x/.style = {treenode, rectangle, draw=black, minimum width=0.5em, minimum height=0.5em}
}


%Zum runden
\DeclarePairedDelimiter{\ceil}{\lceil}{\rceil}

%Document Feineinstellungen
\usepackage[a4paper, left=2cm, right=2cm, top=2.5cm]{geometry}

%Deaktiviert Seitenummerierung
\pagenumbering{gobble}

%Farben für Einstellungen der Code Fragmente
\definecolor{mygreen}{rgb}{0,0.6,0}
\definecolor{mygray}{rgb}{0.5,0.5,0.5}
\definecolor{okmama}{RGB}{150,170,120}
\definecolor{character}{RGB}{179, 178, 255}

%Einstellungen für Code Fragmente
\lstset{ 
	backgroundcolor=\color{white},   % choose the background color; you must add \usepackage{color} or \usepackage{xcolor}; should come as last argument
%	basicstyle=\footnotesize,        % the size of the fonts that are used for the code
%	breakatwhitespace=false,         % sets if automatic breaks should only happen at whitespace
%	breaklines=true,                 % sets automatic line breaking
	captionpos=b,                    % sets the caption-position to bottom
	commentstyle=\color{mygreen},    % comment style
	deletekeywords={...},            % if you want to delete keywords from the given language
	escapeinside={(*}{*)},          % if you want to add LaTeX within your code
	extendedchars=true,              % lets you use non-ASCII characters; for 8-bits encodings only, does not work with UTF-8
	frame=single,	                   % adds a frame around the code
	keepspaces=true,                 % keeps spaces in text, useful for keeping indentation of code (possibly needs columns=flexible)
	keywordstyle=\color{blue},       % keyword style
	language=Octave,                 % the language of the code
	morekeywords={*, Algorithm, ...},            % if you want to add more keywords to the set
	numbers=left,                    % where to put the line-numbers; possible values are (none, left, right)
	numbersep=5pt,                   % how far the line-numbers are from the code
	numberstyle=\tiny\color{mygray}, % the style that is used for the line-numbers
	rulecolor=\color{black},         % if not set, the frame-color may be changed on line-breaks within not-black text (e.g. comments (green here))
	showspaces=false,                % show spaces everywhere adding particular underscores; it overrides 'showstringspaces'
	showstringspaces=false,          % underline spaces within strings only
	showtabs=false,                  % show tabs within strings adding particular underscores
	stepnumber=1,                    % the step between two line-numbers. If it's 1, each line will be numbered
	stringstyle=\color{mymauve},     % string literal style
	tabsize=2,	                   % sets default tabsize to 2 spaces
	title=\lstname                   % show the filename of files included with \lstinputlisting; also try caption instead of title
}
%Einstellungen für Tasks
\settasks{
	label-width=4ex,
	label-offset = 0pt,
	item-indent = 1em,
}




%Global Align
\newcommand*{\LongestName}{\ensuremath{h(x)+g(x)}}% function name
\newcommand*{\LongestValue}{\ensuremath{-1}}% function value
\newcommand*{\LongestText}{fermentum fringilla mauris }%

%Bigger fracs 
\newcommand\ddfrac[2]{\frac{\displaystyle #1}{\displaystyle #2}}

\newlength{\LargestNameSize}%
\newlength{\LargestValueSize}%
\newlength{\LargestTextSize}%

\settowidth{\LargestNameSize}{\LongestName}%
\settowidth{\LargestValueSize}{\LongestValue}%
\settowidth{\LargestTextSize}{\LongestText}%

% Choose alignment of the various elements here: [r], [l] or [c]
\newcommand*{\MakeBoxName}[1]{{\makebox[\LargestNameSize][r]{\ensuremath{#1}}}}%
\newcommand*{\MakeBoxValue}[1]{\ensuremath{\makebox[\LargestValueSize][l]{\ensuremath{#1}}}}%
\newcommand*{\MakeBoxText}[1]{\makebox[\LargestTextSize][l]{#1}}%

%Isomorph zeichen
\newcommand\iso{\xrightarrow{
		\,\smash{\raisebox{-0.65ex}{\ensuremath{\scriptstyle\sim}}}\,}}

%Erzeugt das "bewiesen" Kästen rechts unten
\newcommand{\bewiesen}{

\begin{flushright}
		$\square$  \\
\end{flushright}}


\newcommand\TBox[3][]{%
	\tikz\node[draw,thick,text width=#2,#1] {#3};}


\title{Inoffizielles Skript für Algebra \& Kombinatorik}

\author{Maximilian Kurz \and Amelie Heindl}

\setlength\parindent{0pt}
\begin{document}
\maketitle
\newpage
\tableofcontents
\newpage

\section{Grundlagen - Vorlesung 1}
\subsection{Diedergruppen}
\subsubsection{Definition}
		   Eine Diedergruppe ist die Bewegungsgruppe des regelmäßigen $n$-Ecks mit $n \geq 3$. Dabei ist die Gruppenverknüpfung die Drehung oder Spiegelung des $n$-Ecks. \\ 
		   
		   Das $n$-Eck wird durch Knoten und Kanten beschrieben und eine Drehung/Spiegelung wird durch eine Abbildung wie folgt definiert: 
		   \begin{align*}
		   	\text{Knoten: } &\mathbb{Z}/n\mathbb{Z} \\
		   	\text{Kanten: } &\{ \{i, i+1\} \mid i \in \mathbb{Z}/n\mathbb{Z} \}  \\
		   	\text{Abbildung von Knoten in Knoten: } &f: \mathbb{Z}/n\mathbb{Z} \iso \mathbb{Z}/n\mathbb{Z}\\
		   \end{align*}
		   
		   Zu jeden $n$ gehört eine $D_n$ mit $|D_n| = 2n$

\subsubsection{Eigenschaften von Diedergruppen}
		Wir bezeichnen die Spieglung mit $\sigma$ und die Drehung im Uhrzeiger mit $\delta$. Dann gelten folgende Regeln:
		
		\begin{align*}
			\sigma^2 = id = \delta^n = \delta^0 \\ 	 
		\end{align*}
		In Worten bedeutet dies, dass $n$-faches drehen oder doppeltes Spiegeln eines  $n$-Ecks keine Änderung bewirkt.  \\
		Daraus lässt sich dann folgender Satz ableiten mit $\forall \varepsilon, \varepsilon', m, m' \in \mathbb{Z}$: 
		
		\begin{align*}
			\sigma^\varepsilon\delta^m = \sigma^{\varepsilon'}\delta^{m'} \iff \varepsilon \equiv \varepsilon' (2) \text{ und } m \equiv m' (n)
		\end{align*}
		Weiter gilt, dass $\sigma\delta = \delta^{-1}\sigma$, also dass erst spiegeln und dann drehen den selben Effekt hat wie in die andere Richtung drehen und dann spiegeln. Im folgenden wird gezeigt weshalb dies gilt
		
		\begin{align*}
			\sigma \delta(j) = \sigma(j+1) = -(1 + j) = -j -1 = \delta^{-1}(-j)=\delta^{-1}\sigma(j)
		\end{align*}
		\bewiesen
		Dieser Satz gibt uns also die Möglichkeit das $\sigma$ nach rechts zu schieben.
		
\subsubsection{Fixpunkte in Diedergruppen}
		Ein Fixpunkt $x$ einer Abbildung ist ein Punkt bei dem $f(x) = x$ gilt. Für Diedergruppen fragen wir uns also ab wann 
		\begin{align*}
			\sigma\delta^i(j) = j = \sigma(i+j) = -i -j
		\end{align*} gilt. 
		Dies ist der Fall, wenn $-i = 2j \in \mathbb Z /n \mathbb Z$ mit ungeradem $n$ gilt. Bei Spiegelungen existiert also genau ein Knoten $j$ für alle $i$. Falls $n$ gerade ist, existieren entweder genau $2$ (das wäre dann $-i = 2j$ und $ i + \frac{n}{2}$) oder gar kein Fixpunkt. Für Drehungen ist nur $n$ bzw. $id$ eine Fixpunkt. 
		
		
\section{Gruppentheorie 1 - Vorlesung 2}
\subsection{Untergruppen}
\subsubsection{Definition}
		Sei im folgenden $G=(G, \cdot, 1)$ eine Gruppe. Dann gilt 
		\begin{align*}
			H \subseteq G \text{ ist eine Untergruppe} \iff H \not = \emptyset \text{ und } \forall g, h \in H : gh^{-1} \in H
		\end{align*}
		$H$ ist also eine Untergruppe von $G$, genau dann wenn $H$ nicht leer ist und für alle Elemente Inverse vorhanden sind.  


\subsection{Nebeklassen}
\subsubsection{Definition}
		Für eine Untergruppe $H \leq G$ sind die folgenden Nebenklassen definiert
		\begin{align*}
			 G/H &= \{gH \subseteq G \mid g \in G\} \quad \text{Rechtsnebenklasse } \\
			 H/G &= \{Hg \subseteq G \mid g \in G\} \quad \text{Linksnebenklasse } 
		\end{align*}
		wobei $gH$ die Multiplikation von  g mit allen Elementen aus $H$ ist. 

\subsubsection{Größe von Nebenklassen}
		Wir definieren die folgende Abbildung
		\begin{align*}
			\forall f,g \in G \text{ ist } (f^{-1}g \cdot): gH \iso fH \text{ bijektiv mit der Umkehrabbildung } (gf^{-1}\cdot)
		\end{align*}
		Wir können dann folgern dass 
		\begin{align*}
			gH \iso H \iso Hf \quad \text{ für alle } f,g \in G
		\end{align*}
		Es gilt also $|gH| = |Hf| = |H|$

\subsubsection{Gleichheit von Nebenklassen}
		Im Folgenden wird gezeigt dass $gH \cap fH \not = \emptyset \iff gH = fH$ gilt.
		Die Rückrichtung ist hierbei trivial, da der Schnitt von $gH$ und $fH$ bei Gleichheit offensichtlich nicht leer ist. \\
		
		Für die Hinrichtung seinen $h_1, h_2 \in H$ mit $gh_1 = fh_2$
		\begin{align*}
			&\implies g=fh_2h_1^{-1} \in fH \\
			&\implies gH \subseteq fH. 
		\end{align*}
		Analog dazu gilt $fH \subseteq gH \implies gH = fH$
		\bewiesen
		
\subsubsection{Graphische Interpretation von Nebenklassen}
Sei $G= (G, \cdot, 1)$ eine Gruppe und $H\leq G$
\begin{figure}[h!]
	\centering
	$G=\begin{array}{c}
	
	\begin{tikzpicture}[every fit/.style={inner sep=0pt, outer sep=0pt, draw}]
	\begin{scope}[yshift=1.5cm,y=1cm]
	\node [fit={(0,0) (1,3)}, label=center:{$H$}] {};
	\end{scope}
	\begin{scope}[yshift=1.5cm,y=1cm]
	\node [fit={(1,0) (5,1)}, label=center:{$fH$}] {};
	\end{scope}
	\begin{scope}[yshift=2.5cm,y=1cm]
	\node [fit={(1,0) (5,1)}, label=center:{$\vdots$}] {};
	\end{scope}
	\begin{scope}[yshift=3.5cm,y=1cm]
	\node [fit={(1,0) (5,1)}, label=center:{$gH$}] {};
	\end{scope}
	
	\end{tikzpicture} 
\end{array}
= 
\begin{array}{c}
	\begin{tikzpicture}[every fit/.style={inner sep=0pt, outer sep=0pt, draw}]
	\begin{scope}[yshift=1.5cm,y=1cm]
	\node [fit={(0,0) (1,3)}, label=center:{$H$}] {};
	\end{scope}
	\begin{scope}[yshift=1.5cm,y=1cm]
	\node [fit={(1,3) (2,0)}, label=center:{$Hg$}] {};
	\end{scope}
	\begin{scope}[yshift=1.5cm,y=1cm]
	\node [fit={(2,0) (3,3)}, label=center:{$\ldots$}] {};
	\end{scope}
	\begin{scope}[yshift=1.5cm,y=1cm]
	\node [fit={(3,3) (4,0)}, label=center:{$Hf$}] {};
	\end{scope}
	
	\end{tikzpicture} 
\end{array}$
\caption{Zerlegung von $G$ in Links bzw. Rechtsnebenklassen}
\end{figure}

Wir sehen dass $|H|=|gH| = |Hf|$ gilt

\subsection{Repräsentantensysteme}		
\subsubsection{Definition}
	Sei $R_H := \{r(gH) \mid gH \in G/H\} \subseteq G$ mit der folgenden bijektiven  Abbildung
	\begin{align*}
		R_H \iso G/H \\
		r(gH) \mapsto gH 
	\end{align*}

\subsection{Satz von Lagrange}		
\subsubsection{Definition und Beweis}
	Wir definieren die folgende Abbildung
	\begin{align*}
		\lambda: (G/H) \times H &\iso G \text{ ist bijektiv} \\
		(gH, h) &\mapsto r(gH) \cdot h
	\end{align*}
	 
	Nun bezeichnen wir mit $[G:H] = |G/H|$ den Index von $H$ in $G$. \\ 
	Wenn jetzt $R = \{r(gH) \mid g \in G \}$ ein Linksrepresentantensystem (LRS) ist, \\ 
	dann ist $R^{-1} = \{r(gH)^{-1} \mid g \in G \}$ ein RRS. 
	Denn es gilt $(rH)^{-1} = H^{-1}r^{-1} = Hr^{-1}$, da $H=H^{-1}$.
	Also gilt $|G/H| = |H/G|$ \\
	
	Wenn für $|G| < \infty $ gilt, ist $|H| < \infty $ und $|G| = |G/H| \cdot |H|$ und 
	außerdem sind $\{gH \mid g \in G \}$ und $\{Hg \mid g \in G \}$ Partitionen. Weiter gilt $|gH| = |Hf| \;\;\forall g,f \in G$. \\
	Also gilt $|G| = [G:H] \cdot |H|$ 


\subsection{Der Homomorphiesatz}		
\subsubsection{Grundlagen Homomorphismus}
	Eine Abbildung $\varphi: G \to F$ heißt Homomorphismus\footnote{ab hier abgekürzt Hom.} falls $\forall g,h \in G : \varphi(gh) = \varphi(g)\varphi(h)$ \\
	\begin{enumerate}[label=\roman*)]
		\item $\varphi(1_G) = 1_H$ ~\par
			Denn $g=1\cdot g \implies \varphi(g)=\varphi(1)\varphi(g) \implies 1_H = \varphi(g) \varphi(g)^{-1} = \varphi(1)$
		
		\item $\varphi(g^{-1})=\varphi(g)^{-1}$ ~\par
			Denn $\varphi(1)=\varphi(gg^{-1}) \implies 1_h = \varphi(g)\varphi(g^{-1}) \implies \varphi(g)^{-1} = \varphi(g^{-1})$
	\end{enumerate}

\subsubsection{Lemma}
	Ein Hom. $\varphi: G \to F$ ist injektiv $\iff$ $ker(\varphi) = \{g \in G \mid \varphi(g)= 1\} = \{1\}$ 
	
	Im Folgenden geben hierfür einen Beweis an in dem wir $\varphi(g) = \varphi(h) \iff \varphi(gh^{-1}) = \{1\} $ zeigen:
	\begin{align*}
		\varphi \text{ ist injektiv } &\implies ker(\varphi) = \{1\} \\
		&\implies [\varphi(gh^{-1}) = \{1\} \implies gh^{-1} = 1 \implies g = h] \\ 
		&\implies \varphi \text{ ist injektiv}
	\end{align*}

\subsubsection{Homomorphiesatz}
	Sei $\varphi: G \to F$ ein Hom.  und $N = ker(\varphi)$. \\
	Dann induziert $\varphi$ ein Hom. $\overline{\varphi} : G/N \to F$ der injektiv ist durch $\overline{\varphi}(gH) = \varphi(g)$ 
	\newline
	
	Also induziert $\varphi$ einen Isomorphismus $\overline{\varphi}: G/N \iso im(G) = \{\varphi(g) \mid g \in G\} \leq F$

\subsubsection{Beweis Homomorphiesatz}
	\begin{enumerate}[label=\arabic*)]
		\item $\overline{\varphi}$ ist wohldefiniert ~\par
				Sei $f\in gH \implies \varphi(f) \in \varphi(g)N = \{\varphi(g)\}$
		
		\item $\overline{\varphi}$ ist injektiv ~\par
			$ker(\overline{\varphi}) = ker(\varphi) = N = 1_{G/N}$
	\end{enumerate}

\subsubsection{Beispiel für den Homomorphiesatz}
	Wir definieren 
		\begin{align*}
			exp = \begin{cases}
			\mathbb{R} \to \mathbb{C}^\ast\\
			x \mapsto e^{2\pi i x}
			\end{cases}
		\end{align*}
	Dann ist $ker(exp) = \mathbb Z $.\\ 
	 Also ist $\mathbb R / \mathbb Z \cong \{z \in \mathbb C \mid |z| = 1\}$	



\subsection{Normalteiler}		
\subsubsection{Definition}
	$H \subseteq G$ heißt Normalteiler ($H\trianglelefteq G$) falls $\forall g \in G : gH = Hg$. 

\subsubsection{Satz zu Normalteilern}
	Sei $H \subseteq G$, dann sind die folgenden Aussagen äquivalent: 
	\begin{enumerate}[label=\arabic*)]
		\item $H\trianglelefteq G$
		\item $H \leq G$ und $gH \cdot fH = gfH$ definiert eine Gruppenstruktur auf $G/H$
		\item Es existiert ein Hom. $\varphi: G \to F$ mit $ker(\varphi) = \{g \in G \mid \varphi(g)= 1\} = H$
		\item $\forall g \in G : gHg^{-1} \subseteq H$ und $H \leq G$
	\end{enumerate}

	Im Folgenden wird die Äquivalenz der Aussagen gezeigt: 
	\begin{itemize}
		\item $1)\rightarrow 2)$ ~\par
			$H\trianglelefteq G \implies gHfH = g(Hf)H = g(fH)H = gfH$
			\begin{align*}
				&\text{ Neutrales Element: } &&1_{G/H} = H \\
				&\text{ Inverse: } &&(gH)^{-1} = H^{-1}g^{-1}= Hg^{-1}=g^{-1}H \\
				&\text{ Assoziativität: } &&(fH)(gH)(hH)= fghH\\
			\end{align*}
			
		\item $2)\rightarrow 3)$ ~\par
		Sei $\varphi : G \to G/H $ und $\varphi(g) = gH$, dann
			\begin{align*}
				ker(\varphi) = \{g \in G \mid gH = H\} = H = 1_{G/H}
			\end{align*}
		
		\item $3)\rightarrow 4)$~\par
		Sei $H = ker(\varphi)$, dann
			\begin{align*}
				\varphi(gHg^{-1}) = \varphi(gg^{-1}) = \varphi(1) \subseteq H 
			\end{align*}
		
		\item $4)\rightarrow 1)$~\par
		$\forall g \in G : gHg^{-1} \subseteq H$
			\begin{align*}
			&\implies  \forall g \in G : H \subseteq g^{-1}Hg \\
			&\implies \forall g \in G : H \subseteq gHg^{-1} \\
			&\implies gHg^{-1} = H \\
			&\implies gH = Hg
			\end{align*}
	\end{itemize}



\subsection{G-Mengen}		
\subsubsection{Definition}
	Sei $G = (G, \cdot, 1)$ eine Gruppe und $X$ eine Menge.
	Dann operiert $G$ auf $X$ falls folgende Abbildung existiert 
	\begin{align*}
		\cdot : G \times X \to X \\ 
		(g,x) \mapsto gx
	\end{align*}
	Mit der Eigenschaft $g(h(x)) = gh(x)$ und $g(1) = id_x$

\subsubsection{Homomorphismus auf G-Mengen}
	Ein Hom. von $G$-Mengen $X,Y$ ist eine Abbildung 
	\begin{align*}
		f: X \to Y \text{ mit } f(gx) = gf(x)
	\end{align*}

\subsubsection{Fixpunkte in G-Mengen}
	Sei $X$ eine  $G$-Menge. Dann heißt $x$ ein Fixpunkt von $g \in G$ falls $gx=x$. \\
	Weiter ist die Menge der Fixpunkte eine Untergruppe von $G$
	\begin{align*}
		Fix(x) = \{g \in G \mid  gx = x\} \leq G 
	\end{align*}


\subsection{Das Bahnenlemma}		
\subsubsection{Definition einer Bahn}
	Wir definieren eine Bahn wie folgt $$Bahn(x) = Orbit(x) = G\cdot x = \{gx \mid  g \in G \}$$
	
	
	Ebenfalls gilt das $Gx \cong G/Fix(x)$ ein Isomorphismus von $G$-Mengen ist, wobei $Gx$ ist eine $G$-Menge ist. 
	Dies ist wie folgt zu begründen. Wir betrachten die folgende Abbildung
	\begin{align*}
		G/Fix(x) \to Gx \\ 
		gFix(x) \mapsto gx
	 \end{align*}
	 Diese ist wohldefiniert, da $h \in gFix(x) \implies h=gf$ mit $f \in Fix(x)$. \\ 
	 Also $h(x) = gf(x) = g(x)$ 


\subsubsection{Bahnenlemma}
	Das Bahnenlemma besagt $$Gx \cap Gy \not = \emptyset \iff Gx = Gy$$
	Im Allgemeinen also $$ X = \dot{\bigcup} \{Gx \mid  x \in X \}$$ Wobei hier die disjunkte Vereinigung mit dem Punkt gekennzeichnet wird. 
\subsubsection{Beweis des Bahnenlemmas}
	Die Rückrichtung ist hier ebenfalls trivial (wie bei dem sehr ähnlichen Resultat über Nebenklassen). Wir zeigen deshalb die andere Richtung. \\ 
	
	Sei $gx = hy$ für $g,h \in G$
	\begin{align*}
		&\implies h^{-1}g(x) = y \\
		&\implies y \in Gx \\
		&\implies Gy \subseteq Gx
	\end{align*}
	Analog dazu $Gx \subseteq Gy \implies Gx = Gy$

\subsubsection{Graphische Interpretation einer Bahn}

Sei $X$ eine $G$-Menge und 

\begin{figure}[h!]
	\centering
	$X=\begin{array}{c}
	
	\begin{tikzpicture}[every fit/.style={inner sep=0pt, outer sep=0pt, draw}]
	\begin{scope}[y=1.5cm]
	\node [fit={(4,0) (7,1)}, label=center:{$Gx_1$}] {};
	\end{scope}		
	\begin{scope}[yshift=1.5cm,y=1cm]
	\node [fit={(0,-1.5) (4,1)}, label=center:{$Gx_2$}] {};
	\node [fit={(4,0) (7,1)}, label=center:{$Gx_3$}] {};
	\end{scope}
	\begin{scope}[yshift=2.5cm,y=1.2cm]
	\node [fit={(0,0) (5,1)}, label=center:{$Gx_4$}] {};
	\node [fit={(5,0) (7,1)}, label=center:{$Gx_5$}] {};
	\end{scope}
	\end{tikzpicture}
\end{array}$
\caption{Disjunkte Zerlegung von $X$ in Bahnen}
\end{figure}
Wir wissen, dass $|Gx|=|G/Fix(x)|$ ein Teiler von $|G|$ ist. Also gilt $|G| = p$ wobei $p$ eine Primzahl ist, und daraus folgt $\forall x \in X : |Gx|=1$ oder $|Gx|=p$

\subsection{Satz von Cauchy}		
\subsubsection{Definition}
	Sei $G$ eine endliche Gruppe und $p \in \mathbb P$ mit $p \mid |G|$. \\
	Dann existiert $g \in G$ mit $g \not = 1$ und $ g^p = 1$. Also $ord(g)= p$
\subsubsection{Beweis}
	Wir betrachten die Menge der Tupel $(g_1, g_p) \in G^p$ mit $g_1 \cdot \ldots \cdot g_p = 1$. 
	Sei nun 
	\begin{align*}
		T &= |\{(g_1, \ldots, g_p) \mid  g_1 \cdot \ldots \cdot g_p = 1 \}| \\
		&= |\{(g_1, \ldots, g_{p-1}) \in G^{p-1} \}| \\
		&= |G^{p-1}|
	\end{align*}
	Nun soll $\mathbb Z /p\mathbb Z$ auf $T$ operieren durch
	 \begin{align*}
		T = (\underbrace{g_1, \ldots, g_m}_{\substack{u}}, \underbrace{g_{m+1}, \ldots, g_p}_{\substack{v}}) \implies m\cdot T = (\underbrace{g_{m+1}, \ldots, g_p}_{\substack{v}}, \underbrace{g_1, \ldots, g_m}_{\substack{u}})
	\end{align*}
	Es gilt also $uv = 1 \iff v(uv)v^{-1} = 1 = u = 1$, da $|G^{-1}| = n^{p-1}$ und $p \mid n^{p-1} $ da $p \mid n$.
	Weiter haben Bahnen entweder die Länge $1$ oder $p$, da $\mathbb Z /p\mathbb Z$ nur zwei Untergruppen hat.

	\begin{align*}
		\mathbb Z /p\mathbb Z \cdot (1, \ldots, 1) = (1, \ldots, 1) \\
		\{(g, \ldots, g) \mid gP= 1 \} \implies \exists g \text{ mit } gP = 1
	\end{align*}
\bewiesen

\section{Gruppentheorie 2 - Vorlesung 3}
\subsection{p-Gruppen}
\subsubsection{Definition p-Gruppe}
	$G$ heißt $p$-Gruppe falls $p$ eine Primzahl ist und $|G| \in p^{\mathbb N}$

\subsubsection{Das Zentrum einer Gruppe}
	Wir definieren das Zentrum einer Gruppe wie folgt $$Zentrum(G)= Z(G) = \{g \in G \mid  \forall x \in G: xg = gx\}$$

\subsubsection{Satz über das Zentrum}
	\begin{align*}
		G \text{ ist eine $p$-Gruppe} \implies Z(G) \not = \{1\}
	\end{align*}

\subsubsection{Beweis des Satzes}
	$G$ operiert auf sich selbst durch $x \mapsto gxg^{-1}$. Dann ist $G$ disjunkt zerlegt in Bahnen. Für jede Bahn gilt $|G/Fix(x)|$. 
	Also $Fix(x) \not= G \implies p \mid |G/Fix(x)|$. Dann gilt 
	\begin{align*}
		|G| &\equiv \sum_{x \in G} |G/Fix(x)|  \mod p\\
		&\equiv |Z(G) |\mod p
	\end{align*}
	Daraus folgt $|Z(G)| \geq 1$ und durch die Kongruenz zu $p$ muss $|Z(G)|\geq p$ gelten. \\
	Also $p \mid |Z(G)|$. 
	\newline
	
	Wir können dann daraus schließen dass $G/Z(G)$ eine $p$-Gruppe mit $Z(G/Z(G)) \not = \{1\}$ oder $G=Z(G)$ ist.
	\bewiesen
	
\subsection{Zyklische Gruppen}
\subsubsection{Nilpotenz}
	Eine Gruppe $G$ heißt nilpotent falls es eine Kette von Gruppen gibt, sodass
	\begin{align*}
		\langle 1 \rangle &= G_0, G_1, \ldots, G_M = G  \text{ und } G_K/Z(G) = G_{k-1} \text{ für } 1 \leq k \leq k
	\end{align*}

\subsubsection{Beispiele nilpotenter Gruppen}
	\begin{itemize}
		\item $|D_3| = 6$ ist die kleinste nicht nilpotente Gruppe, da $Z(D_3) = \{1\}$ und somit nicht kommutativ.
		\item $|D_4| = 8$ ist nilpotent und kommutativ, da $D_4/Z(D_4) = \mathbb Z/2 \mathbb Z \times \mathbb Z /2\mathbb Z$
	\end{itemize}

\subsubsection{Zyklische Gruppe}
	Eine Gruppe $G$ heißt zyklisch falls $\exists g \in G$ mit $G=g^{\mathbb Z}$

\subsubsection{Satz zu zyklischen Gruppen}
	$|G| < \infty$ und $G$ zyklisch $\implies \exists g \in G$ mit $G =\{g^0, g^1, \ldots, g^{|G| -1} \}$  

\subsubsection{Beweis des Satzes }
 Betrachte $g, g^2, g^3, \ldots$ für $g^{\mathbb Z}$. Dann existiert ein $i,j$ mit $0 \leq i < j $ sodass $g^i = g^j$. Daraus ergibt sich dass $1 = g^{j-i}$ gilt. \\
 Betrachte nun ein minimales $n \in \mathbb N_+$ mit $g^n = g^0 = 1$ (\OE  $\, \,G \not = \{1\}$).
 Da $n$ minimal ist, sind  $g, \ldots g^{n-1}$ paarweise verschieden. Ferner gilt $g^i = g^j \implies i \equiv j \mod n \implies G \cong (\mathbb Z /n\mathbb Z, +, 0)$
 \bewiesen
 
\subsubsection{Satz zu zyklischen Gruppen}
 	$H \leq G$ zyklisch $\implies H$ ist zyklisch
 	
\subsubsection{Beweis des Satzes}
 	\OE  $\, \, \{1\} \not = H \not = G$ und  $G \cong \mathbb Z /n\mathbb Z$ für $n=0,2,3$\\ 
 	\newline
 	Es existiert ein minimales $d > 0$ mit $g^d \in H$ für $G = g^{\mathbb Z}$. Dann gilt $d\mathbb Z \leq H$ in $\mathbb Z /n\mathbb Z$. 
 	Wir zeigen $d\mathbb Z = H$: 
 	
 	Sei ein $a \in H$, dann gilt $a\in r + d\mathbb Z$ mit $0 \leq r < d$ 
 	\begin{align*}
 		&\implies r \in H \\
 		&\implies r = 0 \\
 		&\implies a \in d\mathbb Z \\ 
 		&\implies H \text{ ist zyklisch}
 	\end{align*}
		Falls $n=0$ ist, dann $d\mathbb Z \cong \mathbb Z$, und falls $|\mathbb Z /n\mathbb | Z < \infty$ ist, also $n \geq 2$, 
		dann ist $d\cdot \frac{n}{d} \equiv 0 \mod n$ und $d$ hat die Ordnung $\frac{n}{d}$. Also $|G/H|= d$

\subsection{Permutations Gruppen}
\subsubsection{Permutations Gruppe}
		Wir schreiben ab jetzt $[n] = \{1, \ldots, n\}$ für $n \in \mathbb N$.  \\ 
		\newline
		Sei $X$ eine Menge, dann ist die Permutationsgruppe zu $X$ wie folgt definiert 
		$$Perm(X)= \{ f: X \iso X \mid f \text{ ist eine bijektive Abbildung}\}$$

\subsubsection{Symetrische Gruppe}
Eine symetrische Gruppe $S_n$ wird gebildet durch $$S_n = Perm([n])$$ mit $\pi = (\pi(1), \ldots, \pi(n)) \in S_n$

\subsubsection{q-Zykel}
	Ein $\pi \in S_n$ heißt $q$-Zykel falls $$ \exists I \in \binom{[n]}{q} = \text{ Träger von $\pi$}$$ mit $\pi(i_j) = i_{(j+1)}, \, j \in \mathbb Z / q \mathbb Z$ und $\pi(m)= m$ $\forall m \in [n]  \setminus  I$. \\ 
	Wir schreiben dann für einen $q$-Zykel eine Folge $(i_1, \ldots, i_q) = (i_{p+1}, \ldots, i_q, i_1, \ldots, i_p)$

\subsubsection{Transposition}
	Ein $2$-Zykel heißt Transposition $$\mathcal T_n = \{ \tau \in S_n  \mid \tau = (i, j), \, 1 \leq i \leq j \leq n\}$$
	Weiter gilt $$\binom{n}{k} = |\{ \pi \in S_n \mid \pi \text{ ist Produkt von $k$- Zykeln  mit paarweise verschiedenen disjunkten Trägern} \}|$$
\subsubsection{Graphische Interpretation von Zykeln}
	\begin{figure}[h!]
		\centering
		$ \pi = c_1, \ldots, c_k = \begin{array}{ccc}
		\begin{tikzpicture}[->,scale=.7] 
		\foreach \a/\t in {90/.,-30/.,210/.}{
			\node (\t) at (\a:1cm) {$\bullet$};
			\draw (\a-20:1cm)  arc (\a-20:\a-100:1cm);
		} 
		\end{tikzpicture}
	\end{array} \ldots
	\begin{array}{ccc}
	\begin{tikzpicture}[->,scale=.7] 
	\foreach \a/\t in {90/.,-90/.}{
		\node (\t) at (\a:1cm) {$\bullet$};
		\draw (\a-20:1cm)  arc (\a-20:\a-160:1cm);
	} 
	\end{tikzpicture}
	\end{array}
	\ldots
	\begin{array}{ccc}
	\begin{tikzpicture}[->,scale=.7] 
	\foreach \a/\t in {90/.}{
		\node (\t) at (\a:1cm) {$\bullet$};
		\draw (\a-20:1cm)  arc (\a-20:\a-340:1cm);
	} 
	\end{tikzpicture}
	\end{array}$
	\caption{$c_1$ bis $c_k$}
	\end{figure}

\subsubsection{Bubblesort-Lemma}
		 $\tau_n$ erzeugt $S_n = Perm(\{1, \ldots, n\})$. Oder anders fomuliert: \\
		 Die symetrische Gruppe $S_n$ wird durch $n-1$ Transpositionen $(i, i+1)$ erzeugt.
	 	\newline
	 	
	 	Den Beweis liefert der Bubblesort Algorithmus! 
		Genauer gilt: $S_n$ wird erzeugt von $\tau \in T_n$ mit $\tau= (i, i+1)$ für $1 \leq i < n$
		\newline
			
		Sei $\pi$ eine beliebige permutation in $S_n$. Verwendet man nun den Bubblesort Algorithmus zum sortieren durch die Transposition $(i, i+1$ wobei $1 \leq i <n $ gilt, so erhält man die Identität. Man führe nun den Algorithmus rückwärts aus, starte also bei der Identität, so kommt man zum gewünschten Ergebnis.

\subsubsection{Signum}
 	Sei $\pi \in S_n$. Dann ist das Signum wie folgt definiert $$sign(\pi) = \prod_{1 \leq i \leq j \leq n} \frac{\pi(j) - \pi(i)}{j -i}$$

\subsubsection{Signum - Betrags - Lemma}
	Es gilt: $|sign(\pi) |= 1$

\subsubsection{Beweis des Lemmas}
	Die Aussage gilt da 
	\begin{align*}
		sign(\pi) &= \ddfrac{\prod_{1 \leq i \leq j \leq n}  |\pi(j) - \pi(i)|}{\prod_{1 \leq i \leq j \leq n}  |j -i|} = \ddfrac{ \prod_{ \{i,j\} \in \binom{[n]}{2}} |\pi(j) - \pi(i)| }{ \prod_{ \{i,j\} \in \binom{[n]}{2}} |j - i|} = 1
	\end{align*}
	Sei weiter $F(\pi) = $ Menge der Fehlstellungen $= (i,j)$ mit $(\pi(j) - \pi(i))(j-i) \leq -1$. \\
	
	Dann ist $sign(\pi)= 1^{|F(\pi)|}$ und somit ist $sign(id) = +1$ und $sign(\tau) = \frac{j-i}{i-j} = -1$ für $\tau= (i,j) \in T_n$
	\bewiesen

\subsubsection{Signum - Hom. - Lemma}
	$$ sign: S_n \to \{ \pm 1\} \text{ ist ein Hom.}$$

\subsubsection{Beweis des Lemmas}
	Es gilt 
	\begin{align*}
		sign(\pi\sigma) = \prod_{ i < j} \ddfrac{\pi\sigma(j) - \pi \sigma(i)}{j -i} &=  \ddfrac{\pi\sigma(j) - \pi \sigma(i)}{\sigma(j) -\sigma(i)} \cdot \prod_{ i < j}\ddfrac{\pi\sigma(j) - \pi \sigma(i)}{j -i} = sign(\pi) sign(\sigma)
	\end{align*}
	\bewiesen
	Weiter gilt $ker(sign) = A_n = $ alternerirende Gruppe über $n$ und $A_n \trianglelefteq S_n$ und $|A_n| = \binom{n!}{2}$ für $n \geq 2$ 

\section{Gruppentheorie 3 - Vorlesung 4}
\subsection{Semidirekte Produkte}
\subsubsection{Direktes Produkt}
	Für zwei Gruppen $H$ und $F$ ist das direkte Produkt die Gruppe $H \times F$ mit  $$(h,f)(h', f') = (hh', ff')$$

\subsubsection{Semidirektes Produkt}
	$G = H \rtimes F$ ist ein semi direktes Produkt, falls $H \trianglelefteq G$, $F \leq G$,  $G = H\cdot F$ und $H \cap F = \{1\}$ mit 
	$$(h,f)(h',f') = hfh'(f^{-1}f)f' = (h\varphi(f)(h'), ff')$$
	mit $\varphi \in Aut(H)$ wobei $Aut(H)$ die Menge der Automorphismen in $H$ ist.

\subsubsection{Konstruktion 1 von semidirekten Produkten}
	Seien $H, F$ Gruppen und $\varphi: F\to Aut(H)$ ein Hom. \\ 
	Wir nehmen $G = H \times F$ als Menge und $$ (h,f)(h',f') = (h\varphi(f)h', ff')$$
	Dann ist $H= H \times \{1\}$ und $H \trianglelefteq H \times F$. 
	Weiter ist dann $F = \{1\} \times F \leq F$ und $G= H\cdot I$ und $H\cap F = \{1\}$

\subsubsection{Korrektheit der Konstruktion 1}
	Es ist $(h_1, f_1)(h_2, f_2)(h_3, f_3) = (h, f_1f_2f_3)$ mit 
	\begin{align*}
		h&=h_1\varphi(f_1)[h_2\varphi(f_2)(h_3)] \\
		&=h_1\varphi(f_1)(h_2)\varphi(f_1f_2)(h_3)
	\end{align*} 
	\\ 
	Und für die Inversen ist $(h, f)^{-1} = (h', f^{-1}) \text{ mit } h\varphi(f)(h')= 1$ aber $h' = \varphi(f^{-1})(h^{-1})$

\subsubsection{Exakte Sequenzen}
Eine Sequenz von Homs. $\varphi_i: G_{i-1} \to G_i$ mit $i \in I$ wobei $I$ linear geordnet ist, heißt exakt falls $$\forall i \in I : im(\varphi_i) = ker(\varphi_{i+1})$$ gilt
	\begin{figure}[h!]
		\centering
	\begin{equation*}
		\begin{tikzcd}[column sep=3.5pc]
				\ldots \arrow{r}{\varphi_{i-1}} & G_{i -1} \arrow{r}{\varphi_{i}} & G_{i} \arrow{r}{\varphi_{i+1}} & G_{i+1} \ldots 
		\end{tikzcd}
	\end{equation*}
		\caption{Graphische Darstellung einer Sequenz}
\end{figure}

\subsubsection{Konstruktion 2 von semidirekten Produkten}
Eine zweite Möglichkeit zur Konstruktion von semidirekten Produkten ist durch sogenannte kurze exakte Sequenzen möglich. 
Wir betrachten dabei die folgende Sequenz
\begin{figure}[h!]
	\centering
	\begin{equation*}
	\begin{tikzcd}[column sep=2.5pc]
	1 \arrow{r}{} & H \arrow{r}{\varphi} & G \arrow{r}{\psi} & F \arrow{r}{} & 1
	\end{tikzcd}
	\end{equation*}
	\caption{Kurze exakte Sequenz}
\end{figure} 

Dabei gilt, dass $\varphi : H \to G$ injektiv ist und dass $\psi : G \to F$ surjektiv ist. Somit entsteht der folgende Isomorphismus $$\overline{\psi} : G/H \iso F$$

\subsubsection{Satz über Konstruktion 2}
$G$ ist ein semidirektes Produkt $G = H\rtimes F$ mit der exakten Sequenz $$1 \to H \to G \to F \to 1$$ falls ein Hom. $s: F \to G$ existiert  mit $\psi s = id_F$

\section{Gruppentheorie 4 - Vorlesung 5}
\subsection{Einfache Gruppen}
\subsubsection{Definition einfach}
	Eine Gruppe $G$ heißt einfach wenn sie nur zwei Normalteiler besitzt, also nur $1$ und $G$ selbst.

\subsubsection{Beispiele für einfache Gruppen}
 \begin{itemize}
 	\item $\mathbb Z / p \mathbb Z$ für eine Primzahl $p$ ist einfach 
 	\item $G$ ist einfach und nicht abelsch, falls $[G, G] =1$
 \end{itemize}

\subsection{Kommutative Untergruppen}
\subsubsection{Definition kommutative Untergruppe}
	Für eine Gruppe $G$ ist $[G, G] = \langle [g, h] \mid g, h \rangle$ mit $[g, h] = g^{-1}h^{-1}gh$ die kommutative Untergruppe. 

\subsubsection{Satz über Kommutativität}
	Für eine Gruppe $G$ gilt $$ G \text{ abelsch } \iff [G, G] = 1$$

\subsubsection{Beweis des Satzes}
	Es gilt $g^{-1}h^{-1}gh = 1 \iff gh = hg$ also folgt $G$ ist abelsch. 
	\bewiesen

\subsubsection{Definition von auflösbar}
	Jede abelsche Gruppe ist auflösbar. Jede nicht abelsche Gruppe ist auflösbar falls die ''abgeleitete'' Untergruppe $[G, G]$ von $G$ verschieden ist und auflösbar. 
	\begin{equation*}
		1 \leq \ldots \leq G^{(2)} = [G^{(1)}, G^{(1)} ] \leq G^{(1)} = [G, G ] \leq G  
	\end{equation*}

\subsection{Alternierende Gruppen}
\subsubsection{Erzeugungs Lemma}
	Für $n \geq 5$ wird $A_n$ von der Menge der doppelten Transpositionen oder der Menge der $3$-Zykel erzeugt.

\subsubsection{Beweis des Lemmas}
	Wir wissen dass $\pi \in A_n \iff sign(\pi) = 1$ gilt, und somit auch dass $A_n$ von einem Produkt $\tau \tau'$ mit $\tau, \tau' \in \mathcal T_n$ erzeugt wird, 
	da $S_n$ von $\mathcal T_n$ erzeugt wird. 
	\newline 
	
	Die Menge der $3$-Zykel erzeugt $A_n$ durch 
	\begin{align*}
		(1,2)(3,4) = (1,2,3)(3,4,5) \text{ und } (1,2)(2,3) = (1,2,3)
	\end{align*}
	Mit $n\geq 5$ können wir schreiben 
	\begin{align*}
		(1,2, 3) = (1,2)(2,3) =((1,2)(4,5))((4,5)(2,3))
	\end{align*}
	
	Deshalb kann man $(1,2,3)$ als Produkt von zwei Permutationen des Typ $(2,2)$ schreiben.
	\bewiesen

\subsubsection{Lemma}
Sei $n \geq 5$ und $\sigma$ ein $3$-Zykel, dann existiert ein $\pi \in A_n$, sodass $\pi\sigma\pi^{-1} = (1,2,3)$

\subsubsection{Beweis des Lemmas}
	Man sieht deutlich, dass $\pi \in S_n$ existiert, sodass $\pi\sigma\pi^{-1} = (1,2,3)$. 
	Wenn $\pi \in S_n$ gilt, dann sind wir bereits fertig. 
	\newline
	
	Anderenfalls definieren wir $\pi' = (4,5)$ und $\pi \in A_n$.  Wir schließen daraus 
		\begin{align*}
			\pi'\sigma\pi'^{-1} = (4,5)\pi\sigma\pi^{-1}(4,5) = (4,5)(1,2,3)(4,5) = (1,2,3)
		\end{align*}
		\bewiesen
		
\subsubsection{Lemma}
	Sei $n \geq 3$ und $K \trianglelefteq S_n$, dann ist $K \in \{ \{1\}, A_n, S_n\}$
	
\subsubsection{Beweis des Lemmas}
	Wir nehmen an dass $\{1\} \not = K$. Sei $N= K \cap A_n$, dann ist $N$ ein Normalteiler in $S_n$. Es reicht an dieser Stelle aus, zu zeigen dass dies $N=A_n$ impliziert. \\ 
	Tatsächlich gilt $N = A_n \implies K \in \{ A_n, S_n\}$, da $[S_n : A_n] = 2$ ist.
	\newline
	
	Um nun zu erkennen dass $N=A_n$ ist, wählen wir $\sigma \in N$ mit $I \not = \sigma(i)$. Da $\sigma$ keine Transposition ist, gibt es kein $j \in \{i, \sigma(i)\}$ sodass $j \not=\sigma(j)$ gilt. \\
	
	Sei nun $\tau = (i, j)$ und $\rho = (\sigma(i), \sigma(j))$. 
	Wir behaupten dann dass $\rho\tau = \sigma\tau\sigma^{-1}\tau$ gilt, und $\sigma\tau\sigma^{-1}\tau \in N$ ist. Um die Behauptnung nun zu zeigen, beweisen wir $\rho\sigma = \sigma\rho$. 
	\begin{align*}
		\rho\sigma(j) &= \sigma(i) = \sigma\tau(i) \text{ und } \rho\sigma(i) = \sigma(j) = \sigma\tau(i) \\
		\rho\sigma(k) &= \sigma(k) = \sigma\tau(k) \text{ für } k \not \in \{i, j\}
	\end{align*}
	
	Wenn nun $\sigma(j) = i$ gilt, dann ist $\rho\tau \in N$ ein $3$-Zykel und somit $N=A_n$. 
	Anderenfalls ist $\rho\tau \in N$ vom Typ $(2,2)$. Alle Permutationen des Typ(2,2) sind in $S_n$ konjugiert. Daher haben wir wieder $N=A_n$
	\bewiesen
	
\subsubsection{Lemma}
	Sei $N \trianglelefteq A_n$ und $\pi \in S_n$, dann ist auch $\pi N\pi^{-1} \trianglelefteq A_n$. Genauer sogar $\tau N \tau \trianglelefteq A_n$ für alle $\tau \in \mathcal T_n$
\subsubsection{Beweis des Lemmas}
	Zuerst ein direkter Beweis: \\
	Für $\pi \in A_n$ ist bereits alles bekannt. Es reicht also zu zeigen dass $\sigma\rho N\tau\sigma^{-1} = \tau N \tau$ für $\sigma \in A_n$ und $\tau\in S_n \setminus A_n$. Dies ist äquivalent zu $ \tau \sigma \tau N \tau \sigma^{-1}\sigma \subseteq N$, da $\tau\sigma\tau \in A_n$ und $N\trianglelefteq A_n$. 
	\newline
	
	Als zweites eine elegantere Version: \\
	Wir haben $N \trianglelefteq H \trianglelefteq G$ mit $H = A_n$ und $G= S_n$. Für alle $\pi \in G$ erhalten wir einen Automorphismus $\varphi \in Aut(H)$ durch $\varphi(H) = \pi h\pi^{-1}$. Nun ist $\varphi(N) = \pi N\pi^{-1}$ ein Normalteiler in $H$, da $\varphi^{-1}(h) \in H$ ist. 
	Deshalb gilt: 
	\begin{align*}
		h\varphi(N)h^{-1} = \varphi(\varphi^{-1}(h)N\varphi^{-1}(h^{-1})) = \varphi(N)
	\end{align*}
	
	Die allgemeine Aussage ist, dass $N \trianglelefteq H$ und $\varphi \in Aut(H)$ und somit $\varphi(N) \trianglelefteq H$ gilt. 
	\bewiesen

\subsubsection{Satz über die Einfachheit von alternierenden Gruppen}
	Für $n \geq 5$ ist $A_n$ einfach. (C.Jordan, 1875).

\subsubsection{Strategie zum Beweis des Satzes}
	$\tau \in \mathcal T_n$ soll eine feste Transposition sein. Wir zeigen durch Widerspruch dass $A_n$ einfach ist. Dazu nehmen wir an, dass ein $N \trianglelefteq A_n$ existiert, sodass $\{1\} \triangleleft N \triangleleft A_n$ gilt.  \\
	
	Wir wissen, dass $S_n$ nur die Normalteiler $\{1\}, A_n$ und $S_n$ hat, und dadurch können wir schließen, dass $\tau N\tau \not = N$ ist, da sonst $N = \tau N \tau $ ein Normalteiler in $S_n$ wäre. Wir erinnern uns daran, dass $\tau N \tau \trianglelefteq A_n$ gilt. \\
	Durch folgende Schritte können wir dann einen Widerspruch herleiten: 

\begin{enumerate}[label=\arabic*.]
	\item Der Schnitt ist trivial: $N \cap \tau N \tau = \{1\}$
	\item Das Produkt $N(\tau N \tau)$ ist ein Normalteiler in $S_n$, da $N \subseteq N(\tau N \tau) \leq A_n$. \\ 
		Dies implieziert $N(\tau N \tau) = A_n$.
	\item Da $N\cap \tau N \tau =  \{1\}$ und $N(  \tau N \tau ) = A_n$ ist, schließen wir, dass $n! = 2 |N|^2$ ist. Sogar, dass $4 \mid n$ und somit das $|N|$ gerade sein muss. 
 	\item Der Widerspruch entsteht entweder dadurch dass wir zeigen dass $\frac{n!}{2}$ keine Quadratzahl ist, mit der Hilfe des Bertrandschen Postulates, oder durch den Satz von Cauchy
\end{enumerate}
	Die Punkte $1$ -$3$ werden durch die folgenden Lemma (\ref{lemma1}, \ref{lemma2}, und \ref{lemma3}) gezeigt. 

\subsubsection{Lemma \label{lemma1}}
	$N \cap \tau N \tau = \{1\}$, also dass der Schnitt trivial ist.
\subsubsection{Beweis des Lemma \ref{lemma1}}
		$N \cap \tau N \tau$ ist ein Normalteiler in $A_n$. Jedes $\pi \in S_n \setminus A_n$ kann auch als $\pi = \tau\sigma$ mit $\sigma \in A_n$ geschrieben werden.
		Wir betrachten 
		\begin{align*}
			\pi N \pi^{-1} &= \tau \sigma N \sigma^{-1} \tau = \tau N \tau  \\ 
			\pi \tau N \tau \pi^{-1} &= N
		\end{align*} 
		da $\pi\tau \in A_n$ ist. 
		Deshalb ist $\pi(\tau N \tau \cap N) \pi^{-1} \subseteq N \cap \tau N \tau$ und $\tau N \tau \cap N$ ist Normalteiler in $S_n$. Da $S_n$ ganz in $A_n$ liegt, folgt daraus dass $\tau N \tau \cap N = \{1\}$. 
		\bewiesen

\subsubsection{Lemma \label{lemma2}}
		Das Produkt $N(\tau N \tau)$ ist ein Normalteiler in $S_n$, da $N \subseteq N(\tau N \tau) \leq A_n$. \\ 
		Dies impliziert $N(\tau N \tau) = A_n$.
\subsubsection{Beweis des Lemma \ref{lemma2}}
		Nach Blatt 1, Aufgabe 3 der Übungen ist $N(\tau N \tau) \trianglelefteq A_n$. Ferner ist für $\pi = \sigma\tau \in S_N \setminus A_n$ mit $\sigma \in A_n$ 
		\begin{align*}
			\pi N(\tau N\tau) \pi^{-1} &= \sigma \tau N (\tau N \tau) \tau \sigma^{-1}  \\ 
			&= \sigma N \tau N \tau \sigma^{-1} \\
			&= N(\tau N \tau)
		\end{align*}
	\bewiesen

\subsubsection{Lemma \label{lemma3}}
	Es gilt $N\cap \tau N \tau =  \{1\}$ und $N(  \tau N \tau ) = A_n$. Also muss $n! = 2 |N|^2$ ein. Sogar, $4 \mid n$ und somit $|N|$ gerade . 
\subsubsection{Beweis des Lemma \ref{lemma3}}
Sei $\sigma, \sigma' \in N$ und $\pi', \pi \in \tau N \tau$, sodass $\sigma\pi' = \sigma'\pi$ ist. Dann ist $\sigma^{-1}\sigma' = \pi'\pi^{-1} \in N \cap \tau N \tau = \{1\}$. Dadurch ist $\sigma = \sigma'$ und $\pi' = \pi$. Somit ergibt sich 
\begin{align*}
	|A_n| = |N\tau N\tau| = |\tau N \tau| = |N|^2
\end{align*}
Also ist $|A_n| = n!$ eine Quadratzahl. Die Kardinalität von $N$ ist gerade da $n!$ duch $4$ teilbar ist für $n \geq 4$. 
\bewiesen

\subsubsection{Beweis des Satzes}

\bewiesen


\end{document}