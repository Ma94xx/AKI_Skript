\documentclass[12pt, german]{article}
\usepackage[ngerman]{babel}
\usepackage[T1]{fontenc}  
\usepackage[utf8]{inputenc}
\usepackage{amsmath}
\usepackage{dsfont}
\usepackage{array}
\usepackage{amssymb}
\usepackage{enumitem}
\usepackage{upgreek}
\usepackage{graphicx}
\usepackage{pdfpages}
\usepackage{listings}  
\usepackage{mathtools}
\usepackage{listings}
\usepackage{endnotes}
\usepackage{color}
\usepackage{tasks}
\usepackage{mathtools}
\usepackage{forest}
\usetikzlibrary{backgrounds,fit}
\usepackage[outline]{contour}
\usetikzlibrary{shapes.multipart}
\usetikzlibrary{shapes,fit}
\usetikzlibrary{fit}
\usepackage{wrapfig}
\usepackage{hyperref}
\usepackage{mathrsfs}
\usepackage{tikz-cd}
\usepackage[bottom]{footmisc}
\usepackage{booktabs}
\usetikzlibrary{decorations,decorations.markings,decorations.text}
\usepackage{tabularx}
\usetikzlibrary{arrows}
\usepackage{float}
\usepackage{dsfont}
\usepackage{relsize}
\usepackage{ stmaryrd }
\usepackage{multirow,bigdelim}
\usepackage{stackengine}
\usepackage{amsmath}

\usepackage{array, makecell}

\usepackage{pst-node}



\stackMath


\hypersetup{
	colorlinks,
	citecolor=black,
	filecolor=black,
	linkcolor=black,
	urlcolor=black
}

%Für Binärbaum
\usepackage{tikz}
\tikzset{
	treenode/.style = {align=center, inner sep=0pt, text centered, font=\upshape},
	arn_b/.style = {treenode, circle, black, font=\upshape, draw=black, fill=okmama, text width=1.5em, thick},
	arn_x/.style = {treenode, rectangle, draw=black, minimum width=0.5em, minimum height=0.5em}
}


%Zum runden
\DeclarePairedDelimiter{\ceil}{\lceil}{\rceil}

%Document Feineinstellungen
\usepackage[a4paper, left=2cm, right=2cm, top=2.5cm]{geometry}

%Deaktiviert Seitenummerierung
%\pagenumbering{gobble}

%Farben für Einstellungen der Code Fragmente
\definecolor{mygreen}{rgb}{0,0.6,0}
\definecolor{mygray}{rgb}{0.5,0.5,0.5}
\definecolor{okmama}{RGB}{150,170,120}
\definecolor{character}{RGB}{179, 178, 255}

%Einstellungen für Code Fragmente
\lstset{ 
	backgroundcolor=\color{white},   % choose the background color; you must add \usepackage{color} or \usepackage{xcolor}; should come as last argument
	%	basicstyle=\footnotesize,        % the size of the fonts that are used for the code
	%	breakatwhitespace=false,         % sets if automatic breaks should only happen at whitespace
	%	breaklines=true,                 % sets automatic line breaking
	captionpos=b,                    % sets the caption-position to bottom
	commentstyle=\color{mygreen},    % comment style
	deletekeywords={...},            % if you want to delete keywords from the given language
	escapeinside={(*}{*)},          % if you want to add LaTeX within your code
	extendedchars=true,              % lets you use non-ASCII characters; for 8-bits encodings only, does not work with UTF-8
	frame=single,	                   % adds a frame around the code
	keepspaces=true,                 % keeps spaces in text, useful for keeping indentation of code (possibly needs columns=flexible)
	keywordstyle=\color{blue},       % keyword style
	language=Octave,                 % the language of the code
	morekeywords={*, Algorithm, ...},            % if you want to add more keywords to the set
	numbers=left,                    % where to put the line-numbers; possible values are (none, left, right)
	numbersep=5pt,                   % how far the line-numbers are from the code
	numberstyle=\tiny\color{mygray}, % the style that is used for the line-numbers
	rulecolor=\color{black},         % if not set, the frame-color may be changed on line-breaks within not-black text (e.g. comments (green here))
	showspaces=false,                % show spaces everywhere adding particular underscores; it overrides 'showstringspaces'
	showstringspaces=false,          % underline spaces within strings only
	showtabs=false,                  % show tabs within strings adding particular underscores
	stepnumber=1,                    % the step between two line-numbers. If it's 1, each line will be numbered
	stringstyle=\color{mymauve},     % string literal style
	tabsize=2,	                   % sets default tabsize to 2 spaces
	title=\lstname                   % show the filename of files included with \lstinputlisting; also try caption instead of title
}


%Für Klammern neben Align Zeilen
\makeatletter

\newcommand*{\rbracedalign}[5][c]{%
	\sbox0{\m@th$\begin{aligned}[b]#3\end{aligned}$}%
	\sbox1{\m@th$\vcenter{\hrule \@width\z@ \@height\ht0 \@depth\dp0}$}%
	\sbox0{\raise\dimexpr\dp1-\dp0\relax\hbox{\m@th$\left. \box1 \right\rbrace #5$}}%
	\dp0=\z@ \ht0=\z@
	\begin{aligned}[#1]
		\if\relax\detokenize{#2}\relax\expandafter\@gobble\else\expandafter\@firstofone\fi{#2&\\}
		#3& \box0
		\if\relax\detokenize{#4}\relax\expandafter\@gobble\else\expandafter\@firstofone\fi{\\#4&}
	\end{aligned}
}

\makeatother



%Global Align
\newcommand*{\LongestName}{\ensuremath{h(x)+g(x)}}% function name
\newcommand*{\LongestValue}{\ensuremath{-1}}% function value
\newcommand*{\LongestText}{fermentum fringilla mauris }%

%größere Brüche
\newcommand\ddfrac[2]{\frac{\displaystyle #1}{\displaystyle #2}}

\newlength{\LargestNameSize}%
\newlength{\LargestValueSize}%
\newlength{\LargestTextSize}%

\settowidth{\LargestNameSize}{\LongestName}%
\settowidth{\LargestValueSize}{\LongestValue}%
\settowidth{\LargestTextSize}{\LongestText}%

% Choose alignment of the various elements here: [r], [l] or [c]
\newcommand*{\MakeBoxName}[1]{{\makebox[\LargestNameSize][r]{\ensuremath{#1}}}}%
\newcommand*{\MakeBoxValue}[1]{\ensuremath{\makebox[\LargestValueSize][l]{\ensuremath{#1}}}}%
\newcommand*{\MakeBoxText}[1]{\makebox[\LargestTextSize][l]{#1}}%

%Isomorph zeichen
\newcommand\iso{\xrightarrow{
		\,\smash{\raisebox{-0.65ex}{\ensuremath{\scriptstyle\sim}}}\,}}

%Mathematische Zeichen
\newcommand{\R}{\mathbb{R}}
\newcommand{\Z}{\mathbb{Z}}
\newcommand{\N}{\mathbb{N}}
\newcommand{\B}{\mathbb{B}}
\newcommand{\sigstern}{\Sigma^\ast}
\newcommand{\inv}{^{-1}}
\newcommand{\pom}{^{\omega}}
\newcommand{\rat}{\mathsf{RAT}}
\newcommand{\rec}{\mathsf{REC}}
\newcommand{\starfree}{\mathsf{SF}}
\newcommand{\synt}{\mathsf{Synt}}
\newcommand{\wop}{\mathsf{WP}}
\newcommand{\equivWP}{\equiv_{\mathsf{WP}(G)}}


\newcommand{\grel}{\sim_{\mathcal{L}}}
\newcommand{\grer}{\sim_{\mathcal{R}}}
\newcommand{\grej}{\sim_{\mathcal{J}}}
\newcommand{\greh}{\sim_{\mathcal{H}}}
\newcommand{\gred}{\sim_{\mathcal{D}}}
\newcommand{\grey}{\sim_{\mathcal{Y}}}

\newcommand{\lgreleq}{\leqslant_{\mathcal{L}}}
\newcommand{\lgrereq}{\leqslant_{\mathcal{R}}}
\newcommand{\lgrejeq}{\leqslant_{\mathcal{J}}}
\newcommand{\lgreheq}{\leqslant_{\mathcal{H}}}
\newcommand{\lgredeq}{\leqslant_{\mathcal{D}}}
\newcommand{\lgreyeq}{\leqslant_{\mathcal{Y}}}
\usepackage{blkarray}

\newcommand{\lcal}{\mathcal L}
\newcommand{\rcal}{\mathcal R}
\newcommand{\hcal}{\mathcal H}
\newcommand{\jcal}{\mathcal J}
\newcommand{\dcal}{\mathcal D}

\newcommand{\rgreleq}{\geqslant_{\mathcal{L}}}
\newcommand{\rgrereq}{\geqslant_{\mathcal{R}}}
\newcommand{\rgrejeq}{\geqslant_{\mathcal{J}}}
\newcommand{\rgreheq}{\geqslant_{\mathcal{H}}}
\newcommand{\rgredeq}{\geqslant_{\mathcal{D}}}
\newcommand{\rgreyeq}{\geqslant_{\mathcal{Y}}}
\newcommand{\aast}{A^{\ast}}
\newcommand{\bast}{B^{\ast}}
\newcommand{\east}{E^{\ast}}

%MSO
\newcommand{\fv}{\mathsf{FV}}
\newcommand{\fo}{\mathsf{FO}}
\newcommand{\so}{\mathsf{SO}}
\newcommand{\mso}{\mathsf{MSO}}
\newcommand{\reg}{\mathsf{REG}}
\newcommand{\ap}{\mathsf{AP}}
\newcommand{\ltl}{\mathsf{LTL}}
\newcommand{\tl}{\mathsf{TL}}

%LTL
\newcommand{\sX}{\mathsf{X}}
\newcommand{\sF}{\mathsf{F}}
\newcommand{\sG}{\mathsf{G}}
\newcommand{\sT}{\mathsf{T}}
\newcommand{\sY}{\mathsf{Y}}
\newcommand{\sS}{\mathsf{S}}
\newcommand{\sU}{\mathsf{U}}




%Erzeugt das "bewiesen" Kästen rechts unten
\newcommand{\bewiesen}{
	
	\begin{flushright}
		$\square$  \\
\end{flushright}}


%\newcommand\TBox[3][]{%
%	\tikz\node[draw,thick,text width=#2,#1] {#3};}


\title{Inoffizielles Skript für Algebra \& Kombinatorik}

\author{Maximilian Kurz \and Amelie Heindl}
\setlength\parindent{0pt}

\begin{document}
	\maketitle
	\newpage
	\tableofcontents
	\newpage
	
	\section{Grundlagen - Vorlesung 1}
	\subsection{Diedergruppen}
	\subsubsection{Definition}
	Die Diedergruppe $D_n$ ist die Bewegungsgruppe des regelmäßigen $n$-Ecks mit $n \geq 3$. Dabei ist die Gruppenverknüpfung die Drehung oder Spiegelung des $n$-Ecks. \\ 
	
	Nun ist das $n$-Eck durch folgende Knoten und Kanten definiert: 
	\begin{align*}
		\text{Knotenmenge: } &\mathbb{Z}/n\mathbb{Z} \\
		\text{Kanten: } &\{ \{i, i+1\} \mid i \in \mathbb{Z}/n\mathbb{Z} \}  \\
	\end{align*}
	
	Weiter sind die Drehungen und Spiegelungen bijektive Abbildung von Knoten auf Knoten der Form
	\begin{align*}
		&f: \mathbb{Z}/n\mathbb{Z} \iso \mathbb{Z}/n\mathbb{Z},\, \text{ wobei }\\
		&f(i+1) \in \{f(i)+1, f(i)-1\}\, \text{ gilt}
	\end{align*}
	
	Zu jeden $n \in \mathbb{N}$ gehört eine Diedergruppe $D_n$ mit $|D_n| = 2n$. 
	Diese $2n$ Elemente sind Äuivalenzklassen aus Sequenzen von $\delta s$ und $\sigma s$.
	Wie in \ref{sec:diederEigenschaften} zu sehen sein wird, führt bei $\sigma$ die Kongruenz modulo $2$ und bei $\delta$ die Kongruenz modulo $n$ zum selben Bild, weshalb es gesamt $2n$ verschiedene Bilder gibt.
	
	
	Folglich lassen sich Diedergruppen so definieren, dass sie aus der Menge $\{\Sigma_{i=1}^{k}\alpha_i \, | \, \alpha_i \in \{\sigma, \delta\}, k \in \N\}$ besteht und als Verknüpfung die Konkatenation hat, also die Hintereinanderausführung der Abbildungssequenzen.
	
	
	\subsubsection{Eigenschaften von Diedergruppen}
	\label{sec:diederEigenschaften}
	Wir bezeichnen die Spiegelung mit $\sigma$ und die Drehung im Uhrzeiger mit $\delta$. Dann gilt Folgendes:
	
	\begin{align*}
		\sigma(j)&=-j\\
		\delta(j)&=j+1\\
		\sigma^2 = id &= \delta^n = \delta^0 \\ 	 
	\end{align*}
	Dies bedeutet, dass $n$-faches drehen oder doppeltes Spiegeln eines  $n$-Ecks keine Änderung bewirkt.  \\
	Daraus lässt sich dann folgender Satz ableiten mit $\forall \varepsilon, \varepsilon', m, m' \in \mathbb{Z}$: 
	
	\begin{align*}
		\sigma^\varepsilon\delta^m = \sigma^{\varepsilon'}\delta^{m'} \iff \varepsilon \equiv \varepsilon' (2) \text{ und } m \equiv m' (n)
	\end{align*}
	Weiter gilt, dass $\sigma\delta = \delta^{-1}\sigma$, also dass erst spiegeln und dann drehen denselben Effekt hat wie in die andere Richtung drehen und dann spiegeln. Im Folgenden wird gezeigt, weshalb dies gilt:
	
	\begin{align*}
		\sigma \delta(j) = \sigma(j+1) = -(j + 1) = -j -1 = \delta^{-1}(-j)=\delta^{-1}\sigma(j)
	\end{align*}
	\bewiesen
	Dieser Satz gibt uns also die Möglichkeit das $\sigma$ nach rechts oder links zu schieben und wir können die Gleichheit $\sigma\delta\sigma = \delta^{-1}$ folgern.
	
	\subsubsection{Fixpunkte in Diedergruppen}
	Ein Fixpunkt $x$ einer Abbildung $f$ ist ein Punkt, für den $f(x) = x$ gilt. Da die Elemente der Diedergruppe durch $\sigma\delta^i(j)$ dargestellt werden können, fragen wir uns, wann folgende Gleichheit gilt:
	\begin{align*}
		\sigma\delta^i(j) = j
	\end{align*} 
	
	Wegen
	\begin{align*}
		\sigma\delta^i(j)  = j &\iff \sigma(j+1)  = j  \\
		&\iff 	-j -i  = j \\
		&\iff 	-i  = 2j
	\end{align*} 
	ist dies der Fall, wenn $-i = 2j \textrm{ in } \mathbb Z /n \mathbb Z$ gilt. Für ungerades $n$ existiert für alle $i$ genau ein $j$, welches diese Gleichheit erfüllt. Für gerades $n$ unterscheiden wir weiter nach der Parität von $i$. Falls auch $i$ gerade ist, existieren genau $2$ verschiedene $j$ und falls $i$ ungerade ist, existiert keine Lösung. Damit ergeben sich bestimmte Fixpunkte für die Drehungen und Spiegelungen, welche in Tabelle~\ref{tab:diederFixpunkte} aufgezeigt sind.
	\begin{table*}[h]
		\centering
		\begin{tabular}{lll}
			\toprule[2pt]
			Abbildung & Parität von $n$ & Fixpunkte\\
			\midrule
			$\sigma$ & ungerade & $\{i\}$ \\
			\addlinespace
			$\sigma$ & gerade & $ \left \{
			
			%Lässt mehr Abstand zwischen den Zeilen, erhöht lesbarkeit denke ich :) 
			\arraycolsep=1.4pt\def\arraystretch{1.5}
			\begin{array}{ll}
				\{0,\frac{n}{2}\}\textrm{, falls }i \equiv 0 \, (2)\\
				\emptyset\textrm{, falls } i \equiv 1\, (2)
			\end{array}
			\right. $ \\
			\addlinespace
			$\delta^k$ & gerade oder ungerade & $ \left \{
			\arraycolsep=1.4pt\def\arraystretch{1.5}
			\begin{array}{ll}
				\emptyset\textrm{, falls }1\leq k < n\\
				\Z/n\Z\textrm{, falls } k \equiv 0\, (n)
			\end{array}
			\right. $ \\
			\bottomrule[2pt]
		\end{tabular}
		\caption{Fixpunkte der Abbildungen in Diedergruppen.}
		\label{tab:diederFixpunkte}
	\end{table*}
	
	\subsubsection{Erweiterung auf unendliche Gruppen}
	
	Im Fall $n=0$ ist das $n$-Eck die unendliche gerade $\Z/0\Z = \Z$.
	
	Durch Einführen eines weiteren Zeichens $\overline{\delta}$ für negative Potenzen von $\delta$ gelten dieselben Rechenregeln wie im endlichen Fall.
	
	Dann gilt $D_0 = \{\sigma, \delta, \overline{\delta}\}^*$ mit
	\begin{align*}
		\sigma^i(j) & = j+i &&\delta(j)  = -j \\
		\sigma^2 & = 1 &&\delta\overline{\delta} = \overline{\delta}\delta = 1 \\
		\sigma\delta\sigma & = \overline{\delta}
	\end{align*}
	
	\section{Gruppentheorie 1 - Vorlesung 2}
	\subsection{Untergruppen}
	\subsubsection{Definition}
	Sei im folgenden $G=(G, \cdot, 1)$ eine Gruppe. Dann gilt 
	\begin{align*}
		H \subseteq G \text{ ist eine Untergruppe} \iff H \not = \emptyset \text{ und } \forall g, h \in H : gh^{-1} \in H
	\end{align*}
	$H$ ist also eine Untergruppe von $G$ genau dann, wenn $H$ nicht leer ist, abgeschlossen ist und für alle Elemente ein Inverses enthält. Aus diesen Bedingungen folgt direkt, dass dann auch $1 \in H$ gilt.
	Für Untergruppen ist die Schreibweise $H\leq G$ üblich.
	
	
	\subsection{Nebenklassen}
	\subsubsection{Definition}
	Für eine Untergruppe $H \leq G$ sind die folgenden Nebenklassen definiert:
	\begin{align*}
		G/H &= \{gH \subseteq G \mid g \in G\} \quad \text{Menge der Linksnebenklassen } \\
		H/G &= \{Hg \subseteq G \mid g \in G\} \quad \text{Menge der Rechtsnebenklassen } 
	\end{align*}
	wobei $gH$ die Multiplikation von  g mit allen Elementen aus $H$ ist, also die Linksnebenklasse bezüglich dem Element $g$. 
	
	\subsubsection{Grö\ss e von Nebenklassen}
	Wir definieren die folgenden Abbildungen $	\forall f,g \in G$ ist
	\begin{align*}
		(fg^{-1} \cdot): gH &\iso fH \\
		gh &\mapsto fh = fg^{-1}gh
	\end{align*}
	Diese ist bijektiv mit der Umkehrabbildung $(gf^{-1}\cdot)$.
	Wir können dann folgern, dass $\forall f,g \in G$
	\begin{align*}
		gH \iso H \iso Hf 
	\end{align*} 
	Es gilt also $|gH| = |Hf| = |H|\text{ für alle } f,g \in G$.
	
	\subsubsection{Gleichheit und Disjunktheit von Nebenklassen}
	Sowohl $G/H$ als auch $H/G$ sind jeweils Partitionen von $G$.
	Um das zu beweisen, wird im Folgenden gezeigt, dass $gH \cap fH \not = \emptyset \iff gH = fH$ gilt.
	Die Rückrichtung ist hierbei trivial, da der Schnitt von $gH$ und $fH$ bei Gleichheit offensichtlich nicht leer ist, sofern die Nebenklassen an sich nicht leer sind, was per Definition gegeben ist. \\
	
	Für den Beweis der Hinrichtung seinen $h_1, h_2 \in H$ mit $gh_1 = fh_2$
	\begin{align*}
		&\implies g=fh_2h_1^{-1} \in fH \\
		&\implies gH \subseteq fH. 
	\end{align*}
	Analog dazu kann man $fH \subseteq gH$ zeigen, womit $gH = fH$ folgt.
	\bewiesen
	
	\subsubsection{Graphische Interpretation von Nebenklassen}
	Sei $G= (G, \cdot, 1)$ eine Gruppe und $H\leq G$. 
	In Abbildung \ref{fig:nebenklassen} sind mögliche Partitionen von $G$ dargestellt.
	Dies veranschaulicht, dass Links- und Rechtsnebenklassen (i.A. verschiedene) gleichmä\ss ige Partitionen von $G$ sind.
	\begin{figure}[H]
		\centering
		$G=\begin{array}{c}
			
			\begin{tikzpicture}[every fit/.style={inner sep=0pt, outer sep=0pt, draw}]
				\begin{scope}[yshift=1.5cm,y=1cm]
					\node [fit={(0,0) (1,3)}, label=center:{$H$}] {};
				\end{scope}
				\begin{scope}[yshift=1.5cm,y=1cm]
					\node [fit={(1,0) (4,1)}, label=center:{$fH$}] {};
				\end{scope}
				\begin{scope}[yshift=2.5cm,y=1cm]
					\node [fit={(1,0) (4,1)}, label=center:{$\vdots$}] {};
				\end{scope}
				\begin{scope}[yshift=3.5cm,y=1cm]
					\node [fit={(1,0) (4,1)}, label=center:{$gH$}] {};
				\end{scope}
				
			\end{tikzpicture} 
		\end{array}
		= 
		\begin{array}{c}
			\begin{tikzpicture}[every fit/.style={inner sep=0pt, outer sep=0pt, draw}]
				\begin{scope}[yshift=1.5cm,y=1cm]
					\node [fit={(0,0) (1,3)}, label=center:{$H$}] {};
				\end{scope}
				\begin{scope}[yshift=1.5cm,y=1cm]
					\node [fit={(1,3) (2,0)}, label=center:{$Hg$}] {};
				\end{scope}
				\begin{scope}[yshift=1.5cm,y=1cm]
					\node [fit={(2,0) (3,3)}, label=center:{$\ldots$}] {};
				\end{scope}
				\begin{scope}[yshift=1.5cm,y=1cm]
					\node [fit={(3,3) (4,0)}, label=center:{$Hf$}] {};
				\end{scope}
				
			\end{tikzpicture} 
		\end{array}$
		\caption{Zerlegung von $G$ in Links- bzw. Rechtsnebenklassen}
		\label{fig:nebenklassen}
	\end{figure}
	
	Wir sehen, dass $|H|=|gH| = |Hf|$ gilt.
	
	\subsection{Repräsentantensysteme}		
	\subsubsection{Definition}
	Ein Repräsentantensystem soll für eine Klasseneinteilung aus jeder Klasse genau ein Element enthalten.
	Auf Linksnebenklassen bezogen, wählt man ein Element aus jeder Klasse $gH \in G/H$ (analog für Rechtsnebenklassen).
	Dieses Element bezeichnen wir mit $r(gH)$ für die Klasse $gH$. 
	Dann ist $R_H := \{r(gH) \mid gH \in G/H\} \subseteq G$ ein Repräsentantensystem.
	Mit der bijektiven  Abbildung
	\begin{align*}
		R_H &\iso G/H \\
		r(gH) &\mapsto gH 
	\end{align*}
	kann man sehen, dass das Repräsentantensystem offensichtlich gleichmächtig wie die Menge der Klassen ist.
	\subsection{Satz von Lagrange}	
	\subsubsection{Definition}
	Für eine Gruppe $G$ mit Untergruppe $H$ gilt:
	\begin{align*}
		|G| = [G:H] \cdot |H|\, ,
	\end{align*}	
	wobei $[G:H]$ als Index von $H$ in $G$ bezeichnet wird und die Anzahl der Nebenklassen von $H$ in $G$ angibt. Für den Index gilt $[G:H] = |G/H| = |H/G|$, da man für ein Linksrepresentantensystem $R = \{r(gH) \mid g \in G \}$, wegen $H=H^{-1}$ und $(rH)^{-1} = H^{-1}r^{-1} = Hr^{-1}$, ein Rechtsrepräsentantensystem
	$R^{-1} = \{r(gH)^{-1} \mid g \in G \}$ definieren kann und somit $|G/H| = |H/G|$ , da gilt.
	
	\subsubsection{Beweis}
	Wir definieren die folgende Abbildung
	\begin{align*}
		\lambda: (G/H) \times H &\iso G \\
		(gH, h) &\mapsto r(gH) \cdot h
	\end{align*}
	und wollen zeigen, dass diese bijektiv ist, woraus der Satz diekt folgt.
	Die Injektivität ist hierbei trivial, da die Nebenklassen disjunkt sind und man mit unterschiedlichen Nebenklassen oder unterschiedlichen Elementen aus $H$ somit nicht dasselbe Element in $G$ über $\lambda$ erreichen kann. Die Surjektivität ist auch trivial, da die Klasseneinteilung eine Partition von $G$ ist.
	\bewiesen	
	
	\subsection{Der Homomorphiesatz}		
	\subsubsection{Grundlagen: Homomorphismus}
	Seien $G,F$ Gruppen. 
	Eine Abbildung $\varphi: G \to F$ hei\ss t Homomorphismus\footnote{ab hier abgekürzt Hom.}, falls $\forall g,h \in G : \varphi(gh) = \varphi(g)\varphi(h)$ gilt. Daraus folgt:\\
	\begin{enumerate}[label=\roman*)]
		\item $\varphi(1_G) = 1_F$ ~\par
		Denn $g=1_G\cdot g \implies \varphi(g)=\varphi(1_G\cdot g)=\varphi(1_G)\varphi(g) \implies \varphi(1_G) = \varphi(g) \varphi(g)^{-1} = 1_F$
		
		\item $\varphi(g^{-1})=\varphi(g)^{-1}$ ~\par
		Denn $\varphi(1_G)=\varphi(gg^{-1}) \implies 1_F =\varphi(gg^{-1})= \varphi(g)\varphi(g^{-1}) \implies \varphi(g)^{-1} = \varphi(g^{-1})$
	\end{enumerate}
	
	\subsubsection{Lemma: Injektive Homomorphismen}
	Ein Hom. $\varphi: G \to F$ ist injektiv $\iff$ $ker(\varphi) = \{g \in G \mid \varphi(g)= 1\} = \{1\}$ 
	
	Die Richtung $"\implies "$ ist trivial, da definiert ist, dass das Einselement auf das Einselement abgebildet wird und bei einer injektiven Abbildung kein weiteres Element aus $G$ auf dasselbe Element in $F$ abgebildet werden kann.
	Im Folgenden beweisen wir die zweite Richtung unter Benutzung der Äquivalenz $\varphi(g) = \varphi(h) \iff \varphi(gh^{-1}) = \{1\} $:
	\begin{align*}
		ker(\varphi) &= \{1\} \\
		&\implies [\varphi(gh^{-1}) = \{1\} \implies gh^{-1} = 1 \implies g = h] \\ 
		&\implies \varphi \text{ ist injektiv}
	\end{align*}
	
	\subsubsection{Definition Homomorphiesatz}
	Sei $\varphi: G \to F$ ein Hom.  und $N = ker(\varphi)$. \\
	Dann induziert $\varphi$ einen injektiven Hom. $\overline{\varphi} : G/N \to F$ durch $\overline{\varphi}(gN) = \varphi(g)$.
	\newline
	
	Genauer induziert $\varphi$ also einen Isomorphismus $\overline{\varphi}: G/N \iso im(G) = \{\varphi(g) \mid g \in G\} \leq F$.
	
	\subsubsection{Beweis Homomorphiesatz}
	\begin{enumerate}[label=\arabic*)]
		\item $\overline{\varphi}$ ist wohldefiniert ~\par
		Sei $f\in gN \implies \varphi(f) \in \varphi(g)N = \{\varphi(g)\}$
		
		\item $\overline{\varphi}$ ist injektiv ~\par
		$ker(\overline{\varphi}) = ker(\varphi) = N = 1_{G/N}$
	\end{enumerate}
	
	\subsubsection{Beispiel für den Homomorphiesatz}
	Wir definieren 
	\begin{align*}
		exp: &\,\mathbb{R} \to \mathbb{C}^\ast\\
		&\, x \mapsto e^{2\pi i x}
	\end{align*}
	Dann ist $ker(exp) = \mathbb Z $.\\ 
	Also ist $\mathbb R / \mathbb Z \cong \{z \in \mathbb C \mid |z| = 1\}$.
	
	
	
	\subsection{Normalteiler}		
	\subsubsection{Definition}
	$H \leq G$ hei\ss t Normalteiler, falls $\forall g \in G : gH = Hg$ gilt.
	Man schreibt dann $H\trianglelefteq G$. 
	
	\subsubsection{Satz zu Normalteilern}
	Sei $H \subseteq G$. Dann sind die folgenden Aussagen äquivalent: 
	\begin{enumerate}[label=\arabic*)]
		\item $H\trianglelefteq G$
		\item $H \leq G$ und $gH \cdot fH = gfH$ definiert eine Gruppenstruktur auf $G/H$
		\item Es existiert ein Hom. $\varphi: G \to F$ mit $ker(\varphi) = \{g \in G \mid \varphi(g)= 1\} = H$
		\item $\forall g \in G : gHg^{-1} \subseteq H$ und $H \leq G$
	\end{enumerate}
	
	Im Folgenden wird die Äquivalenz der Aussagen gezeigt: 
	\begin{itemize}
		\item $1)\implies 2)$ ~\par
		$H\trianglelefteq G \implies gHfH = g(Hf)H = g(fH)H = gfHH = gfH$ mit
		\begin{align*}
			&\text{ Neutralem Element: } &&1_{G/H} = H \\
			&\text{ Inversen: } &&(gH)^{-1} = H^{-1}g^{-1}= Hg^{-1}=g^{-1}H \\
			&\text{ Assoziativität: } &&(fH)(gH)(hH)= fghH\\
		\end{align*}
		
		\item $2)\implies 3)$ ~\par
		Wir wählen $F=G/H$.
		Dann ist $\varphi : G \to G/H $, $\varphi(g) = gH$ ein Hom. mit
		\begin{align*}
			ker(\varphi) = \{g \in G \mid gH = H\} = H = 1_{G/H}
		\end{align*}
		
		\item $3)\implies 4)$~\par
		Sei $H = ker(\varphi)$, dann
		\begin{align*}
			\varphi(gHg^{-1}) = \varphi(gg^{-1}) = \varphi(1) \subseteq H 
		\end{align*}
		
		\item $4)\implies 1)$~\par
		$\forall g \in G : gHg^{-1} \subseteq H$
		\begin{align*}
			&\implies  \forall g \in G : H \subseteq g^{-1}Hg &| \text{ durch } g\cdot \text{ und } \cdot g^{-1}\\
			&\implies \forall g \in G : H \subseteq gHg^{-1} &| \text{ da jedes Elemet ein Inverses ist}\\
			&\implies gHg^{-1} = H \\
			&\implies gH = Hg
		\end{align*}
	\end{itemize}
	
	
	
	\subsection{G-Mengen}		
	\subsubsection{Definition}
	Sei $G = (G, *, 1)$ eine Gruppe und $X$ eine Menge.
	Dann operiert $G$ auf $X$ falls folgende Abbildung existiert 
	\begin{align*}
		\cdot : &G \times X \to X \\ 
		&(g,x) \mapsto gx \quad \text{(man schreibt auch g(x) statt (g,x))}
	\end{align*}
	sodass $\forall g,h \in G: g(h(x)) = (g*h)(x)$ und $(1,X) = id_X$ gilt.
	
	Die Menge $X$ zusammen mit der Operation von $G$ auf $X$ wird dann als G-Menge bezeichnet.
	
	\subsubsection{Homomorphismen auf G-Mengen}
	Ein Hom. von $G$-Mengen $X$ mit der Operation $\cdot$ und $Y$ mit der Operation $\star$ ist eine Abbildung 
	\begin{align*}
		f: X \to Y \text{ mit } f(g \cdot x) = g \star f(x)\, ,
	\end{align*}
	wobei $g \in G$ gilt.
	
	\subsubsection{Fixpunkte in G-Mengen}
	Sei $X$ eine  $G$-Menge. Dann hei\ss t t $x$ ein Fixpunkt von $g \in G$ falls $gx=x$. \\
	Weiter ist für alle $x \in X$ die Menge $Fix(x)$  der Fixpunkte eine Untergruppe von $G$:
	\begin{align*}
		Fix(x) = \{g \in G \mid  gx = x\} \leq G 
	\end{align*}
	
	
	\subsection{Das Bahnenlemma}		
	\subsubsection{Definition Bahn}
	Wir definieren die Bahn eines Elements $x \in X$ wie folgt: $$Bahn(x) = Orbit(x) = G\cdot x = \{gx \mid  g \in G \}$$
	
	
	Ebenfalls gilt, dass die Bahn eine G-Menge und isomorph zu $G/Fix(x)$ ist. 
	Dies kann man mit der folgenden Bijektion begründen:
	\begin{align*}
		G/Fix(x) \to Gx \\ 
		gFix(x) \mapsto gx
	\end{align*}
	Die Bijektion ist wohldefiniert, da für jedes Element $h \in gFix(x)$ gilt, dass $h=gf$ mit einem $f \in Fix(x)$. \\ 
	Also folgt $hx = gf(x) = g(x)$.\\
	Mit dem Wissen, dass $|Gx|=|G/Fix(x)|$ ein Teiler von $|G|$ ist, kann man insbesondere für den Spezialfall, dass $|G| = p$ für eine Primzahl $p$ gilt, folgern, dass $\forall x \in X : |Gx|=1$ $\vee\, $$|Gx|=p$ gelten muss.
	
	
	\subsubsection{Bahnenlemma}
	Das Bahnenlemma besagt $$Gx \cap Gy \not = \emptyset \iff Gx = Gy\, .$$
	Im Allgemeinen folgt also $$ X = \dot{\bigcup} \{Gx \mid  x \in X \}\, ,$$ wobei hier die disjunkte Vereinigung benutzt wird, welche durch den Punkt über dem Vereinigungszeichen gekennzeichnet wird. 
	\subsubsection{Beweis des Bahnenlemmas}
	Die Rückrichtung ist hier ebenfalls trivial (wie bei dem sehr ähnlichen Resultat über Nebenklassen). Wir zeigen deshalb nur die erste Richtung $"\implies"$. \\ 
	
	Sei $gx = hy$ für $g,h \in G$
	\begin{align*}
		&\implies h^{-1}gx = y \\
		&\implies y \in Gx \\
		&\implies Gy \subseteq Gx
	\end{align*}
	Analog dazu kann man $Gx \subseteq Gy$ zeigen, was insgesamt zu $Gx = Gy$ führt.
	
	\subsubsection{Graphische Interpretation einer Bahn}
	
	Sei $X$ eine $G$-Menge. Dann kann man sich die Zerlegung von $X$ durch die Operation von $G$ wie in Abbildung~\ref{fig:bahnen} dargestellt, vorstellen.
	
	\begin{figure}[H]
		\centering
		$X=\begin{array}{c}
			
			\begin{tikzpicture}[every fit/.style={inner sep=0pt, outer sep=0pt, draw}]
				\begin{scope}[y=1.5cm]
					\node [fit={(4,0) (7,1)}, label=center:{$Gx_1$}] {};
				\end{scope}		
				\begin{scope}[yshift=1.5cm,y=1cm]
					\node [fit={(0,-1.5) (4,1)}, label=center:{$Gx_2$}] {};
					\node [fit={(4,0) (7,1)}, label=center:{$Gx_3$}] {};
				\end{scope}
				\begin{scope}[yshift=2.5cm,y=1.2cm]
					\node [fit={(0,0) (5,1)}, label=center:{$Gx_4$}] {};
					\node [fit={(5,0) (7,1)}, label=center:{$Gx_5$}] {};
				\end{scope}
			\end{tikzpicture}
		\end{array}$
		\caption{Disjunkte Zerlegung von $X$ in Bahnen}
		\label{fig:bahnen}
	\end{figure}
	
	
	
	\subsection{Satz von Cauchy}		
	\subsubsection{Definition}
	Sei $G$ eine endliche Gruppe und $p \in \mathbb P$ mit $p \mid |G|$. \\
	Dann existiert $g \in G$ mit $g \not = 1$ und $ g^p = 1$. Also $ord(g)= p$. (Und $G$ enthält eine Untergruppe, die isomorph ist zu $\Z/p\Z$.)
	\subsubsection{Beweis}
	Wir betrachten die Menge der Tupel $(g_1,\ldots , g_p) \in G^p$ mit $g_1 \cdot \ldots \cdot g_p = 1$ und nennen diese Menge $T$. Also $T = \{(g_1, \ldots, g_p) \mid  g_1 \cdot \ldots \cdot g_p = 1 \}$. 
	Nun gilt
	\begin{align*}
		|T| &= |\{(g_1, \ldots, g_p) \mid  g_1 \cdot \ldots \cdot g_p = 1 \}| \\
		&= |\{(g_1, \ldots, g_{p-1}) \in G^{p-1} \}| \\
		&= |G|^{p-1}
	\end{align*}
	Nun soll $\mathbb Z /p\mathbb Z$ auf $T$ operieren durch
	\begin{align*}
		m \cdot (g_1,\ldots ,g_p) = (g_{1+m},\ldots ,g_{p+m})\, ,
	\end{align*}
	wobei die Indizes aus $\Z/p\Z$ sind. 
	Das entspricht einem zyklischen Shift um $m$ und lässt sich folgenderma\ss en veranschaulichen:.
	\begin{align*}
		t = (\underbrace{g_1, \ldots, g_m}_{\substack{u}}, \underbrace{g_{m+1}, \ldots, g_p}_{\substack{v}}) \text{ und } m\cdot t = (\underbrace{g_{m+1}, \ldots, g_p}_{\substack{v}}, \underbrace{g_1, \ldots, g_m}_{\substack{u}})
	\end{align*}
	Es gilt also $uv = 1 \iff v(uv)v^{-1} = 1 \iff vu = 1$.\\
	Weiter gilt $|G|^{p-1} = n^{p-1}$ und $p \mid n^{p-1} $, da $p \mid n$
	und Bahnen haben entweder die Länge $1$ oder $p$, da $\mathbb Z /p\mathbb Z$ nur zwei Untergruppen hat. 
	Damit folgt, dass die Menge $\{(g, \ldots, g) \mid g^p= 1 \}$ mindestens ein von $(1,\ldots, 1)$ verschiedenes Element enthalten muss und damit
	
	\begin{align*}
		\implies \exists g \text{ mit } g^p = 1
	\end{align*}
	\bewiesen
	
	\section{Gruppentheorie 2 - Vorlesung 3}
	\subsection{p-Gruppen}
	\subsubsection{Definition}
	Eine Gruppe $G$ hei\ss t $p$-Gruppe, falls $p$ eine Primzahl ist und $|G| \in p^{\mathbb N}$.
	
	\subsubsection{Zentrum einer Gruppe}
	Wir definieren das Zentrum einer Gruppe wie folgt: $$Zentrum(G)= Z(G) = C(G) = \{g \in G \mid  \forall x \in G: xg = gx\}$$
	
	Das Zentrum enthält also genau die Elemente, die mit allen anderen kommutieren.
	Falls $Z(G)=\{1\}$ gilt, nennt man dies das triviale Zentrum.
	Offensichtlich ist $Z(G)$ immer ein Normalteiler von $G$.
	
	\subsubsection{Satz: p-Gruppen haben kein triviales Zentrum}
	\begin{align*}
		G \text{ ist eine $p$-Gruppe} \implies Z(G) \not = \{1\}
	\end{align*}
	
	\subsubsection{Beweis des Satzes}
	Sei $G$ eine p-Gruppe. Da $xg=gx \iff x=gxg^{-1}$ gilt, sind die Elemente des Zentrums genau die Fixpunkte der Operation $x \mapsto gxg^{-1}$ von $G$ auf sich selbst. Diese Operation zerlegt $G$ in disjunkte Bahnen. Jede Bahn hat dabei die Grö\ss e $|G/Fix(x)|$. 
	Also $Fix(x) \not= G \implies p \mid |G/Fix(x)|$. Dann gilt: 
	\begin{align*}
		|G| &\equiv \sum_{x \in Z(G)} |G/Fix(x)|  \mod p\\
		&\equiv \sum_{x \in Z(G)} 1  \mod p\\
		&\equiv |Z(G) |\mod p
	\end{align*}
	Wir wissen, dass das Zentrum auf jeden Fall die $1$ enthält und, da das Zentrum kongruent modulo $p$ zu einer Gruppe mit $|G| \in p^{\N}$ sein muss, muss es also mindestens $p$ Elemente enthalten. Daraus folgt also $p \mid |Z(G)|$, was $Z(G) \neq \{1\}$ zur Folge hat. 
	\bewiesen
	
	
	Wir können daraus weiter schlie\ss en, dass auch $G/Z(G)$ eine $p$-Gruppe ist mit $Z(G/Z(G)) \not = \{1\}$ oder $G=Z(G)$.
	
	
	\subsection{Zyklische Gruppen}
	\subsubsection{Nilpotenz}
	Eine Gruppe $G$ hei\ss t nilpotent, falls es eine Kette von Gruppen $G_0,\ldots,G_m$ gibt, sodass
	\begin{align*}
		\langle 1 \rangle &= G_0, G_1, \ldots, G_m = G  \text{ und } G_k/Z(G) = G_{k-1} \text{ für } 1 \leq k \leq m \text{ gilt.}
	\end{align*}
	
	\subsubsection{Beispiele: Nilpotenz}
	\begin{itemize}
		\item Kommutative Gruppen sind immer nilpotent, da dann $Z(G)=G$ gilt und eine Kette mit $G_1=G$ die Voraussetzung für Nilpotenz erfüllt.
		\item $D_3$ mit $|D_3| = 6$ ist die kleinste, nicht triviale, nicht nilpotente Gruppe, da $Z(D_3) = \{1\}$ und somit $G/Z(G)=G$ gilt und es keine Kette von Gruppen geben kann, die bei $\{1\}$ endet.
		\item $D_4$ mit $|D_4| = 8$ ist nilpotent, da $D_4/Z(D_4) = \mathbb Z/2 \mathbb Z \times \mathbb Z /2\mathbb Z$.
	\end{itemize}
	
	\subsubsection{Zyklische Gruppen}
	Eine Gruppe $G$ hei\ss t zyklisch, falls $\exists g \in G$ mit $G=g^{\mathbb Z}$.
	
	\subsubsection{Satz: Endliche zyklische Gruppen}
	$|G| < \infty$ und $G$ zyklisch $\implies \exists g \in G$ mit $G =\{g^0, g^1, \ldots, g^{|G| -1} \}$  
	
	\subsubsection{Beweis des Satzes }
	Sei $|G|<\infty$ und $G$ zyklisch. Betrachte $g, g^2, g^3, \ldots$ für $g^{\mathbb Z}=G$. Da die Gruppe endlich ist, existieren $i,j$ mit $0 \leq i < j $, sodass $g^i = g^j$. Daraus ergibt sich dass $1 = g^j-g^i=g^{j-i}$ gilt. \\
	Betrachte nun ein minimales $n \in \mathbb N_+$ mit $g^n = g^0 = 1$ , welches demnach existieren muss (\OE  $\, \,G \not = \{1\}$).
	Da $n$ minimal ist, sind  $g^0, \ldots g^{n-1}$ paarweise verschieden. Ferner gilt $(g^i = g^j \iff i \equiv j \mod n) \implies G \cong (\mathbb Z /n\mathbb Z, +, 0)$ für ein $n \in \N$.
	\bewiesen
	
	\subsubsection{Satz: Zyklische Untergruppen}
	$H \leq G \wedge G$ ist zyklisch $\implies H$ ist zyklisch
	
	\subsubsection{Beweis des Satzes}
	\OE  $\, \, \{1\} \not = H \not = G$ und  $G \cong \mathbb Z /n\mathbb Z$ für $n=0,2,3,\ldots$\\ 
	\newline
	Es existiert ein minimales $d > 0$ mit $g^d \in H$ für $G = g^{\mathbb Z}$.
	Wir setzten nun $G=\Z/n\Z$. Dann gilt $d\mathbb Z \leq H$ in $\mathbb Z /n\mathbb Z$. 
	$d\Z$ ist offensichtlich zyklisch und wir zeigen nun $d\mathbb Z = H$, was dann beweist, dass auch $H$ zyklisch ist. 
	
	Sei ein $a \in H$, dann gilt $a\in r + d\mathbb Z$ mit $0 \leq r < d$ 
	\begin{align*}
		&\implies r \in H &| \text{ wegen Abgeschlossenheit}\\
		&\implies r = 0 &| \text{ da } d \text{ die kleinste Zahl in } H  \text{ ist}\\
		&\implies a \in d\mathbb Z \\ 
		&\implies H \text{ ist zyklisch}
	\end{align*}
	\bewiesen
	Falls $n=0$ ist, dann gilt $d\mathbb Z \cong \mathbb Z$. Falls $|\Z /n\Z| < \infty$ ist, also $n \geq 2$, 
	dann ist $d\cdot \frac{n}{d} \equiv 0 \mod n$ und $d$ hat die Ordnung $\frac{n}{d}$. Also $|G/H|= d$.
	
	\subsection{Permutationsgruppen}
	\subsubsection{Definition}
	
	Sei $X$ eine Menge, dann ist die Permutationsgruppe $Perm(X)$ zu $X$ wie folgt definiert:
	$$Perm(X)= \{ f: X \iso X \mid f \text{ ist eine bijektive Abbildung}\}$$
	Die Gruppenoperation ist dabei die Hintereinanderausführung.
	
	\subsubsection{Symmetrische Gruppe}
	Wir schreiben ab jetzt $[n]$ für $\{1, \ldots, n\}$ mit $n \in \mathbb N$.  \\ 
	\newline
	Eine symmetrische Gruppe $S_n$ enthält alle Permutationen der Zahlen von $1$ bis $n$. Es gilt also: $$S_n = Perm([n])$$ 
	Elemente $\pi \in S_n$ lassen sich darstellen als $\pi = (\pi(1), \ldots, \pi(n))$.
	Es gilt $|S_n|=n!$.
	
	\subsubsection{q-Zykel}
	Ein Element $\pi \in S_n$ hei\ss t $q$-Zykel, falls $$ \exists I \in \binom{[n]}{q}: I=\{i_1,\ldots, i_q\} \text{ mit } (\pi(i_j) = i_{(j+1)}\,\, \forall j \in \mathbb Z / q \mathbb Z) \wedge (\pi(m)= m \,\, \forall m \in [n]  \setminus  I)$$. \\ 
	Die Menge $I$ hei\ss t dann Träger von $\pi$.
	Durch $\pi$ werden also Elemente im Träger zyklisch permutiert, während alle übrigen Element unverändert bleiben.
	Wir können für einen $q$-Zykel die Folge $(i_1, \ldots, i_q) = (i_{p+1}, \ldots, i_q, i_1, \ldots, i_p)$ schreiben.\\
	
	Weiter definieren wir $$\left[\begin{array}{c}n\\k\end{array}\right] = |\{ \pi \in S_n \mid \pi \text{ ist Produkt von $k$ Zykeln  mit paarweise disjunkten Trägern} \}|$$
	als die Anzahl der disjunkten Möglichkeiten $[n]$ in $k$ Zykel zu zerlegen.
	
	
	\subsubsection{Transposition}
	Ein $2$-Zykel hei\ss t Transposition.
	Die Menge $\mathcal T_n$ der Transpositionen von $[n]$ ist damit definiert durch  $$\mathcal T_n = \{ \tau \in S_n  \mid \tau = (i, j), \, 1 \leq i \leq j \leq n \}\,.$$
	
	\subsubsection{Graphische Interpretation von Zykeln}
	Abbildung~\ref{fig:zykel} veranschaulicht ein mögliches Element $\pi \in S_n$ als Produkt von Zykeln.
	
	\begin{figure}[H]
		\centering
		$ \pi = c_1\cdot \ldots\cdot c_k = \begin{array}{ccc}
			\begin{tikzpicture}[->,scale=.7] 
				\foreach \a/\t in {90/.,-30/.,210/.}{
					\node (\t) at (\a:1cm) {$\bullet$};
					\draw (\a-20:1cm)  arc (\a-20:\a-100:1cm);
				} 
			\end{tikzpicture}
		\end{array} \ldots
		\begin{array}{ccc}
			\begin{tikzpicture}[->,scale=.7] 
				\foreach \a/\t in {90/.,-90/.}{
					\node (\t) at (\a:1cm) {$\bullet$};
					\draw (\a-20:1cm)  arc (\a-20:\a-160:1cm);
				} 
			\end{tikzpicture}
		\end{array}
		\ldots
		\begin{array}{ccc}
			\begin{tikzpicture}[->,scale=.7] 
				\foreach \a/\t in {90/.}{
					\node (\t) at (\a:1cm) {$\bullet$};
					\draw (\a-20:1cm)  arc (\a-20:\a-340:1cm);
				} 
			\end{tikzpicture}
		\end{array}$
		\caption{Zerlegung von $\pi$ in $c_1$ bis $c_k$.}
		\label{fig:zykel}
	\end{figure}
	
	\subsubsection{Bubblesort-Lemma}
	Das Bubblesort-Lemma besagt $\mathcal T_n$ erzeugt $S_n = Perm(\{1, \ldots, n\})$. Oder genauer formuliert, gilt: $S_n$ wird erzeugt von $\tau \in \mathcal T_n$ mit $\tau= (i, i+1)$ für $1 \leq i < n$.
	\newline
	
	Den Beweis liefert der Bubblesort Algorithmus, da die Transpositionen als Bubblesort-Schritte gesehen werden können. 
	\newline
	Sei $\pi$ eine beliebige Permutation in $S_n$. Verwendet man nun den Bubblesort Algorithmus zum Sortieren, werden immer zwei Elemente $(i, i+1)$ mit $1 \leq i <n $ vertauscht, was einer Transposition entspricht. Am Ende des Sortierens erhält man die Identität. Wenn man nun die gemachten Schritte rückwärts ausführt, also bei der Identität startet, so kommt man zu allen Permutationen, also zum gewünschten Ergebnis.
	
	\subsubsection{Signum}
	Sei $\pi \in S_n$. Dann ist das zugehörige Signum $sign(\pi)$ wie folgt definiert $$sign(\pi) = \prod_{1 \leq i < j \leq n} \frac{\pi(j) - \pi(i)}{j -i}\, .$$
	
	Sei weiter $F(\pi) = \{(i,j)\, |\, (\pi(j) - \pi(i))(j-i) \leq -1\}$ die Menge der Fehlstellungen von $\pi$. 
	Es hängt also von $F(\pi)$ ab, ob $sign(\pi)$ positiv oder negativ ist, da jede Fehlstellung einen negativen Faktor zu $sign(\pi)$ beiträgt.
	
	\subsubsection{Lemma: Signumsbetrag}
	Es gilt $|sign(\pi) |= 1$.
	
	\subsubsection{Beweis des Lemmas}
	Die Aussage ist wahr, da 
	\begin{align*}
		sign(\pi) &= \ddfrac{\prod_{1 \leq i < j \leq n}  |\pi(j) - \pi(i)|}{\prod_{1 \leq i < j \leq n}  |j -i|} = \ddfrac{ \prod_{ \{i,j\} \in \binom{[n]}{2}} |\pi(j) - \pi(i)| }{ \prod_{ \{i,j\} \in \binom{[n]}{2}} |j - i|} = \ddfrac{ \prod_{ \{i,j\} \in \binom{[n]}{2}} |j - i| }{ \prod_{ \{i,j\} \in \binom{[n]}{2}} |j - i|} = 1
	\end{align*}
	\bewiesen
	
	Dann folgt nun $sign(\pi)= 1^{|F(\pi)|}$. 
	Somit ist $sign(id) = +1$ und $sign(\tau) = \frac{j-i}{i-j} = -1$ für $\tau= (i,j) \in \mathcal T_n$.
	
	
	\subsubsection{Lemma: $sign$ ist ein Hom.}
	$$ \text{Die Abbildung }sign: S_n \to \{ \pm 1\} \text{ ist ein Homomorphismus.}$$
	
	\subsubsection{Beweis des Lemmas}
	Seien $\pi, \sigma \in S_n$. Es gilt 
	\begin{align*}
		sign(\pi\sigma) &= \prod_{1 \leq i < j \leq n} \ddfrac{\pi\sigma(j) - \pi \sigma(i)}{j -i} = \prod_{1 \leq i < j \leq n} \ddfrac{\pi\sigma(j) - \pi \sigma(i)}{\sigma(j) -\sigma(i)} \cdot \prod_{1 \leq i < j \leq n}\ddfrac{\sigma(j) - \sigma(i)}{j -i}\\
		&= \prod_{1 \leq i < j \leq n} \ddfrac{\pi(j) - \pi (i)}{j -i} \cdot \prod_{1 \leq i < j \leq n}\ddfrac{\sigma(j) - \sigma(i)}{j -i} = sign(\pi) sign(\sigma)\, .
	\end{align*}
	\bewiesen
	Weiter bezeichnen wir $ker(sign) = A_n$ als die alternierende Gruppe über $n$.
	Es gilt $A_n \trianglelefteq S_n$ und $|A_n| = \binom{n!}{2}$ für $n \geq 2$. 
	
	\section{Gruppentheorie 3 - Vorlesung 4}
	\subsection{Semidirekte Produkte}
	\subsubsection{Direktes Produkt}
	Für zwei Gruppen $H$ und $F$ ist das Gruppe $H \times F$ mit  der komponentenweisen Verknüfung $$(h,f)(h', f') = (hh', ff')$$ definiert als das direkte Produkt.
	
	\subsubsection{Semidirektes Produkt}
	Das semidirekte Produkt ist ein Spezialfall des direkten Produkts.
	$G = H \rtimes F$ ist ein semidirektes Produkt, falls $H \trianglelefteq G$, $F \leq G$,  $G = H\cdot F$ und $H \cap F = \{1\}$ gilt. 
	Die Gruppenoperation ist dabei 
	$$(h,f)(h',f') = hfh'(f^{-1}f)f' = h(fh'f^{-1})ff'  = (h\varphi(f)(h'), ff')$$
	mit $\varphi(f) \in Aut(H)$.
	$Aut(H)$ ist die Menge der Automorphismen in $H$.
	\newline
	$S_3 = \Z/3\Z \rtimes \Z/2\Z$ ist ein Beispiel für ein semidirektes Produkt.
	\subsubsection{Konstruktionsmöglichkeit 1 für semidirekte Produkte}
	Seien $H, F$ Gruppen und $\varphi: F\to Aut(H)$ ein Hom. \\ 
	Wir nehmen $G = H \times F$ als Menge und $$ (h,f)(h',f') = (h\varphi(f)h', ff')$$ als Verknüpfung.\\
	Dann ist $H= H \times \{1\}$ und damit $H \trianglelefteq H \times F$. 
	Weiter ist dann $F = \{1\} \times F \leq G$ und $G= H\cdot F$, sowie $H\cap F = \{1\}$.
	Gesamt sind also alle Bedingungen für ein semidirektes Produkt erfüllt.
	
	\subsubsection{Korrektheit der Konstruktionsmöglichkeit 1}
	Um die Korrektheit zu beweisen, muss gezeigt werden, dass $G$ eine Gruppe ist.
	Zuerst zeigen wir die Gültigkeit des Assoziativgesetzes für $G$.
	Es ist $(h_1, f_1)(h_2, f_2)(h_3, f_3) = (h, f_1f_2f_3)$ mit 
	\begin{align*}
		h&=h_1\varphi(f_1)[h_2\varphi(f_2)(h_3)] \\
		&=h_1\varphi(f_1)(h_2)\varphi(f_1f_2)(h_3)\\
		&=[h_1\varphi(f_1)(h_2)]\varphi(f_1f_2)(h_3)\, .
	\end{align*} 
	\\ 
	Das neutrale Element ist $(1,1)$.\\
	Nun bleibt die Existenz von Inversen zu zeigen.
	Sei $(h,f)\in G$. $(h, f)^{-1} = (h', f^{-1})$ mit $h' = \varphi(f^{-1})(h^{-1})$, weil $h\varphi(f)(h')= 1$ gelten muss.
		
	\subsubsection{Exakte Sequenzen}
	Eine Sequenz von Homs. $\varphi_i: G_{i-1} \to G_i$ mit $i \in I$ wobei $I$ eine linear geordnete Menge ist, hei\ss t exakte Sequenz, falls $$\forall i \in I : im(\varphi_i) = ker(\varphi_{i+1})$$ gilt.\\
	Abbildung~\ref{fig:sequenz} veranschaulicht die Idee von Hom. Sequenzen.
	\begin{figure}[h!]
		\centering
		\begin{equation*}
			\begin{tikzcd}[column sep=3.5pc]
				\ldots \arrow{r}{\varphi_{i-1}} & G_{i -1} \arrow{r}{\varphi_{i}} & G_{i} \arrow{r}{\varphi_{i+1}} & G_{i+1} \ldots 
			\end{tikzcd}
		\end{equation*}
		\caption{Graphische Darstellung einer Sequenz.}
		\label{fig:sequenz}
	\end{figure}

	\subsubsection{Kurze Exakte Sequenzen}
	Eine kurze exakte Sequenz ist eine exakte Sequenz der Form 
		\begin{equation*}
			\begin{tikzcd}[column sep=2.5pc]
				\{1\} \arrow{r}{} & H \arrow{r}{\varphi} & G \arrow{r}{\psi} & F \arrow{r}{} & \{1\}\, ,
			\end{tikzcd}
		\end{equation*}
	wobei $\varphi: H \mapsto G$ injektiv ist, $\psi: G \mapsto F$ surjektiv ist, und die induzierte Abbildung $\overline{\psi} : G/H \iso F$ ein Isomorphismus ist.
	
	\subsubsection{Konstruktionsmöglichkeit 2 für semidirekte Produkte}
	Eine zweite Möglichkeit zur Konstruktion von semidirekten Produkten entsteht durch den folgenden Zusammenhang mit exakten Sequenzen.\\
	$G$ ist ein semidirektes Produkt $G = H\rtimes F$ mit der exakten Sequenz 
	\begin{equation*}
		\begin{tikzcd}[column sep=2.5pc]
			\{1\} \arrow{r}{} & H \arrow{r}{\varphi} & G \arrow{r}{\psi} & F \arrow{r}{} & \{1\}\, ,
		\end{tikzcd}
	\end{equation*}
	falls ein Hom. $s: F \to G$ existiert  mit $\psi s = id_F$.
	
	\section{Gruppentheorie 4 - Vorlesung 5}
	\subsection{Einfache Gruppen}
	\subsubsection{Definition}
	Eine Gruppe $G$ hei\ss t einfach, wenn sie außer $\{1\}$ und $G$ selbst keine weiteren Normalteiler besitzt.
	
	\subsubsection{Beispiele}
	\begin{itemize}
		\item $\mathbb Z / p \mathbb Z$ mit $p \in \mathbb{P}$ ist einfach. 
		\item $\mathbb Z / n \mathbb Z$ mit $n \in \N \wedge n \notin \mathbb{P}$ ist nicht einfach. 
	\end{itemize}
	
	\subsection{Kommutate Untergruppen}
	\subsubsection{Definition}
	Für eine Gruppe $G$, ist $[G, G] = \langle [g, h] \mid g, h \in G \rangle$ mit $[g, h] = g^{-1}h^{-1}gh$ die kommutate Untergruppe. \\
	\newline
	Wenn $G$ einfach und nicht abelsch ist, dann gilt $[G, G] =1$. 
	
	\subsubsection{Satz: Kommutate Untergruppen abelscher Gruppen}
	Für eine Gruppe $G$ gilt: $$ G \text{ abelsch } \iff [G, G] = 1$$
	
	\subsubsection{Beweis des Satzes}
	Es gilt $g^{-1}h^{-1}gh = 1 \iff gh = hg \iff G$ ist abelsch. 
	\bewiesen
	
	\subsubsection{Definition: auflösbar}
	Jede abelsche Gruppe ist auflösbar. Jede nicht abelsche Gruppe ist auflösbar, falls die ''abgeleitete'' Untergruppe $[G, G]$ von $G$ verschieden und auflösbar ist. 
	Für eine auflösbare Gruppe, kann man also eine Folge der Form
	\begin{equation*}
		1 \leq \ldots \leq G^{(2)} = [G^{(1)}, G^{(1)} ] \leq G^{(1)} = [G, G ] \leq G  
	\end{equation*}
	finden.
	
	\subsection{Alternierende Gruppen}
	
	\subsubsection{Definition}
	Die Alternierende Gruppe $A_n$ für $n \in \N$ enthält genau diejenigen Elemente $\pi \in S_n$, für die $sign(\pi)$ positiv ist, also $$A_n=\{\pi\, | \, \pi \in S_n \wedge sign(\pi) = 1\}\, .$$
	\subsubsection{Lemma: Erzeugung von $A_n$}
	Für $n \geq 5$ wird $A_n$ sowohl von der Menge der Doppel-Transpositionen als auch von der Menge der $3$-Zykel erzeugt.\\
	Hierbei gilt, dass eine Doppel-Transposition das Produkt $(i,j)(k,l)$ zweier Transpositionen mit disjunkten Elementen $i,j,k,l$ ist.
	
	\subsubsection{Beweis des Lemmas}
	Wir wissen dass $\pi \in A_n \iff sign(\pi) = 1$ gilt und somit auch, dass $A_n$ von einem Produkt $\tau \tau'$ mit $\tau, \tau' \in \mathcal T_n$ erzeugt wird, 
	da $S_n$ von $\mathcal T_n$ erzeugt wird. 
	\newline 
	
	Die Menge der $3$-Zykel erzeugt $A_n$, da man alle $3$-Zykel als Doppel-Trasnposition darstellen kann.
	Es gilt 
	\begin{align*}
		(1,2)(3,4) = (1,2,3)(3,4,2) \text{ und } (1,2)(2,3) = (1,2,3)\, .
	\end{align*}
	Mit $n\geq 5$ können wir schreiben 
	\begin{align*}
		(1,2, 3) = (1,2)(2,3) =((1,2)(4,5))((4,5)(2,3))
	\end{align*}
	
	Deshalb kann man $(1,2,3)$ als Produkt von zwei Permutationen des Typ $(2,2)$ schreiben.
	\bewiesen
	
	\subsubsection{Lemma: $3$-Zykel sind konjugiert in $A_n$}
	Sei $n \geq 5$ und $\sigma$ ein $3$-Zykel. Dann existiert ein $\pi \in A_n$, sodass $\pi\sigma\pi^{-1} = (1,2,3)$.
	
	\subsubsection{Beweis des Lemmas}
	Man sieht deutlich, dass $\pi \in S_n$ existiert, sodass $\pi\sigma\pi^{-1} = (1,2,3)$. 
	Wenn $\pi \in A_n$ gilt, dann sind wir bereits fertig. 
	\newline
	
	Anderenfalls definieren wir $\pi' = (4,5)\pi \in A_n$.  Wir schlie\ss en daraus 
	\begin{align*}
		\pi'\sigma\pi'^{-1} = (4,5)\pi\sigma\pi^{-1}(4,5) = (4,5)(1,2,3)(4,5) = (1,2,3)\,,
	\end{align*}
	wodurch die Behauptung folgt.
	\bewiesen
	
	\subsubsection{Lemma: Normalteiler von $S_n$}
	Sei $n \geq 3$ und $K \trianglelefteq S_n$, dann ist $K \in \{ \{1\}, A_n, S_n\}$.
	
	\subsubsection{Beweis des Lemmas}
	Wir nehmen an, dass $\{1\} \not = K$ gilt. Sei $N= K \cap A_n$, dann ist $N$ als Schnitt zweier Normalteiler selbst ein Normalteiler in $S_n$. Es reicht an dieser Stelle aus, zu zeigen, dass dies $N=A_n$ impliziert, da $N = A_n \implies K \in \{ A_n, S_n\}$ dann folgt, wegen $[S_n : A_n] = 2$.
	\newline
	
	Um nun zu zeigen, dass $N=A_n$ gilt, wählen wir $\sigma \in N$ und $i$ mit $i \not = \sigma(i)$. Da $\sigma$ keine Transposition ist, gibt es noch ein $j \notin \{i, \sigma(i)\}$ sodass $j \not=\sigma(j)$ gilt. \\
	
	Sei nun $\tau = (i, j)$ und $\rho = (\sigma(i), \sigma(j))$. 
	Wir behaupten dann, dass $\rho\tau = \sigma\tau\sigma^{-1}\tau$ gilt, und $\sigma\tau\sigma^{-1}\tau \in N$ ist. Um diese Behauptung zu zeigen, beweisen wir $\rho\sigma = \sigma\tau$. 
	Es gilt:
	\begin{align*}
		\rho\sigma(j) &= \sigma(i) = \sigma\tau(j) \text{, } \rho\sigma(i) = \sigma(j) = \sigma\tau(i) \text{ und }\\
		\rho\sigma(k) &= \sigma(k) = \sigma\tau(k) \text{ für } k \not \in \{i, j\}\, .
	\end{align*}
	
	Wenn nun $\sigma(j) = i$ gilt, dann ist $\rho\tau \in N$ ein $3$-Zykel und somit in $A_n$, was zu $N=A_n$ führt. 
	Anderenfalls ist $\rho\tau \in N$ vom Typ $(2,2)$. Alle Permutationen des Typ(2,2) sind in $S_n$ konjugiert. Daher haben wir wieder $N=A_n$.
	\bewiesen
	
	\subsubsection{Lemma: Normalteiler von Normalteilern}
	Sei $N \trianglelefteq A_n$ und $\pi \in S_n$, dann ist auch $\pi N\pi^{-1} \trianglelefteq A_n$. Genauer gilt $\tau N \tau \trianglelefteq A_n$ für alle $\tau \in \mathcal T_n$.
	\subsubsection{Beweis des Lemmas}
	Zuerst ein direkter Beweis: \\
	Falls $\pi \in A_n$ gilt, ist die Behauptung direkt offensichtlich. Es reicht also zu zeigen, dass $\sigma\tau N\tau\sigma^{-1} = \tau N \tau$ für $\sigma \in A_n$ und $\tau\in S_n \setminus A_n$ gilt. Dies ist äquivalent zu $ \tau \sigma \tau N \tau \sigma^{-1}\tau \subseteq N$. Auch in diesem Fall folgt die Behauptung direkt, da $\tau\sigma\tau \in A_n$ und $N\trianglelefteq A_n$. 
	\newline
	
	Als zweites eine elegantere Version: \\
	Wir haben $N \trianglelefteq H \trianglelefteq G$ mit $H = A_n$ und $G= S_n$. Für alle $\pi \in G$ erhalten wir einen Automorphismus $\varphi \in Aut(H)$ durch $\varphi(h) = \pi h\pi^{-1}$. Nun ist $\varphi(N) = \pi N\pi^{-1}$ ein Normalteiler von $H$. Für alle $h \in H$ ist $\varphi^{-1}(h) \in H$. 
	Deshalb gilt: 
	\begin{align*}
		h\varphi(N)h^{-1} = \varphi(\varphi^{-1}(h)N\varphi^{-1}(h^{-1})) = \varphi(N)
	\end{align*}
	
	Die allgemeine Aussage ist, dass aus $N \trianglelefteq H$ und $\varphi \in Aut(H)$ folgt, dass $\varphi(N) \trianglelefteq H$ gilt. 
	\bewiesen
	
	\subsubsection{Satz: Einfachheit von alternierenden Gruppen}
	Für $n \geq 5$ ist $A_n$ einfach. (C.Jordan, 1875).
	
	\subsubsection{Strategie zum Beweis des Satzes}
	$\tau \in \mathcal T_n$ soll eine feste Transposition sein. Wir zeigen durch Widerspruch, dass $A_n$ einfach ist. Dazu nehmen wir an, dass ein $N \trianglelefteq A_n$ existiere, mit $\{1\} \triangleleft N \triangleleft A_n$.  \\
	
	Wir wissen, dass $S_n$ nur die Normalteiler $\{1\}, A_n$ und $S_n$ hat, und dadurch können wir schlie\ss en, dass $\tau N\tau \not = N$ ist, da sonst auch $N = \tau N \tau $ ein Normalteiler in $S_n$ wäre. Wir erinnern uns daran, dass $\tau N \tau \trianglelefteq A_n$ gilt. \\
	Durch folgende Schritte können wir dann einen Widerspruch herleiten: 
	
	\begin{enumerate}[label=\arabic*.]
		\item Der Schnitt von $N$ und $\tau N \tau$ ist trivial: $N \cap \tau N \tau = \{1\}$.
		\item Das Produkt $N(\tau N \tau)$ ist ein Normalteiler in $S_n$. Da $N \subseteq N(\tau N \tau) \leq A_n$, impliziert dies $N(\tau N \tau) = A_n$.
		\item Da $N\cap \tau N \tau =  \{1\}$ und $N(  \tau N \tau ) = A_n$ gilt, schlie\ss en wir, dass $n! = 2 |N|^2$ ist, und weiter, dass $4 \mid n$ gilt und $|N|$ somit gerade sein muss. 
		\item Der Widerspruch entsteht dadurch, dass wir entweder mit Hilfe des Bertrandschen Postulates ($\forall m \in \R\, \exists p \in \mathbb{P}: \, m < p \leq 2m$, sofern $m \geq 1$) zeigen, dass $\frac{n!}{2}$ keine Quadratzahl ist, oder den Satz von Cauchy benutzen. 
	\end{enumerate}
	Die Punkte $1$ -$3$ werden durch die folgenden Lemmata (\ref{lemma1}, \ref{lemma2}, und \ref{lemma3}) gezeigt. 
	
	\subsubsection{Lemma: Trivialer Schnitt \label{lemma1}}
	$N \cap \tau N \tau = \{1\}$
	\subsubsection{Beweis des Lemma \ref{lemma1}}
	$N \cap \tau N \tau$ ist ein Normalteiler in $A_n$. Jedes $\pi \in S_n \setminus A_n$ kann auch als $\pi = \tau\sigma$ mit $\sigma \in A_n$ geschrieben werden.
	Es gilt
	\begin{align*}
		\pi N \pi^{-1} &= \tau \sigma N \sigma^{-1} \tau = \tau N \tau  \\ 
		\pi \tau N \tau \pi^{-1} &= N
	\end{align*} 
	da $\pi\tau \in A_n$ ist. 
	Deshalb ist $\pi(\tau N \tau \cap N) \pi^{-1} \subseteq N \cap \tau N \tau$ und $\tau N \tau \cap N$ ist ein Normalteiler von $S_n$. Da $N$ ganz in $A_n$ liegt, folgt daraus, dass $\tau N \tau \cap N = \{1\}$. 
	\bewiesen
	
	\subsubsection{Lemma: $N(\tau N\tau)$ ist Normalteiler \label{lemma2}}
	Das Produkt $N(\tau N \tau)$ ist ein Normalteiler in $S_n$. Da $N \subseteq N(\tau N \tau) \leq A_n$, impliziert dies $N(\tau N \tau) = A_n$.
	\subsubsection{Beweis des Lemma \ref{lemma2}}
	Nach Blatt 1, Aufgabe 3 der Übungen ist $N(\tau N \tau) \trianglelefteq A_n$. Ferner ist für $\pi = \sigma\tau \in S_N \setminus A_n$ mit $\sigma \in A_n$: 
	\begin{align*}
		\pi N(\tau N\tau) \pi^{-1} &= \sigma \tau N (\tau N \tau) \tau \sigma^{-1}  \\ 
		&= \sigma\tau N \tau N  \sigma^{-1} \\
		&= \sigma N \tau N \tau \sigma^{-1} \\
		&= N(\tau N \tau)\, .
	\end{align*}
	\bewiesen
	
	\subsubsection{Lemma: Grö\ss e von $N$ \label{lemma3}}
	Es gilt $n! = 2 |N|^2$. Weiter gilt $4 \mid n$ und somit folgt, dass $|N|$ gerade ist. 
	\subsubsection{Beweis des Lemma \ref{lemma3}}
	Sei $\sigma, \sigma' \in N$ und $\pi', \pi \in \tau N \tau$, sodass $\sigma\pi' = \sigma'\pi$ ist. Dann ist $\sigma^{-1}\sigma' = \pi'\pi^{-1} \in N \cap \tau N \tau = \{1\}$. Dadurch ist $\sigma = \sigma'$ und $\pi' = \pi$. Somit ergibt sich 
	\begin{align*}
		|A_n| = |N\tau N\tau| = |N| \cdot |\tau N \tau| = |N|^2\, .
	\end{align*}
	Also ist $|A_n| =\frac{n!}{2}$ eine Quadratzahl. Die Kardinalität von $N$ ist gerade da $n!$ duch $4$ teilbar ist für $n \geq 4$. 
	\bewiesen
	
	\subsubsection{Beweis des Satzes mit dem Bertrandschem Postulat}
	Das Bertrandsche Postulat besagt, dass für $1 \leq m \in \mathbb R$ eine Primzahl $p$ mit $m < p \leq 2m$ existiert. 
	\newline
	Sei nun $m =\frac{n}{2}$, dann existiert eine Primzahl $p$ mit $\frac{n}{2} < p \leq n$. Da $n \geq 5$ ist, gilt $p > 2$. Somit gilt $p \mid \frac{n!}{2}$, aber $p^2 > 2p >  n$. Damit teilt $p$ aber nicht $n!$. Ein Widerspruch!
	\bewiesen
	
	\subsubsection{Beweis des Satzes mit dem Satz von Cauchy}
	Da $|N|$ eine nicht triviale Gruppe gerader Ordnung ist, besagt der Satz von Cauchy, dass $N$ ein Element $\sigma$ mit $ord(\sigma) = 2$ enthält. 
	Dann kann $\sigma$ als gerade Anzahl an 2-Zyklen geschrieben werden:
	\begin{align*}
		\sigma = c_0 \cdot \ldots \cdot c_k\, ,
	\end{align*} 
	wobei $k\geq 1$ gilt, $k$ gerade ist und $c_0, \ldots, c_k$ Zyklen der Ordnung $2$ mit paarweise disjunkten Trägern sind. Damit gilt $c_ic_j = c_jc_i$ für alle $i,j$. 
	Sei nun $\tau = c_0$. Dann ist $N \cap \tau N \tau = \{1\}$, da dies für alle $\tau \in \mathcal T_n$ gilt. Durch $\sigma\tau = \tau \sigma$ erhalten wir
	\begin{align*}
		\sigma = \tau\sigma\tau \in N\cap \tau N \tau\, .
	\end{align*} 
	Dies impliziert jedoch $\sigma = 1$. Ein Widerspruch! 
	
	\section{Gruppentheorie 5 - Vorlesung 6}
	\subsection{Kranzprodukte}
	\subsubsection{Vorraussetzungen}
	Seien $M = (M, +, 0)$ nicht notwendig kommutativ und $N=(N, \cdot, 1)$ beides Monoide, sowie $X$ eine Menge. 
	Dann operiert $N$ von rechts auf $X$. Wir betrachten die Menge der Abbildungen von $X$ nach $M$: $M^x = \{ f: X \to M \mid \text{ f ist eine Abbildung} \}$. Wir schreiben statt $f(x)$ nun $xf$. 
	Somit operiert $N$ auf $M^X$ von links durch
	\begin{align*}
		nf : X \to M \\ 
		x(n\cdot f) := (xn)f = xnf \, .
	\end{align*}
	\subsubsection{Definition Konstante Funktion}
	Wir definieren für alle $y$ die konstante Funktion $c_y$ durch	$$c_y(x) = xc_y = y \, \, \, \forall x\, .$$ 
	
	\subsubsection{Definition Kranzprodukt}
	Nun hei\ss t $M \wr N$ Kranzprodukt, falls $M^X \rtimes N$ gilt. Dabei ist das Kranzprodukt für Monoide mit der Eigenschaft
	\begin{align*}
		(f,m)(g,n) = (f + mg, mn) \text{ mit } x(f+mg) = xf + xmg \, \, \, \forall x
	\end{align*}
	definiert.
	Wir können also die Klammern ''vergessen''. Somit ist das Kranzprodukt wieder ein Monoid mit dem Neutralelement $1_{M \wr N}= (c_0,1)$.
	\\
	Im Folgenden zeigen wir noch die Assoziativität von $M \wr N$:
	\begin{align*}
		[(f,k)(g,m)](h,n) = (f +kg, km)(h,n) = (f +kg + kmh, kmn) \\
		(f,k)[(g,m)(h,n)] = (f,k)(g +mh, mn) = (f + k ( g +mh),kmn) 
	\end{align*}
	Nun bleibt zu zeigen, dass $\forall x : x(f+kg + kmh) \overset{!}{=} x(f+k(g+mh))$ gilt. \\
	Dies ist der Fall, wegen 
	\begin{align*}
		x(f+k(g+mh)) = xf+(xk)(g + mh) = xf + xkg + xkmh\, .
	\end{align*}
	\bewiesen
	Es kann auch $X=N$ gesetzt werden.
	
	\subsubsection{Satz: Kranzprodukt bei Gruppen}
	Wenn $M, N$ Gruppen sind, dann ist auch $M^X \rtimes N$ bzw. $M \wr N$ eine Gruppe. 
	
	\subsubsection{Beweis}
	Das Neutralelement $(c_0, 1)$ ist offensichtlich enthalten.\\
	Gesucht ist nun für alle $(f, m) \in M \wr N$ ein $(g,n)$ mit $(f, m)(g,n) = (c_0, 1)$.
	Damit ist in durch die zweite Komponente festgelegt, dass $n=m^{-1}$ ist und es bleibt zu zeigen, dass $(f,m)(g, m^{-1}) = (c_0,1)$  gilt. \\
	Wir wollen also, dass $\forall x \, : \,xf+xmg=0$ gilt. das ist gleichbedeutend mit $xf=-xmg$.
	Dafür setzen wir $xg = -xm^{-1}f$. Dann ist $$xf-xmm^{-1}f = xf-xf = 0\, .$$	\bewiesen
	
	\subsubsection{Satz: Kranzprodukte durch exakte Sequenzen}
	Wir betrachten erneut die exakte Sequenz 
	\begin{figure}[H]
		\centering
		\begin{equation*}
			\begin{tikzcd}[column sep=1.5pc]
				1 \arrow{r}{} & H \arrow{r}{} & G \arrow{r}{\psi} & F \arrow{r}{} & 1
			\end{tikzcd}
		\end{equation*}
	\end{figure} 
	von Gruppen.\\
	Sei $\{[x] \in G \mid Hx, \, \, x \in F\}$ ein Repräsentantensystem von $F$ mit $[1] = 1$. 
	Dann ist $$\varphi: G \to H \wr F = H^F \rtimes F$$ eine Einbettung von Gruppen mit $\varphi(g) = (f_g, \psi(g))$ und $f_g(x) = [x]g[xg]^{-1}$ für $x \in F$. \\  Insbesondere ist dann $$G \leq H^F \rtimes F\, .$$
	
	\subsubsection{Beweis}
	Es gilt $\psi[xg] = H\psi(xg) = Hx\psi(g)$. Daher ist
	\begin{align*}
		\psi([x]g[xg]^{-1}) = Hx\psi(g)(x\psi(g)^{-1}) = H
	\end{align*}
	und $H$ ist das Einselement in F.\\
	Dann bleibt zu zeigen, dass $\varphi$ ein Hom. ist, also, dass
	\begin{align*}
		\varphi((f_h, h)(f_g, g)) = \varphi(f_nhf_g, hg) 
	\end{align*}
	gilt. Wenn man die erste Komponente betrachtet, gilt
	\begin{align*}
		x(f_h \cdot hf_g) = [x]h\underbrace{[xh]^{-1}[xh]}_{\substack{=1}}g[xhg] = [x]hg[xhg] = xf_{hg}\, .
	\end{align*}
	Dies impliziert dann, dass $\varphi((f_h, h)(f_g,g)) = (\varphi f_{hg}, hg)$ und somit $\varphi$ ein Hom. ist. 
	Abschlie\ss end ist nachzuweisen, dass die $1$ auf die $1$ abgebildet wird. Es gilt:
	\begin{align*}
		\varphi(f_g, g) = 1 \in H \wr F &\implies \psi(g) = H \\
		&\implies g \in H
	\end{align*}
	und
	\begin{align*}
		xfg = 1 \, \, \forall x &\implies 1f_g = [1]g[1g]^{-1} = g \cdot 1 = 1 \\
		&\implies g = 1\, .
	\end{align*}
	\bewiesen
	
	\section{Erkennbare und Rationale Mengen 1 - Vorlesung 7}
	\subsection{Erkennbare Mengen, Rationale Mengen und Sternfreie Sprachen}
	\subsubsection{Definition Rationale Menge ($\rat$)}
	Sei $M$ eine Monoid (z.B $M = \sigstern$).
	Dann ist $\rat(M)$ die Menge der rationalen Mengen von $M$ und die Elemente von $\rat(M)$ sind induktiv definiert durch:
	\begin{enumerate}[label=\arabic*)]
		\item $L \subseteq M \wedge |L| \leq \infty \implies L \in \rat(M)$
		\item $L_1, L_2 \in \rat(M) \implies L_1 \cup L_2 \in \rat(M) \, \wedge \, L_1 \cdot L_2 \in \rat(M)$, \\ 
		wobei $ L_1 \cdot L_2 = \{w \in M \mid \exists u,v \in M : w = uv \, \wedge \, u \in L_1 \,\wedge\, v \in L_2\}$
		\item $L \in \rat(M) \implies L^\ast = \bigcup_{i \geq 0} L^i \in \rat(M)\,,$\\
		mit $L^0=\{1_m\}$ und $L^{i+1} = L^iL$.\\ \newline
		Man beachte, dass $\bigcup_{i \geq 0} L^i $ hierbei das erzeugte Untermonoid von $L$ in $M$ ist.
	\end{enumerate}
	
	
	\subsubsection{Definition Sternfreie Sprache ($\starfree$)}
	Für ein Monoid $M$ enthält die Menge der sternfreien Sprachen $\starfree(M)$ folgende Mengen:
	\begin{enumerate}[label=\arabic*)]
		\item $L \subseteq M \wedge |L| < \infty \implies L \in \starfree(M)$
		\item $L_1, L_2 \in \starfree(M) \implies L_1 \cup L_2 \in \starfree(M) \, \wedge \, L_1 \cdot L_2 \in \starfree(M) \, \wedge \, M \setminus L = \overline{L} \in \starfree(M)$\\ 
	\end{enumerate}
	
	\subsubsection{Beispiele für Sternfreie Sprachen}
	Mit $\Sigma = \{a,b\}$ gilt:
	\begin{itemize}
		\item $(aa)^\ast \not\in \starfree(\sigstern)$
				\item $\sigstern \in \starfree(\sigstern)$, da $\sigstern = \overline{\emptyset}$
		\item $(ab)^\ast \in \starfree(\sigstern)$, da $(ab)^\ast \in \starfree(\sigstern) \iff \overline{(ab)^\ast}\in \starfree(\sigstern)$ und $\overline{(ab)^\ast} = \sigstern aa \sigstern \cup \sigstern bb \sigstern \cup b\sigstern \cup \sigstern a$
		\item $A^\ast \in \starfree(\sigstern)$ für $A\subseteq \Sigma$, da $A^\ast = \{ w \in \sigstern \mid \forall b \in \Sigma \setminus A : w \notin \sigstern b\sigstern\}$.
	\end{itemize}
	
	
	
	\subsubsection{Definition erkennbare Sprachen ($\rec$)}
	\label{sec:rec}
	Wir definieren die Menge $\rec(M)$ der erkennbaren Mengen eines Monoids $M$ wie folgt:  $$L \in \rec(M) \iff \exists \varphi : M \to N \text{ ein Hom. von Monoiden mit } |N| < \infty \text{ und } \varphi\inv\varphi(L) = L$$
	Das heißt $\forall w \in M$ gilt $$w\in L \iff \varphi(w) \in \varphi(L) \subseteq N\, .$$
	
	\subsubsection{Definition aperiodisch}
	Ein Monoid $N$ ist aperiodisch genau dann, wenn alle Unterhalbguppen von $N$, die Gruppen sind, sind trivial, bzw. $N$ gruppenfrei ist.\\
	Alternativ gilt: Ein Monoid $M$	ist aperidodisch, falls $$\forall x \in M\,\,\exists t \in \N \, : \, x^{t+1} = x^t \, .$$
	Falls $M$ endlich ist, gilt: $$M \text{ ist aperiodisch} \iff \exists t \in \N\, \forall x \in M \, : x^{t+1} = x^t$$
	
	
	\subsubsection{Satz von Schützenberger}
	Es gilt: 
	\begin{align*}
		L &\in \starfree(\sigstern)\\
		&\iff \\
		\exists \varphi: \sigstern \to N \text{ ist ein Hom. mit } |N| \leq &\infty \text{ und } \varphi\inv\varphi(L) = L \text{ und $N$ ist aperiodisch} \\ 
		&\iff \\
		\text{das syntaktische Monoid } &\synt(L) \text{ ist aperiodisch}
	\end{align*}
	
	\subsubsection{Definition DFA über M}
	Ein DFA (Deterministic Finite Automaton) über einem Monoid $M$ ist ein Tupel $(Q, \cdot, q_0, F)$, wobei
	\begin{enumerate}[label=\arabic*)]
		\item $\cdot: Q \times M \to Q$ ist eine Operation von $M$ auf $Q$ von rechts. Das heißt also
		\begin{align*}
			\forall q \in Q \text{ und } \forall u,v \in M \text{ gilt } (q \cdot u) \cdot v = q(uv) \text{ und } q1_M = q \, . 
		\end{align*}
		\item $|Q| < \infty$ ist eine endliche Menge von Zuständen, $F\subseteq Q$ ist die Menge der Endzustände und $q_0 \in Q$ ist der Startzustand. \\
	\end{enumerate}
	gilt.\\
	Dies definiert dann die Sprache $L(A) = \{w \in M \mid q_0\cdot w \in F\}$.
	
	\subsubsection{Beispiel für einen DFA über M}
	Sei $L \in \rec(M)$ mit $\varphi: M \to N$ wie in der Definition~\ref{sec:rec}. \\ 
	Dann definiert dies einen DFA $A = (Q,\cdot, q_0, F)$ mit $L(A) = L$ und $N = Q$. 
	Hierbei ist: 
	\begin{itemize}
		\item $q_0 = 1_M \in N$
		\item $\cdot : N \times M \to N$ mit $q\cdot u := q\varphi(u)$. Dann gilt
		\begin{align*}
			(q \cdot u)\cdot v = (q\varphi(u))v = q \varphi(u)\varphi(v) = q(\varphi(uv)) = q(uv)
		\end{align*}
		\item $q = q \cdot 1_M = q \cdot \varphi(1_M) = q1_N = q$
		\item $F = \{\varphi(u) \mid u \in L \}$
	\end{itemize}

	\subsubsection{Satz von Kleene}
	Für ein endliches Alphabet $\Sigma$ gilt $$\rat(\Sigma^*)=\starfree(\Sigma^*)=\rec(\Sigma^*)\,.$$
	
	\section{Erkennbare Mengen 2 - Vorlesung 8}
	\subsection{Abschlusseigenschaften von $\rat$ und $\rec$}
	\subsubsection{Abschluss von $\rat$ unter Hom.}
	Wenn $h: M' \to M$ ein Monoid-Hom. ist, dann gilt $$L' \in \rat(M') \implies h(L') \in \rat(M)\, .$$
	Das bedeutet, rationale Mengen sind unter Homomorphismen abgeschlossen.
	
	\subsubsection{Abschluss von $\rec$ unter Hom.}
	Wenn $h: M' \to M$ ein Monoid-Hom. ist, dann gilt $$L \in \rec(M) \implies h\inv(L) \in \rec(M')\, .$$
	Das bedeutet, erkennbare Mengen sind unter Homomorphismen abgeschlossen.
	
	\subsection{Minimaler Erkennender DFA}
	\subsubsection{Definitionen}
	\label{sec:min_dfa}
	Für eine Teilmenge $L \subseteq M$ eines Monoids $M$, soll ein DFA $A_L$ definiert werden.
	Dazu sei $Q_L = \{L(u) \mid u \in M\}$ die Zustandsmenge, wobei $L(u) = \{ w \in M \mid uw \in L\}$ die Leistung eines Wortes ist. Weiter sei $q_{0,L}=L(1)=L$ der Startzustand und $F_L=\{L(u) \, | \, u \in L\}= \{L(u) \, | \, 1 \in L(u)\}$ die Menge der Endzustände.
	Die Übergangsfunktion sei definiert durch:
	\begin{align*}
		\cdot: Q_L \times M  &\to Q_L\, ,\\
		L(u)\cdot v &\mapsto L(uv)
	\end{align*}
	Dann gilt $L(A_L) = L$, also erkennt dieser Automat $L$.\\
	Nun wollen wir zeigen, dass $A_L$ minimal ist.
	Seien $A, A'$ deterministische endliche Automaten über $M$. Dann hei\ss t 
	\begin{align*}
		f: Q \to Q'
	\end{align*} ein Morphismus, falls $f(q)u=f(qu)$, $f(q_0) = q_0'$ und $f(F) \subseteq F'$ gilt.
	Wir folgern daraus, dass $L(A) \subseteq L(A')$ gilt und für Gleichheit $L(A) = L(A')$ genau dann zutrifft, wenn $f\inv(F') = F$ gilt. 
	
	\subsubsection{Lemma: Minimalität des DFA}
	Sei $A = (Q, \cdot, q_0, F)$ ein deterministischer Automat mit $L = L(A)$ und $A_L = (Q_L, \cdot, q_{0,L}, F_L)$ der in Abschnitt~\ref{sec:min_dfa} definierte Automat.  \\
	
	Wir setzen $A' = (\{q_0u \mid u \in M\}, \cdot, q_0, \{q_0u \in F \mid u \in M\})$, dann definiert $$f: \{q_0u \mid u \in M\} \to Q_L$$ durch $f(q_0u) = L(u)$ einen surjektiven Morphismus.\\
	Daraus können wir schlie\ss en, dass $A$ nicht kleiner sein kann als $A_L$ und somit $A_L$ der minimale erkennende Automat für $L$ ist.
	
	
	\subsubsection{Beweisskizze Lemma}
	Um die Wohldefiniertheit von $f$ zu zeigen, definieren wir die Leistung $L(q)$ für $q \in Q$ durch $$L(q) = \{v \in M \mid qv \in F\}\, .$$ Also gilt dann $L(q_0u) = \{v \in M \mid q_0uv \in F\} = \{v \in M \mid uv \in F\}$.
	Dadurch hängt die Leistung dann nur vom Zustand ab. 
	Der Rest des Beweises ist trivial.
	
	\subsection{Syntaktische Kongruenz}
	\subsubsection{Definition}
	Sei $L \subseteq M$ eine Teilmenge eines Monoids $M$. Wir definieren dann $[u] = \{v \in M \mid u \equiv_L v\}$ als die Kongruenzklasse von $u$.\\
	Die Äquivalenzrelation $\equiv_L$ ist dabei definiert durch $$ u \equiv_L v \iff \forall x,y \in M : (xuy \in L \iff xvy \in L)\, .$$
	\\
	Die Menge der Kongruenzklassen nennt man Syntaktisches Monoid oder $Synt(L)$.
	
	\subsubsection{Lemma: $\equiv_L$ ist eine Kongruenz}
	$\equiv_L$ ist eine Kongruenz. Also gilt laut der Definition von Kongruenzen: $$[u][v]=[uv]$$ 
	
	\subsubsection{Beweis des Lemmas}
	Seien $u \equiv_L u'$ und $v \equiv_L v'$. Nun ist zu zeigen, dass $uv \equiv_L u'v'$ gilt. 
	
	Betrachte $x,y \in M$ mit $xuvy \in L$. 
	Es folgt: $$xuvy \in L \iff xu'vy \in L \iff xu'v'y \in L$$ Damit ist auch $\pi: M \to \synt(L), \, u \mapsto [u]$ ein surjektiver Hom.. 
	
	\subsubsection{Satz: Minimalität von Synt(L)}
	Sei $h: M \to N$ ein Monoid-Hom. mit $h\inv(h(L)) = L$. Dann definiert $f:h(M) \to \synt(L)$ durch $h(u) = [u]$ einen surjektiven Hom.. \\
	Durch die Surjektivität folgt wieder, dass $\synt(L)$ nicht größer als $N$ sein kann und damit das kleinste Monoid ist, dass $L$ erkennt. 
	
	\subsubsection{Beweis des Satzes}	 
	Zu zeigen ist (nur) die Wohldefiniertheit, also, dass $h(u) = h(v) \implies [u]=[v]$ gilt. Dies ist trivial.
	
	\subsubsection{Graphische Interpretation}	
	Abbildung~\ref{fig:synt} veranschaulicht den Beweis für $n = h(u) = h(v) \implies [u] = [v]$.
	\begin{figure}[h!]
		\centering
		\begin{tikzpicture}[node distance=3cm, auto]
			\node (A) {$h:M$};
			\node (B) [right of=A] {$h(M) = N' \leq N$};
			\node (C) [below of=A] {$\synt(L)$};
			
			\draw[-{Stealth}] (A) to node {} (B);
			\draw[-{Stealth}, dashed] (B) to node [] {$\exists! f: f(n) = [u], \text{ falls } n = h(u)$} (C);
			\draw[-{Stealth}] (A) to node [swap] {$\pi : u \mapsto [u]$} (C);
		\end{tikzpicture}
		\caption{Graphische Interpretation des Beweises der Minimalität von $Synt(L)$.}
		\label{fig:synt}
	\end{figure}
	
	\subsubsection{Folgerung: Zusammenhang von Synt(L) und erkennenden Automaten}
	\label{sec:equiv}
	Es gilt:\\	
	$L \in rec(M) \iff |\synt(L)| < \infty \iff A_L \text{ ist ein DFA, dh. } Q_L = \{L(u) \mid u \in M\} \text{ ist endlich}$
	
	\subsubsection{Beweis der Folgerung}
	Sei $L \subseteq M$ und  $A = (Q, \cdot, q_0, F)$ ein deterministischer Automat mit $L = L(A)$. \\
	Dann ist $(Q^Q, \circ, id_Q)$ ein Monoid mit  $Q^Q = \{ f : Q \to Q \mid f\text{ ist eine Abbildung}\}$ und $(f \circ g)(q) = g f(q)$.
	Wir definieren einen Homomorphismus $$ \varphi : M \to Q^Q, \, \, u \mapsto f_u$$  mit $f_u(q) = q_u$. 
	Dann ist $(f_u \circ f_v)(q) = quv$.\\
	Weiter sei $T = \{f_u \mid u \in M\}$ das Transformationsmonoid.
	Somit folgt $\varphi: M \to T$. Für $Q=n$ gilt $|T| \leq n^n \in 2^{n \log n}$. 
	Graphisch kann man sich den Beweis nun analog zu Abbildung~\ref{fig:synt} vorstellen, durch
	\begin{figure}[H]
		\centering
		\begin{tikzpicture}[node distance=3cm, auto]
			\node (A) {$M$};
			\node (B) [right of=A] {$T$};
			\node (C) [below of=A] {$\synt(L)$};
			
			\draw[-{Stealth}] (A) to node {} (B);
			\draw[-{Stealth}, dashed] (B) to node [] {$\psi$} (C);
			\draw[-{Stealth}] (A) to node [swap] {$y \mapsto [y]$} (C);
		\end{tikzpicture}
	\end{figure}
	
	mit $\psi(f_u)=[u]$, wobei $f_u = f_v \overset{!}{\implies} [u] = [v]$ zu zeigen ist. 
	
	Wir betrachten nun $x, y \in M$ mit $xuy \in L$.
	Dann gilt:
	\begin{align*}
		xuy \in L &\implies q_0xuy \in F\\
		&\implies q_0xu = f_u(q_0x) = f_v(q_0x)\\
		&\implies q_0xvy \in F \\
		&\implies xvy \in L\\
		&\implies \psi \text{ ist wohldefiniert}
	\end{align*}
	Damit wissen wir, dass die Existenz eines endlichen erkennenden Automaten, die Endlichkeit des syntaktischen Monoids impliziert.\\
	
	Für die Umkehrung sei nun $Q = Q_L = \{L(x) \mid x \in M\}$ die Zustandsmenge des minimalen deterministischen Automaten. Dann gilt $[u]= [v] \implies f_u = f_v$. Dies ist begründet durch:
	\begin{align*}
		[u]= [ v ] \implies f_u(L(x)) &= L(x)u \\
		&= L(xu) = \{y \mid xuy \in L\} \\ 
		&= \{y \mid xvy \in L\} \\
		&= L(x)v = f_v(L(x))
	\end{align*}
	Aus $\forall x\, :\, L(x)u = L(x)v$ folgt $f_u = f_v$.\\
	Insgesamt gilt damit die Äquivalenz aus Abschnitt~\ref{sec:equiv}. 
	\bewiesen
	
	\subsubsection{Konsequenz: Isomprphie von Synt(L) und T}
	Für das syntaktische Monoid $Synt(L)$ und das Trasnformationsmonoid $T$ gilt:
	$$\synt(L) \cong T = \{f_u \mid u \in M\}$$
	Das bedeutet, man kann $Synt(L)$ aus dem minimalen DFA berechnen.
	
	\subsubsection{Satz von McKnight}
	Es gilt: 
	\begin{align*}
		\rec(M) \subseteq \rat(M) &\iff M \in \rat(M)\\ 
		&\iff M \text{ ist endlich erzeugt} \\ 
		&:\iff \exists \, \Sigma \text{ endlich und } \pi : \sigstern \to M \text{ ist surjektiver Hom. }
	\end{align*}
	
	\subsubsection{Beweis des Satzes von McKnight}
	Es gilt $\emptyset, M \in \rec(M)$ durch das triviale Monoid, vermöge $$h: M \to \{1\}\, .$$ Wenn $M \in \rat(M)$ ist, dann folgt daraus, dass $M=L$ für einen rationalen Ausdruck für $M$. 
	Da dieser nur eine endliche Teilmenge von $M$ benutzt, folgt die Hin-Richtung der Behauptung.
	
	Zu zeigen ist also nun noch die andere Richtung. 
	\newline
	
	Sei $\pi : \sigstern \to M$ ein surjektiver Hom. und $|\Sigma| < \infty$. 
	Wir wollen zeigen, dass dann $L \in \rec(M) \implies L \in \rat(M)$ gilt.\\
	Wir haben also
	
	\begin{equation*}
		\begin{tikzcd}[column sep=1.5pc]
			\sigstern \arrow{r}{\pi} & M \arrow{r}{h} & N 
		\end{tikzcd}
	\end{equation*}
	mit $|N| < \infty$ und $h\inv(h(L)) = L$. Dann folgt $\pi\inv(L) \in \rec(M)$ wegen der Abgeschlossenheit unter Hom..
	Weiter gilt dann $\exists \text{ DFA } A \text{ mit } \pi\inv(L) = L(A)$, woraus $\pi\inv(L) \in \rat(\Sigma^*)$ folgt.
	Wegen der Surjektivität von $\pi$ gilt $\pi\pi\inv(L) = L \in \rat(M)$.
	\bewiesen
	
	\subsubsection{Definition NFA über M}
	Ein NFA $A$ über einem Monoid $M$ ist ein Tupel $(Q, \delta, I, F)$ mit $I, F \subseteq Q$, wobei $$\delta \subseteq Q \times M \times Q$$ und $|\delta| < \infty$ gilt. 
	Dann ist die erkannte Sprache $L(A) = \{w \in M \mid \exists u_1, \ldots, u_n \in M : w = u_1 \ldots u_n \, \wedge \text{ es gibt Pfad } \begin{tikzcd}[column sep=1.7pc]
		q_0\arrow{r}{u_1} &  q_1 \arrow{r}{u_2} & \ldots \arrow{r}{u_n} & q_n
	\end{tikzcd} \text{ aus Übergängen in } \delta \text{ mit } q_0 \in I \wedge q_n \in F\}$. 

	
	\subsubsection{Satz: Zusammenhang von rationalen Mengen und NFAs}
	Es gilt $$ L \in \rat(M) \iff \exists \text{ NFA mit } L(A) = L \, .$$
	
	\subsubsection{Beweis des Satzes}
	Wie in der Vorlesung Formale Sprachen und Automatentheorie.
	
	\section{Syntaktische Monoide - Vorlesung 9}
	\subsection{Wortproblem in Gruppen}
	\subsubsection{Definition}
	Sei $\pi : \sigstern \to G$ eine Präsentation einer endlich erzeugten Gruppe(Eine präsentation einer Gruppe ist eine Menge von Elementen, die die Gruppe erzeugen, und eine Menge von Relationen, die zwischen diesen Erzeugern bestehen.). Dann definiert $$\wop(G) = \{ w \in \sigstern \mid \pi(w)=1_G\}$$das Wortproblem in dieser Gruppe.
	
	\subsubsection{Lemma: Syntaktisches Monoid des Wortproblems}
	Es gilt $\synt(\wop(G))=G$.\\
	Wiederholung: Das syntaktische Monoid wird gebildet durch: 
	\begin{align*}
		\sigstern &\to \synt(L) \\
		u &\mapsto [u]
	\end{align*}
	
	\subsubsection{Beweis des Lemmas}
	Für die Rück-Richtung sei $\pi(u) = \pi(v)$. Daraus folgt dann
	\begin{align*}
		xuy \in \wop(G) &\iff \pi(x)\pi(u)\pi(y) = 1  \\
		&\iff \pi(x)\pi(v)\pi(y) = 1 \\
		&\iff xvy \in \wop(G)
	\end{align*}
	Daraus schlie\ss en wir $u \equivWP v$.
	\newline
	
	Für die Hin-Richtung sei jetzt $u \equivWP v$. Dann gilt 
	\begin{align*}
		\pi(y) = \pi(u)\inv &\iff  1\cdot u \cdot y \in \wop(G)\\
		&\iff 1\cdot v \cdot y \in \wop(G)\\
		&\iff \pi(y) = \pi(v)\inv
	\end{align*}
	Also folgt $\pi(u) = \pi(v)$.
	\bewiesen
	
	\subsubsection{Korrolar: Regularität des Wortproblems}
	Wir können ohne Beweis also schlie\ss en, dass $$|G| < \infty \iff \wop(G) \text{ ist reguläre Sprache}$$ gilt.
	
	\subsubsection{Satz: Untergruppen erkennbarer Sprachen}
	Sei $L \in \rec(G)$. Dann gibt es eine Untergruppe (sogar einen Normalteiler $H \trianglelefteq G$) von endlichem Index und es existieren Elemente
	$g_1, \ldots, g_n$ so, dass $L = g_1H \cup g_2H \cup \ldots \cup g_nH$.   
	
	\subsubsection{Beweis des Satzes}
	Sei $L \in \rec(G)$. Dann gilt $L= h\inv(F)$ für einen Hom. $h:G \to N$ mit $|N| < \infty$ und $F \subseteq N$. \\
	
	\OE $\, \,$gilt $h$ ist surjektiv und da $h:G \to N$ ein Hom. ist, folgt, dass $N$  eine Gruppe ist. 
	
	Sei nun $h(g) \in N$ und $H= ker(h) \trianglelefteq G$ ein Normalteiler von endlichem Index, dann folgt $h(gH) \in F$ und somit $gH \subseteq L$.
	\bewiesen
	
	\subsubsection{Folgerung des Satzes}	
	Ist $G$ eine unendliche Gruppe und $L \subseteq G$ endlich (zB. $L = \{1\}$), dann ist $L \in rat(G) \setminus \rec(G)$
	
	\subsection{Idempotenz}
	\subsubsection{Definition}
	Ein Element $e$ aus einem Monoid $M$ hei\ss t idempotent, falls $$e^2 = e$$gilt.
	Damit können wir die folgende Menge der Idempotente definieren: $$E(M) = \{ e \in M \mid e^2 = 2\}$$ Es gilt immer $1\in E(M)$.
	
	\subsubsection{Definition lokales Monoid}
	Sei $S$ eine Halbgruppe, $E(S) = \{e \in S \mid e^2 = e\}$ und $e \in E(S)$. Dann ist die Unterhalbgruppe $eS^1e \leq S$ sogar ein Monoid, da 	$e(exe) = exe = (exe)e$, wobei 
	\begin{align*}
		eS^1e &:= eSe \cup \{ee\} \\
		&=eSe \cup \{e\}
	\end{align*}
	gilt.\\
	Für ein Monoid $M$, nennt man das Monoid $eMe$ das lokale Monoid bei $e \in E(M)$.
	
	\subsubsection{Definition Teilwort}
	Ein Wort $a = a_1 \ldots a_n$ hei\ss t Teilwort von $w$ falls $w \in \sigstern a_1 \sigstern a_2 \sigstern \ldots a_n\sigstern$ gilt.
	
	\subsubsection{Beispiele} 
	\begin{itemize}
		\item Sei $\Sigma = \{a, b\}$ und $\synt((ab)^\ast) = \{1, a ,b , ab, ba, 0\} = B^1_2$ das Brandt-Monoid mit $2$ Erzeugenden und den folgenden Rechenregeln für alle $x \in B^1_2$: 
		\begin{align*}
			1x &= x1 = x   &&aba = a  &&&a^2=b^2 = 0\\
			0x &= x0 = 0   &&bab = b 
		\end{align*}
		Weiter ist $$\pi: \sigstern \to B^1_2$$ definiert durch $\pi(a) = a$ und $\pi(b) = b$. \\
		Es gilt:
		\begin{align*}
			(ab)^n &\equiv 1 \text{ für } n = 0 \\
			(ab)^n &\equiv ab \text{ für } n \geq 1 
		\end{align*} Die idempotenten Elemente sind also: $1, ab ,ba, 0$.
		\item Wir betrachten $\mathsf{Synt}(\Sigma^*a\Sigma^*b\Sigma^*)$ für $\Sigma= \{a,b\}$. Dieses syntaktische Monoid lässt sich darstellen als $M = \{1, a, b, ba, 0\}$ mit den folgenden Rechenregeln: 
		\begin{align*}
			ab &= 0    && a^2 = a &&&b^2 = b \\
			b(ba) &= ba   &&(ba)a = ba &&& (ba)b = 0 \\
			a(ba) &= 0  &&(ba)^2 = (ba)(ba) = 0
		\end{align*}
		Somit ist zu erkennen, dass $ba$ das einzige nicht idempotente Element ist. 
		\item Kontextfreie Grammatiken erlauben Regeln der Form $$A \to \alpha \text{ mit } A \in V, \alpha \in (\Sigma \cup V)^\ast$$ oder $$\varepsilon \to \alpha, \text{ mit } \alpha \in (V \cup \Sigma)^\ast\, .$$ 
		
		Für $\Sigma = \{a, b\}$, betrachten wir nun die folgende Grammatik $\mathscr{G}$:
		\begin{align*}
			S &\to \varepsilon \\ 
			\varepsilon &\to a^2 \mid b^2 
		\end{align*} 
		Sei $L=L(\mathscr{G})$.\\
		\textbf{Frage:} $a^nb^2a^m \in L$? 
		\newline
		\textbf{Antwort:} Ja, genau dann, wenn $n+m \equiv 0\, (2)$ gilt.\\
		\newline
		\textbf{Frage:} $(ab)^n(ba)^m \in L$? 
		\newline
		\textbf{Antwort:} Ja, genau dann, wenn $n=m$ gilt.\\
		\newline
		Somit folgt: $L$ ist nicht regulär.\\
		Diese Sprache ist das Wortproblem einer Gruppe:\\
		Es gilt $L(\mathscr{G})=\mathsf{WP}(G)$ für die Gruppe $G=\Z \rtimes \Z/2\Z$ mit $(m,i)(n,j) = (m+(-1)^in, i+j\, mod\, 2) $.
	\end{itemize}
	
	\section{Halbgruppentheorie Teil 1 - Vorlesung 10}
	\subsection{Greens Relations}
	\subsubsection{Definition Ideal}
	Bei Idealen eines Rings unterscheidet man in sogenannte Links-, Rechts- und Beidseitige Ideale. 
	Sei $I \subseteq R$ für einen Ring $R$. Dann ist $I$ ein Linksideal, falls 
	\begin{enumerate}[label= \arabic*.)]
		\item $(I, +)$ bildet eine Untergruppe von $R$ 
		\item $\forall a \in I, r \in R : r \cdot a \in I  $
	\end{enumerate}
	\subsubsection{Definition Greensrelations}	
	Die folgenden Relationen sind alles Äquivalenz Relationen. 
	\begin{align*}
		\mathcal L \subseteq M \times M: x \grel y &\iff Mx = My \\ 
		&\iff x,y \text{ erzeugen identische Linksideale} \\
		\mathcal R \subseteq M \times M: x \grer y &\iff xM = yM  \\ 
		&\iff x,y \text{ erzeugen identische Rechtsideale} \\
		\mathcal J \subseteq M \times M: x \grej y &\iff MxM = MyM \\ 
		\mathcal H = \mathcal L \cap \mathcal R: x \greh y &\iff x \grel y \text{ und } x \grer y
	\end{align*}
	
	Sowie 
	\begin{align*}
		\mathcal L \subseteq M \times M: x \lgreleq y &\iff Mx \subseteq My \\
		\mathcal R \subseteq M \times M: x \lgrereq y &\iff xM \subseteq yM \\
		\mathcal J \subseteq M \times M: x \lgrejeq y &\iff MxM \subseteq MyM  \\
		\mathcal H \subseteq M \times M: x \lgreheq y &\iff x \lgreleq y \land x \lgrereq y \\
		(s,t) \in \mathcal{D} &\iff \exists z: (s,z) \in \mathcal L \land (z,t) \in \mathcal R \\ 
		&\iff (s,t) \in \mathcal L \mathcal{R} =  \mathcal R \mathcal{L}
	\end{align*}
	
	Im Allgemeinen gilt für $R \subseteq X \times Y$ und $S \subseteq Y \times Z$
	\begin{align*}
		R \circ S &= \{(x,y) \in X \times Z \mid \exists y \in Y: (x,y) \in R, (y, z) \in S\} 
	\end{align*} 
	Bezogen auf die Greensrelations bedeutet das $$\mathcal L \circ \mathcal R = \{ (x,y) \subseteq M \times M \mid \exists z \in M : x \grel z \grer y \}$$
	\newline
	
	Somit entsteht dann entsteht folgendes Bild 
	\begin{figure}[h!]
		\centering
		\begin{tikzpicture}[node distance=3cm, auto]
			\node (H) {$\mathcal{H}$};
			\node (L) [above right of=H] {$\mathcal{L}$};
			\node (R) [below right of=H] {$\mathcal{R}$};
			\node (LUR) [below right of=L] {$\mathcal L \cup \mathcal R$};
			\node (LDR) [right of=LUR] {$\mathcal L \circ \mathcal R = \mathcal D$};
			\node (J) [right of=LDR] {$\mathcal J$};
			
			\draw[-] (H) -- (L) node[midway,sloped,above,rotate=0] {$\subseteq$};
			\draw[-] (H) -- (R) node[midway,sloped,below,rotate=0] {$\subseteq$};
			\draw[-] (L) -- (LUR) node[midway,sloped,above,rotate=0] {$\subseteq$};
			\draw[-] (R) -- (LUR) node[midway,sloped,below,rotate=0] {$\subseteq$};
			\draw[-] (LUR) -- (LDR) node[midway,sloped,above,rotate=0] {$\subseteq$};
			\draw[-] (LDR) -- (J) node[midway,sloped,above,rotate=0] {$\subseteq$};
		\end{tikzpicture}
		\caption{Die Verbindungen zwischen den einzelnen Greensrelations}
	\end{figure} \\ 
	
	Im Allgemeinen gilt $\mathcal D \not = \mathcal J$, für endliche Halbgruppen jedoch $\mathcal D = \mathcal J$. Dies werden wir später noch zeigen. 

	\subsubsection{Definition K-Trivial}		
	Sei $K \in \{\mathcal{H}, \mathcal{L},\mathcal{R}, \mathcal{D}, \mathcal{J}\}$, dann heißt $\sim$, also die Relation,  $K$-Trivial, falls $$x \sim_K y \implies x = y$$
	
	\subsubsection{Lemma}
	Sei $x \in M$ und $|M| < \infty$, dann existiert genau eine Potenz $x^n \in M$ mit $x^n = x^{2n}$. Also gilt $x^n \in E(M))$ und wird mit $x\pom = x^n$ bezeichnet. 		
	
	\subsubsection{Beweis des Lemmas}
	Es gilt $\forall x \in M $ existiert ein $t \in \N$ (thershold)  und $p \in \N$ mit $p \geq 1$ und $x^{t+p} = x^t \in M$
	\begin{figure}[h!]
		\centering
		\begin{tikzpicture}[node distance=1.9cm, auto]
			\node (1) {$1$};
			\node (x) [right of=1] {$x$};
			\node (x2) [right of=x]{$x^2$};
			\node (dots) [right of=x2]{$\ldots$};
			\node (xt) [right of=dots]{$x^t$};
			\node (xt1) [above right of=xt] {$x^{t+1}$};
			\node (xtpm1)  [ below right of = xt] {$x^{t+p -1}$};
			\node (dotslow) [below right of=xt1]{ $\ldots$};
			
			
			\draw[ ->] (1) -- (x) node[ midway,sloped,above,rotate=0] { };
			\draw[->] (x) -- (x2) node[midway,sloped,above,rotate=0] {};
			\draw[->] (x2) -- (dots) node[midway,sloped,above,rotate=0] {};
			\draw[->] (dots) -- (xt) node[midway,sloped,above,rotate=0] {};
			\draw[->] (xt) edge   [bend left] (xt1) node[midway,sloped,above,rotate=0] {};
			\draw[->] (xt1) edge   [bend left](dotslow) node[midway,sloped,above,rotate=0] {};
			\draw[->] (dotslow) edge   [bend left] (xtpm1) node[midway,sloped,above,rotate=0] {};
			\draw[->] (xtpm1) edge   [bend left] (xt) node[midway,sloped,above,rotate=0] {};
			
		\end{tikzpicture}
	\end{figure} 
	\newline
	
	Wir sehen dass in der Folge $( x^i \mid i \in \N)$ ein $i < j$ mit $x^i = x^j$ existiert. Wir wählen nun $i=t$ minimal, dann ist $t+p = j$ minimal. 
	Es folgt 
	\begin{align*}
		x^i = x^{2i} \implies i \geq t &\iff x^{t+j} = x^{2t + 2j} = x^{i + t2j} \\
		&\iff j \equiv t + 2j \mod p \\
		&\iff t \equiv -j \mod p
	\end{align*}
	\bewiesen
	
	\subsubsection{Beobachtung}
	Sei $x^{t+p} = x^t$, dann gilt 
	\begin{align*}
		x^t \geqslant_{\mathcal{L, R}} x^{t+1} \geqslant_{\mathcal{L, R}} \ldots \geqslant_{\mathcal{L, R}} x^{t+p-1} \geqslant_{\mathcal{L, R}} x^{t+p} = x^t
	\end{align*}
	Weiter sehen wir
	\begin{align*}
		&\implies x^t \greh x^t \, \, \forall i \in \N \\
		&\implies \forall x \text{ gilt } x^{t+i} \in \mathcal{H}(x\pom), \text{ wobei } x\pom \in E(M)
	\end{align*}
	
	\subsubsection{Lemma}
	Sei $G \leq M$ eine Unterhalbgruppe, die eine Gruppe ist, dann $x\pom = y\pom = 1_G$
	Es gilt dann $\forall x,y \in G \implies \exists x\pom = e \in M, \, e = 1_G$ und $ G \leq \mathcal H(e)$
	
	
	\subsubsection{Lemma 7.42 (DAM)}
	Sei $s \grej t$ und $M$ endlich mit $s = ptu$. Dann ist $t=qsv$. Nun gilt folgendes 
	\begin{enumerate}[label = \alph*)]
		\item $s \lgrereq t \implies s \grer t$ (analog dazu für $\grel$)
		\item $\mathcal{L}(s) \cap \mathcal{R}(t) \not = \emptyset$
		\item $\mathcal{D} = \mathcal{J}$
	\end{enumerate}
	
	\subsubsection{Beweis des Lemma 7.42 \label{dam}(DAM)}
	Der Beweis dazu ist auch im Buch (englischer Version) unter dem Lemma 7.42 zu finden.
	\begin{enumerate}[label = \alph*)]
		\item Sei $s \lgrereq t \implies s = tu$. Dann gilt 
		\begin{align*}
			t=qsv = qtuv&= q(qtuv)uv \\
			&= q\pom t(uv)\pom \\
			&= q\pom t(uv)\pom(uv)\pom \\
			&= t(uv)\pom \in tuM = sM
		\end{align*}
		Es folgt also $t \lgrereq s \implies t \grer S$
		\bewiesen
		
		\item Sei $s=ptu$ und $t=qsv$. Dann gilt 
		\begin{align*}
			s=ptu &= (pq)s(uv) \\
			&=(pq)\pom s(vu)\pom \\
			&= (pq)\pom(pq)\pom s (uv)\pom \\
			&= (pq)\pom s
		\end{align*}
		
		$\implies s \lgreleq qs \implies s \grel qs$. Es bleibt dann zu zeigen, dass $qs \grer t$. 
		Dies folgt durch Symetrie: $s \lgrereq sv \lgrereq s \implies s \grer sv$. \\ 
		Also gilt $qs \grer qsv = t \implies qs \in \mathcal{L}(s) \cap \mathcal{R}(t) \not = \emptyset$
		\bewiesen
		
		\item Da $\mathcal{D} \subseteq \mathcal{J}$ gilt, reicht es zu zeigen dass $(s,t) \in \mathcal{J}$ und $(s,t) \in \mathcal{L} \circ \mathcal{R}$ gilt. 
		Nach b) gilt $\exists qs = z$ mit $s \grel z \grer t$. Dies zeigt die Behauptung.
		\bewiesen
	\end{enumerate}
	
	\section{Divisoren, aperiodische Monoide - Vorlesung 11}
	\subsection{Divisoren}
	\subsubsection{Definition Subquotient}
	Sei $L \subseteq M$ und $A$ ein DFA mit $L=L(A)$. \\ 
	Dann existiert ein Subautomat $A'$ mit $Q_L = \{L(u) \mid u \in M\}$ und $L(u) = \{ v \in M \mid uv \in L\}$. 
	Wir nennen dann $A_L$ den Subquotient von $A$.
	
	\begin{figure}[h!]
		\centering
		\begin{tikzpicture}[node distance=2.5cm, auto]
			\node (AS) {$A'$};
			\node (A) [right of=AS] {$A$};
			\node (AL) [below of=AS] {$A_L$};
			
			\draw[right hook->] (AS) -- (A) node[ midway,sloped,above,rotate=0] { };
			\draw[ ->>] (AS) -- (AL) node[ midway,right,rotate=0] { $q = u \mapsto L(u)$};
		\end{tikzpicture}
	\end{figure} 
	
	\subsubsection{Definition Subquotient}
	Wir betrachten nun $\varphi: M \to N$ mit $L = \varphi\inv \varphi(L)$ und $\varphi$ erkennt $L \subseteq M$.
	\begin{figure}[h!]
		\centering
		\begin{tikzpicture}[node distance=2.9cm, auto]
			\node (M) {$M$};
			\node (PM) [right of=AS] {$\varphi(M) \leq N$};
			\node (ML) [below of=AS] {$M_L$};
			
			\draw[->] (M) -- (PM) node[ midway,sloped,above,rotate=0] { $\varphi$};
			\draw[ ->] (M) -- (ML) node[ midway,right,rotate=0] { $\varphi_L$};
			\draw[ dotted, ->] (PM) -- (ML) node[ midway,right,rotate=0] {$\exists! \psi$};
		\end{tikzpicture}
	\end{figure} 
	
	Wobei $M_L = \{[u] \mid u \in M\}$, $ \psi(\varphi(u))= [u]$, und $[u] =\{u \in M \mid u \equiv_L v\}$ gilt
	
	\subsubsection{Definition Divisor}
	$M$ ist ein Divisor von $N$, falls es ein Untermonoid $N'$ von $N$ gibt und ein surjektiverHom. $$\psi: N' \to M$$ existiert. Wir schreiben dann $M \prec N$
	
	\subsubsection{Beispiel}
	Sie $Q_8 = \pm \{1, i ,j, k\}$ mit $i^2 = j^2 = k^2 = -1$ und den folgenden Rechenregeln
	\begin{align*}
		ij &= k && jk = i  \\ 
		ik&=-j 	&& ki = j \\
		ji&= -k && kj = -i \\
	\end{align*}
	Wir betrachten also eine Art ''Stellwerk'''
	\begin{figure}[h!]
		\centering
		$ \begin{array}{ccc}
			\begin{tikzpicture}[->,scale=.7] 
				\foreach \a/\t in {90/j,-30/k,210/i}{
					\node (\t) at (\a:1cm) {$\t$};
					\draw (\a-20:1cm)  arc (\a-20:\a-100:1cm);
				} 
			\end{tikzpicture}
		\end{array}$
	\end{figure}
	
	Dann gilt $Z(Q_8) = \{1, -1\}$ und $Q_8/Z(Q_8) = V = \Z/2\Z \times\Z/2\Z$.
	Wir behaupten dann, dass $V$ der Divisor in $Q_8  \rtimes \Z / 3\Z$ (semidirektes Produkt), aber weder eine Untergruppe noch ein Quotient ist.
	
	Zu Erklärung betrachte man: $Q_8  \rtimes \Z / 3\Z = SL(2, \mathbb{F}_3)$ und $\mathbb{F}_3 = \Z / 3\Z$ ist ein Körper.
	
	\subsubsection{Definition lokaler Divisor}
	Sei $M$ eine Monoid und $c \in M$. Dann sei $M_c = cM \cap Mc \subseteq M$ und $\circ$ die Multiplikation mit $$xc \circ cy = xcy$$
	Dann heißt $M_c = (\{cM \cap Mc\}, \circ, c)$ der lokale Divisor von $M$ bei $c$.
	
	\subsubsection{Wohldefiniertheit und Assoziativiät der Definition}
	Für die Wohldefiniertheit sei $x'c \circ cy' = x'cy'$, und seien $x = x'c = x''c$ und $y = cy' = cy''$. Es ist zu zeigen dass $x'c \circ cy' = x''c \circ cy''$ gilt. Dies ist der Fall, da $$x'c \circ cy' = x'cy' = x''cy' = x''cy''$$ gilt. 
	
	Für die Assoziativität gilt 
	\begin{align*}
		(x'c \circ y'c)z'c &= (x'y')z'c \\
		&= x'(y'z')c \\
		&= x'c \cdot (y'c \cdot z'c)
	\end{align*}
	\bewiesen
	
	\subsubsection{Beweis $M_c$ ist Divisor von M}
	Betrachte $\{x \in M \mid xc \in cM\}$ und die Rechtsmultiplikation durch $\rho_c$. Sei $\{x \in M \mid xc \in cM\}$ das lokale Monoid, dann gilt $1 \in \{x \in M \mid xc \in cM\}$ und $xc \in cM$ sowie $yc \in cM$ und so folgt $x \cdot yc \in M$. Nun gilt $$\rho_c(xy) = xyc = xc \circ ye = \rho_c(x) \circ \rho_c(y)$$ und somit ist 
	\begin{align*}
		\rho_c : \{x \in M \mid xc \in cM\} &\to cM \cap Mc \\
		x &\mapsto xc
	\end{align*} wobei $\rho_c$ ein surjektiver Hom. ist. \\
	Mit $cx = yc \in M_c$ gilt dann $z = \rho_c(y)$ und $y \in \{x \in M \mid xc \in cM\}$
	\bewiesen
	
	\subsubsection{Lemma}
	Sei $c\in M$, dann gilt $$M_c = cM \cap Mc = M \iff c \in U(M) \iff 1 \in cM \cap Mc$$
	Also $\exists d : cd = dc = 1$, und insbesondere $c \in U(M)$ und $|M| < \infty$. Somit gilt $|M_c| < |M|$
	
	\subsubsection{Beweis des Lemmas}
	Sei $c \in U(M)$ und $cd = dc = 1$, dann gilt $\{x \in M \mid xc \in cM\}  = M$ und $cM = Md = M$ und $\rho_c\rho_d = id_M = \rho_d\rho_c$. 
	Beachte: $U(M) = M \iff M$ ist Gruppe! 
	Sei jetzt $c \in U(M)$ und angenommen $cM \cap Mc = M \implies \exists d : cd = d$. Dann folgt dass jedes $c$ ein Rechtsinverses hat und somit dass $M$ eine Gruppe ist.
	\bewiesen
	
	\subsection{Aperiodische Monoide}
	\subsubsection{Definition lokales Untermonoid}
	Sei $M$ eine Halbgruppe  und $E(M) = \{eeM \mid e^2 = e\}$. Dann ist das lokale Untermonoid bei $e$ definiert durch die Menge $eMe$ mit 
	\[
	\left. 
	\begin{array}{lr}
		exe = eye \in M\\
		e(exe) = (exe)e = exe 
	\end{array}\right\} \implies e \text{ ist neutral}
	\]
	Falls $M$ ein Monoid ist und $e\not = 1$, dann gilt $eMe \subsetneq M$ für $e^2 = e$
	
	\subsubsection{Satz}
	Sei $N$ ein Divisor von $M$, dann gilt: $$M \text{ ist aperiodisch} \implies N \text{ ist aperiodisch}$$
	
	\section{Beweis Satz von Schützenberger - Vorlesung 12}
	\subsection{Vorbereitungen}
	\subsubsection{Satz}
	Seien $L_1, L_2 \in \starfree(\aast)$ und $t_i \in \N$ gewollt mit $\forall n \in \aast: u^{t_i +1} \equiv_{L_i} u^{t_i}$ für $i = 1,2$.
	Dann gilt $$u^{t+1} \equiv_K u^t$$ für $t = t_1 +t _2 +1$ und $K = L_1 \cdot L_2$
	
	\subsubsection{Beweis des Satzes}
	Zu zeigen ist 
	\begin{align*}
		\forall x,y \in \aast: xu^{t-1}y \in L_1L_2 \iff  xu^{t}y \in L_1L_2 = K
	\end{align*}
	Dafür sei $s \in \{t, t-1\}$ und $xu^sy \in L_1L_2$. Nun existieren drei Fälle: 
	\begin{enumerate}[label= \arabic*.]
		\item $x=u_1x$ und $x'u^sy \in L_2$ sowie $u_1 \in L \implies xu^{\pm1} \in L_1L_2$ 
		\item $xu^sy' \in L_1$ und $y = y'u_2$ sowie $u \in L_2$ (Analaog!)
		\item $xu^{s_1}u'u''u^{s_2}y$ mit $xu^{s_1}u' \in L_1$ und $u''u^s= y \in L_2$ mit $u'u'' = u$
	\end{enumerate}
	Die Behauptung folgt nun, da $s_1 + s_2 + 1 = s$ gilt.
	\bewiesen
	
	\subsubsection{Folgerung des Satz}
	Sei $M$ aperiodisch und $N$ ein Divisor, dann ist $N$ auch aperiodisch. 
	Sei nun $L \subseteq \aast$ und $\varphi\inv(\varphi(L))= L$ für einen Hom. $\varphi: M \to N$ und $|N| < \infty$ aperiodisch. 
	Dann ist das $\synt$ ein Divisor von $N$ also aperiodisch. 
	
	\subsection{Satz und Beweis}
	\subsubsection{Satzes (Schützenberger)}
	Es gilt 
	\begin{align*}
		L \in \starfree(A) &\iff L \text{ wird von einem endlichen aperiodischen Monoid erkannt } \\ 
		&\iff \synt(L) \text{ ist aperiodisch }
	\end{align*}
	
	\subsubsection{Beweis des Satzes (Schützenberger) - Erste Richtung}
	Sei $L \in \starfree(A)$ und $|L| < \infty$, dann folgt $\synt(L)$ ist aperiodisch. Seien nun $L_1, L_2 \in \starfree(A)$ und 
	\begin{align*}
		\varphi_{i} : \aast \to N_i
	\end{align*} ein erkennender Hom. mit $\varphi_i\inv\varphi_{i}(L_1) = L_1$ und $|N_i| < \infty$ wobei $N_i$ aperiodisch ist.
	Es folgt nun dass $|N_1 \times N_2| \infty$ und $N_1 \times N_2$ aperiodisch ist. Wir setzen 
	\begin{align*}
		\varphi_{i}: \aast &\to N_1 \times N_2 \\
		u &\mapsto (\varphi_1(u), \varphi_2(u))
	\end{align*} $\varphi$ erkennt nun $L_1, \,L_2,  \, L_1 \cap L_2,   \,L_1 \cup L_2,  \, \aast \setminus L_i$ für $i = 1,2$.
	Man sieht nun dass $L_1, Ö_2 \in \starfree(A) \implies \synt(L_i)$ aperiodisch gilt (induktiv). Wir schließen daraus, dass $\synt(L_1 \cdot L_2)$ aperiodisch ist. 
	Es bleibt die Rückrichttung zu zeigen. 
	
	\subsubsection{Beweis des Satzes (Schützenberger) - Lemma \label{Lemma:Schütz}}
	Sei $M$ aperiodisch, und $\forall x \in M: x^{t+1} = x^t $, wobei $M \not = \{1\}$ und $M = \langle \Sigma \rangle$ gilt. 
	Dann gibt es ein $c \in \Sigma$ mit $1 \not \in cM \cap Mc$
	
	\subsubsection{Beweis des Satzes (Schützenberger) - Beweis des Lemmas}
	Sei $w \in McM$ mit $w = 1$. Dann ist $w = ucv$. 
	\begin{align*}
		&\implies uc(ucu)v = 1 \\
		&\implies 1 = w = (uc)^t v^t \\
		&\implies w = (uc)^t v^{t+1} \\
		&\implies v = 1
	\end{align*}
	Analog dazu gilt $ucv = 1 \implies u = 1$. Also gilt $w = 1 \implies c = 1$. 
	Da nun $M \not \{1\}$ gilt, existiert ein $c = \Sigma$ mit $c \not = 1$. Sollte es eine $1 \in cM \cap Mc \subseteq McM$ geben, dann ist $c = 1$
	\bewiesen
	
	\subsubsection{Beweis des Satzes (Schützenberger) - Zweite Richtung}
	Sei $\synt(L)$ aperiodisch und endlich, dann gilt $ L \in \synt(A)$. Weiter sei $\varphi: \aast \to \synt(L)$ ein surjektiver Hom. Beachte $A_1 = \{a \in A \mid \varphi(a) = 1\}$.
	Dann folgt aus dem Lemma \ref{Lemma:Schütz} $$\varphi\inv(1) = \aast_1$$ sowie $\aast_1 \in \starfree(A)$.
	\newline
	
	Wir nehmen nun one Einschränkung an, dass $A\setminus A_1 \not = \emptyset$ ist, da sonst $\synt(L) = \{1\}$ wäre. Also existiert ein $c \in A$ mit $\varphi(c) \not = 1$. Wir setzen nun $B = A \setminus \{c\}$ und es ist $|B| < |A|$ sowie $|\varphi(c)M \cap M\varphi(c)|< |M|$. 
	Der Beweis wird nun durch Indusktion geführt nach $(|M|, |A|)$ mit lexikographischer Ordnung. 
	\newline
	
	Man betrachte nun den \textit{Gigant-Step}: Es gilt
	\begin{align*}
		(|M'|, |A'|) \leq (|M|, |A|) \iff |M'| < |M|
	\end{align*}
	Oder  den \textit{Baby-Step}: Es ist $|M'| = |M|$ und $|A'| \leq |A|$ falls $|A'| < |A|$. 
	\newline
	
	Sei nun $\varphi\inv\varphi(L) = \bigcup \{\varphi\inv(P) \mid p \in \varphi(L)\}$. Wir zeigen dann $ \varphi\inv(P) \in \starfree(A)$. \\ 
	Es ist $\varphi\inv(1) = \aast_1$ mit $A_1 = \{a \in A \mid \varphi(a) = 1\}$ und 
	\begin{align*}
		\aast_1 = A \setminus \bigcup \{\aast a \aast \mid  a \in A \setminus A_1\} \in \starfree(A)
	\end{align*}
	
	da $\aast = \aast \setminus \emptyset$ und $\emptyset \in \starfree(A)$. Wir zeigen daher nur $\varphi\inv(p) \in \starfree(A)$ für $P \not = 1$.  \\
	
	Es gilt $$\varphi\inv(p) = \bigcup \{\varphi\inv (s) c \varphi\inv(r) \mid s\varphi(c)r = p, \, \varphi(c) \not = 1, \, c \in A\}$$
	Sei nun $c\in A$ fest gewählt mit $\varphi(c) \not = 1$. Es wird gezeigt dass $\varphi\inv(s)c\varphi\inv(r) \in \starfree(A)$ gilt: 
	\newline
	
	Wir nehmen hier nun ohne Einschränkung an dass $\aast \cap M = \{1, c\}$ duch Mengenschreibweise umbenannt werden kann. 
	Man schreibt dann $\aast = \bast \cup \bast (c \bast) c \bast$ für $B = A \setminus \{c\}$. Zeige nun $\varphi\inv(p) \cap \bast(c\bast)^\ast c\bast \in \starfree(A)$: 
	\begin{align*}
		\varphi\inv(p) \cap \bast(c\bast)^\ast c\bast = \bigcup \{(\varphi\inv(q) \cap \bast)(\varphi\inv(r)\cap(c\bast)^\ast c) (\varphi\inv(s) \cap \bast) \mid p = qrs\}
	\end{align*} 
	Also gilt $\varphi\inv(q) \cap \bast \in \starfree(B) \subseteq \starfree(A)$ nach dem \textit{Baby-Step} ($|B| < |A|$). 
	\newline
	
	Es reicht also wenn wir zeigen dass $\varphi\inv(p) \cap (c\bast)c \in \starfree(A)$ für alle $p \in M$ gilt. 
	
	\begin{figure}[h!]
		\centering
		\begin{tikzpicture}[node distance=5.5cm, auto]
			\node (cb) {$(c\bast)^\ast$};
			\node (t) [right of=cb] {$T^\ast$};
			\node (xm) [below of=cb] {$\{x \in M \mid xc \in cM\}$};
			\node (mc) [right of=xm] {$M_c = (\{cM\cap Mc, \cdot, c\})$};
			
			\draw[->] (cb) -- (t) node[ midway,sloped,above,rotate=0] { $\sigma$};
			\draw[ ->] (cb) -- (xm) node[ midway,right,rotate=0] { $\varphi$};
			\draw[ ->] (xm) -- (mc) node[ midway,above,rotate=0] { $\rho_c$};
			\draw[ ->] (t) -- (mc) node[ midway,right,rotate=0] { $\psi$};
		\end{tikzpicture}
	\end{figure} wobei  $\rho_c(x) = xc$ gilt. Wir stellen nun folgende Dinge fest: 
	\begin{itemize}
		\item $T = \{[v] \mid  [v] = \varphi(v), \, v \in \bast\} \subseteq M$ zu dem wird $T$ ein endliches Alphabet.
		\item $\psi[v] = c\varphi(v)c \in cM \cap Mc$
		\item $\sigma(cv_1 \cdots cv_k) = [v_1] \cdots [v_k] \in T^k \in T^\ast$
	\end{itemize}
	Es gilt $$\varphi\ast(p) \cap (c\bast)^\ast c = \bigcup \{\varphi\inv(q) \cap (c\bast)^\ast\mid qc = p\}c$$ Wähle nun $q \in \varphi(c\bast)^\ast$ fest und zeige $\varphi\inv(q) \cap (c\bast)^\ast \in \starfree(A)$. \\ 
	Es gilt 
	\begin{align*}
		w \in \varphi\inv(q) \cap (c\bast)^\ast c \iff w = vc \text{ für ein } v \text{ mit } \varphi(v) = q
	\end{align*} 
	Also ist $\rho_c\varphi(v) = q \cdot c$ und $v =cv_1 \cdots cv_k$ mit $k \geq 0$ für alle $i \in \bast$ eindeutig. Man sieht dann $\sigma(v) = [v_1] \cdot [v_k] \in T^k \subseteq T^\ast$ mit 
	\begin{align*}
		\psi(v) &= [cv_1c] \circ \ldots \circ [cv_kc] \\
		&= cv_1 \cdots cv_kc \\
		&= \varphi(v) \cdot c \\ 
		&=\rho_c \varphi(v) \\
		&= qc \in M_c = cM \cap Mc
	\end{align*}
	\newline
	
	Nun \textit{ Gigant-Step- Induktion}: Es gilt $\psi(qc) \in \starfree(T)$, da $|M_c| < |M|$. Beachte $|T| \leq |M|$. Es bleibt jetzt nur noch zu zeigen dass 
	$$ K \in \starfree(T) \implies \sigma\inv(K) \in \starfree(A)$$ Wir zeigen $K = \{1\} \implies \sigma\inv(1) \cap (c\bast)^\ast = \{1\} \in \bast$ und $\{1\} \in \starfree(A)$.
	\newline
	
	Dafür sei $t \in T$, dann ist $\sigma\inv(t) \in \starfree(K)$, denn 
	\begin{align*}
		w \in \sigma\inv(t) \cap(c\bast)^\ast \iff w = cv \text{ mit } c \in \bast \text{ und } v \varphi(v)c = t
	\end{align*} 
	Durch den \textit{Bayb-Step} gilt $\{v \in B^t \mid c \varphi(v)c = t\} \in \starfree(B) \subseteq \starfree(A)$. \\

	
	Weiter gilt $\sigma\inv(K_1 \cap K_2) = \sigma\inv(K_1) \cup \sigma\inv(K_2)$ sowie $\sigma\inv(T^\ast \setminus K_1) = (c\bast)^\ast \setminus K_1 \in \starfree$, denn $(c\bast)^\ast = c \aast$
	\begin{figure}[h]
		\centering
		\begin{tikzpicture}[node distance=5.5cm, auto]
			\node (cb) {$\{1\} \cup c\aast =(c\bast)^\ast$};
			\node (t) [right of=cb] {$T^\ast$};
			\node (xm) [below of=cb] {$M' = \{c \in M \mid xc \in cM \}$};
			\node (mc) [right of=xm] {$M_c = (\{cM \cap Mc\}, \cdot, c)$};
			
			\draw[->] (cb) -- (t) node[ midway,sloped,above,rotate=0] { $\sigma$};
			\draw[ ->] (cb) -- (xm) node[ midway,right,rotate=0] { $\varphi$};
			\draw[ ->] (xm) -- (mc) node[ midway,above,rotate=0] { $\rho_c$};
			\draw[ ->] (t) -- (mc) node[ midway,right,rotate=0] { $t \mapsto .42ctc$};
		\end{tikzpicture}
	\end{figure} 
	
	Es gilt also $\sigma(cv_1 \cdots cv_k) = [v_1] \cdots [v_k] \in T^k \subseteq T^\ast$ mit $T = \{ \varphi(v) \mid  v \in \bast\} \subseteq M$ und $T$ ein endliches Alphabet.
	\newline
	
	Noch zu zeigen bleibt also $K_1, K_2 \in \starfree(T)$ mit struktureller Induktion. Es operiert $\sigma\inv (K_i) \in \starfree(A) \implies \sigma\inv(K_1K_2) \in \starfree(A)$.
	So folgt, dass $\sigma\inv(K_1K_2) = \sigma\inv(K_1) \sigma\inv(K_2) \in \starfree(A)$. 
	\bewiesen
	
	\section{Halbgruppentheorie Teil 2- Vorlesung 13}
	\subsection{Greens Relations Teil 2}
	\subsubsection{Lemma}
	Sei $s \lgreleq t$ und $s \grej t$, dann gilt $s \grel t$ (Analog für $\grer$)
	
	\subsubsection{Lemma}
	Es gilt $\hcal(c) \subseteq M_c = cM \cap Mc$, denn 
	\begin{align*}
		x \in \hcal(c) &\implies x \grer c \land x \grel c \\ 
		&\implies x \lgreheq c \\ 
		&\implies x \in cM \cap Mc
	\end{align*}
	
	\subsubsection{Strukturelle Aussagen}
	Wir wollen die folgenden Sätze zeigen:
	\begin{enumerate}[label = \alph*)]
		\item  \label{Aussage1}$(\hcal(c), \circ, c) = U(M_c, \circ, c)$ ist das Schützenberger Produkt auf $\hcal(c)$
		\item  \label{Aussage2} Seien $s \grer t$ mit $s = tu \land t = sv$. Dann gilt $\rho_v: M_s \to M_t$ ist ein Isomorphismus von $(sM \cap Ms, \circ, s)$ auf $(tM \cap Mt, \cdot, t)$ und vermöge $x \mapsto xv$
		\item  \label{Aussage3} Sei $s = tu$, $t = sv$ und $s \grej t$ sowie $s \grer t$, dann gilt $\rho_v: \lcal(s) \iso \lcal(t)$ ist bijektiv. 
		\item  \label{Aussage4} $\rho_c$ ist respektiv die $\hcal$-Klasse mit $\hcal(x)v = \hcal(xv)$
	\end{enumerate}
	
	\subsubsection{Beweis der Aussage \ref{Aussage1}}
	Für $x \greh c $ folgt $c = yx$ und $yc \in Mc$, da $cM = xM$
	\begin{align*}
		&\implies yc \circ x = yc = c \\
		&\implies x \in U(M_c, \circ, c)
	\end{align*}
	Für die andere Richtung sei $x = x'c = cx'' \in U(M_c)$, dann gilt: 
	\begin{align*}
		&\implies \exists y'c = c y'' \text{ mit } (x'c)y'' = y'(cx'') =c \\
		&\implies c \lgreheq x \quad\text{ Hier gilt auch } x \in M_c
	\end{align*}
	Somit $x \lgreheq c \implies x \greh c$
	\bewiesen
	
	\subsubsection{Beweis der Aussage \ref{Aussage2}}
	Seien $x = x's = sx''$ und $\rho_c(x) = xv$, dann gilt $xv = x'sv \in Mt$. Ferner ist $xv = sx''v = tux''v \in tM$. 
	Es folgt dann $\rho_s(M_s) \subseteq Mt$ und $\rho_v$ ist ein Hom. Dies gilt da:  
	\begin{align*}
		\rho_v(x's)\circ \rho_v(y's) &= x'sv \circ y'sv \\ 
		&= x'y'sv \\
		&= x'y't
	\end{align*}
	Somit ist dann 
	\begin{align*}
		\rho_v(x's \circ y's) &= \rho_v(x'y's) \\
		&= x'y'sv \\
		&= x'y't
	\end{align*}
	Weiter gilt dass $\rho_v: Ms \to Mt$ ein Isomorphismus für $s \grer t$ ist. Um dies zu sehen, betrachten wir $x = x's \in Ms$. Dann gilt 
	\begin{align*}
		\rho_u\rho_v(x) &= \rho_u(x'sv) \\
		&= x'svu \\
		&= x'tu \\
		&=x's \\
		&\implies \rho_v \text{ ist injektiv}
	\end{align*}
	Analog dazu für $\rho_u: Mt \to Ms$ ist injektiv. Also folgt $\rho_u\rho_v = \rho_v\rho_u = id$
	\bewiesen
	
	\subsubsection{Beweis der Aussage \ref{Aussage3}}
	Sei $x \grel s$, aber $Mx = Ms$, dann folgt $Mxv = Msu = Mt$ und ist somit bijektiv, da $\rho_u\rho_v = id$
	\bewiesen
	
	\subsubsection{Beweis der Aussage \ref{Aussage4}}
	Sei $x\grel s$ dann folgt daraus $ xv \grel sv \grel t \implies xv \grej t$. Und somit ist $xv \lgrereq x$, nach dem Lemma \ref{dam} (im Buch 7.42) gilt dann $xu \grer x$ 
	
	%Sorry dass das hässlich is aber ich bekomme es nicht hin dass es zentriert in der Seite eingebettet ist :(
	%Wegen mir können wir das auch raus lassen
	
	\begin{figure}[H]
		\centering
		\begin{tikzpicture}[every fit/.style={inner sep=0pt, outer sep=0pt, draw}]
			
			\begin{scope}[yshift=1.5cm,y=1cm]
				\node [fit={(0,2) (1,3)}, label={[align=left] $\lcal$- Klasse \\ $\quad \quad \downarrow$}] {$s$}; 
			\end{scope}
			
			\begin{scope}[yshift=0.5cm,y=1cm]
				\node [fit={(0,1) (1,3)}, label=center:{$\vdots$}] {};
			\end{scope}
			
			\begin{scope}[yshift=0.5cm,y=1cm]
				\node [fit={(0,0) (1,1)}, label=below:{}] {$x$};
			\end{scope}
			
			\begin{scope}[yshift=3.5cm,y=1cm]
				\node [fit={(1,0) (4,1)}, label=center:{$\cdots$}] {};
			\end{scope}
			
			\begin{scope}[yshift=3.5cm,y=1cm]
				\node [fit={(1,-3) (4,-2)}, label=center:{$\cdots$}] {};
			\end{scope}
			
			\begin{scope}[yshift=3.5cm,y=1cm]
				\node [fit={(4,0) (5,1)}, label=below:{}] {$t$};
			\end{scope}
			
			\begin{scope}[yshift=3.5cm,y=1cm]
				\node [fit={(4,0) (5,-2)}, label=center:{$\vdots$}] {};
			\end{scope}
			
			\begin{scope}[yshift=3.5cm,y=1cm]
				\node [fit={(4,-2) (5,-3)}, label=right:{$\leftarrow \rcal$- Klasse }] {$xv$};
			\end{scope}
			
			\begin{scope}[yshift=2.5cm,y=1cm]
				\node [fit={(1,-1) (4,1)}] { $\vdots$};
			\end{scope}
			
		\end{tikzpicture} 
		
		\caption{Eggbox Diagram}
	\end{figure}
	Die Aussage folgt wenn man $x \grer y \implies xv \grer x \sim y \grer yv$ und \\ 
	$x \grel y \implies xv \grel t \sim y \grer yv$ betrachtet. 
	\bewiesen
	
	\subsubsection{Korrolar}
	Seien $s\grej t \in M$ und $M$ endlich, also $\exists z: s \grer z \grel t$ ($\dcal = \jcal$). 
	Dann ist $s = zv$ und $t =uz$, und somit ist: 
	
	\begin{figure}[h]
		\centering
		\begin{tikzpicture}[node distance=3.5cm, auto]
			\node (lam) {$\lambda_u\rho_v: (\hcal(s), \circ, s)$};
			\node (ht) [right of=lam] {$(\hcal(t), \circ, t)$};
			\node (hz) [below of=lam] {$(\hcal(z), \circ, z)$};
			
			\draw[->] (lam) -- (ht) node[ midway,sloped,above,rotate=0] { $\sim$};
			\draw[  ] (lam) -- (hz) node[midway,sloped, above, rotate=0] { $\sim$};
			\draw[ ->] (lam) -- (hz) node[ midway,left=.2,rotate=0] { $\rho_v$};
			\draw[ ->] (ht) -- (hz) node[ midway,right=.2cm,rotate=0] { $\lambda_u$};
			\draw[  ] (ht) -- (hz) node[midway,sloped, above, rotate=0] { $\sim$};
		\end{tikzpicture}
	\end{figure} 
	
	\subsubsection{Definition Regulär}	
	Ein $s \in M$ heißt regulär, falls $\exists r: srs = s$ gilt. 
	
	\subsubsection{Lemma}
	Es gilt $srs=s \implies s \grer sr$ und $(sr)^2 = sr \in E(M)$	
	
	\subsubsection{Beweis des Lemmas}
	Sei $s\lgrereq s$ und $sr \lgrereq srs = s$. Dann folgt daraus $s \grer sr$. Außerdem ist $(sr)(sr) = (srs) \land sr \in E(M)$
	\bewiesen
	
	\subsubsection{Satz}
	Für ein endliches $M$ sind die folgenden Aussagen äuquivalent für eine $\dcal$-Klasse D: 
	\begin{enumerate}[label=\arabic*.)]
		\item $D$ ist regulär $\iff \exists s \in D$ und $v \in M$ mit $srs = s$ (d.h $\exists s \in D$ und $s$ regulär)
		\item $\exists e \in E(M)$ und $e \in D$
		\item $\forall s \in D: s$ ist regulär
		\item Jede $\lcal$-Klasse in $D$ und jede $\rcal$-Klasse von $D$ enthält eine Idempotenz
	\end{enumerate}
	
	\subsubsection{Beweis des Satzes}
	Sei $srs = s$ regulär und $s \in D \implies sr \grer s$ sowie $sr = e \in E(M)$. Dann folgt 
	\begin{align*}
		&\implies \exists e \in M: e \grer s \text{ und } e^2 = e \\
		&\implies s \in eM \\
		&\implies es = s \text{ da } s=es' \implies es = es' = s
	\end{align*}
	Somit folgt 2.)
	\newline
	
	Wir zeigen jetzt $e^2 = e \grer s$. Betrachte dazu $e=sr \implies s = es = srs$. Daraus folgt $s$ ist regulär und somit dass alle $\rcal$-Klassen zu $e$ reguläre sein müssen.
	Analog gilt die für die $\lcal$-Klassen. \\
	Somit folgt 3.)
	\newline
	
	Also existiert $s \in D$ regulär und somit folgt $\rcal(s)$ und $\lcal(s)$ enthalten eine Idempotenz. \\
	Somit folgt 4.)
	
	\subsubsection{Satz}
	Seien $s,t \in D$ für eine $\dcal$-Klasse $D$ und sei $M$ endlich. Sei weiter $st \in D$, dann ist $D$ regulär und $st \in \rcal(s) \cap \lcal(t)$. 
	Ferner gilt $\exists e \in \lcal(s) \cap \rcal(t)$, und somit $$\exists e \in \rcal(t) \cap \lcal(s) \text{ mit } e=e^2 \implies st \in \rcal(s) \cap \lcal(t)$$
	\begin{figure}[H]
		\centering
		\begin{tikzpicture}[every fit/.style={inner sep=0pt, outer sep=0pt, draw}]
			
			\begin{scope}[yshift=1.5cm,y=1cm]
				\node [fit={(0,2) (1,3)}] {$s$}; 
			\end{scope}
			
			\begin{scope}[yshift=0.5cm,y=1cm]
				\node [fit={(0,1) (1,3)}, label=center:{$\vdots$}] {};
			\end{scope}
			
			\begin{scope}[yshift=0.5cm,y=1cm]
				\node [fit={(0,0) (1,1)}, label=below:{}] {$e$};
			\end{scope}
			
			\begin{scope}[yshift=3.5cm,y=1cm]
				\node [fit={(1,0) (4,1)}, label=center:{$\cdots$}] {};
			\end{scope}
			
			\begin{scope}[yshift=3.5cm,y=1cm]
				\node [fit={(1,-3) (4,-2)}, label=center:{$\cdots$}] {};
			\end{scope}
			
			\begin{scope}[yshift=3.5cm,y=1cm]
				\node [fit={(4,0) (5,1)}, label=below:{}] {$st$};
			\end{scope}
			
			\begin{scope}[yshift=3.5cm,y=1cm]
				\node [fit={(4,0) (5,-2)}, label=center:{$\vdots$}] {};
			\end{scope}
			
			\begin{scope}[yshift=3.5cm,y=1cm]
				\node [fit={(4,-2) (5,-3)}] {$t$};
			\end{scope}
			
			\begin{scope}[yshift=2.5cm,y=1cm]
				\node [fit={(1,-1) (4,1)}] { $\vdots$};
			\end{scope}
			
		\end{tikzpicture} 
	\end{figure}
	
	\subsubsection{Beweis des Satzes}
	Seien $s,t \in D$ mit $st \in D$. Dann ist $s \grej t \grej st$ sowie $st \lgrereq s$ und $st \lgreleq t \implies st \in \rcal(s) \cap \lcal(t)$
	Betrachte nun 
	\begin{align*}
		\rho_t : \lcal(s) &\to \lcal(st) = \lcal(t) \\
		x &\mapsto xt
	\end{align*}
	Dann folgt dass $\exists! e \in \lcal(s): et = t$. Ferner gilt $e \in R(t)$, da $\rho_t$ die $\rcal$-Klasse  enthält und somit folgt 
	\begin{align*}
		e = ty \implies e^2 &= (et)y \\
		&=ty = e
	\end{align*}
	$\implies e \in E(M)$ mit $e \in \lcal(s) \cap \rcal(t)$. 
	\newline
	
	Umgekehrt sei $e^2 = e \in \lcal(s) \cap \rcal(t)$. Zu zeigen ist $st \in \lcal(s) \cap \rcal(t)$. \\
	Wegen $ t \in eM$ und $e^2 = e$ gilt $et =t$.
	Nun folgt durch $e \in \lcal(s)$, dass $\rho_t = st \in \lcal(s)$ ist. \\ 
	Ferner gilt $\rho_t(s) \grer s \implies st \in \lcal(s) \cap \rcal(t)$.
	\bewiesen
	
\section{Monadic Second Order Logic (MSO) - Vorlesung 14}
\subsection{Grundlagen}
\subsubsection{Allgemeines}
Wir bertachten hier nur $(V, E ,\lambda)$ als Modelle mit $E \subseteq V \times V$ und $\lambda : V \to \Sigma$ wobei $\Sigma$ ein endliches Alphabet ist.
Zum Synatx: Es gibt Variablen für Elemente $x$ in Mengen $V$ und für Teilmengen $X$ von $V$ (Monadische Prädikate)

\subsubsection{Freie Variablen}
	Es gilt $\fv(x=y)=\{x,y\}$ und falls $x \in X$ ist $\fv(x \in X) = \{x, X\}$

\subsubsection{Atomare Formeln}
	Folgende Formeln sind alle atomar: $x=y$, $\lambda(x) = a$ mit $a \in \Sigma$ (oft auch: $P_a$), sowie $x$, $T= true$, $\fv(T) = \emptyset$, und $(x,y) \in E$
	

	
\subsubsection{Definitonen der Makros}
\begin{itemize}
	\item $(x = y) \iff \forall x : x \in X \iff x \in Y$ und dann ist $\fv{(x = y ) = \{x,y\}}$
	\item $(x \subseteq y) \iff \forall x : x \in X \implies x \in Y$
	\item $\phi \land \psi := \neg (\neg \phi \lor \neg \psi)$ etc. und  $\forall z \phi := \neg (\exists z (\neg\phi))$
\end{itemize}

\subsubsection{Bemerkung}
	Sei $z \in \{x, X\}$ und $\phi$ eine $\mso$-Formel, dann $\exists z\phi$ mit $\fv(\exists z \phi) = \fv(\phi) \setminus \{z\}$

\subsubsection{Boolsche Formeln}
	Seine $\phi, \psi$ Formeln, dann sind auch $(\phi \lor \psi)$ und $\neg \phi$ Formeln, 
	mit $\fv(\phi \cup \psi) = \fv(\phi) \cup \fv(\psi)$ und $\fv(\neg \phi) = \fv(\phi)$

\subsubsection{Semantik}
	Sei $\phi$ eine Formel mit $\fv(x_1, x_2, X_1, X_2)$ und $I: \fv(\phi) \to V \cup 2^V$ wobei $I(x) \in V$ und $I(X_j) \subseteq V$. \\
	
	Dann heißt $\phi((V,E,\lambda), I)$ ein Wahrheitswert in $\B = \{\bot, \top\}$ mit $1 = \top$ und $0 = \bot$. 
	\begin{itemize}
		\item 	Falls $\phi((V,E,\lambda), I) = \top$, dann ist $(V,E,\lambda) \models_I \phi$ ein Modell für $\phi$ unter $I$. 
		\item	Falls $\fv(\phi) = \emptyset$ und $(V,E,\lambda) \models_{\emptyset} \phi$, dann heißt $(V,E,\lambda)$ Modell von $\phi$ und $\phi \in \mso(V,E,\lambda)$
	\end{itemize}

\subsection{First Order Logic $\fo$}
\subsubsection{Grundlagen}
	$\fo$ ist ein Fragment von $\mso$ ohne die Menge $X$.
	
\subsubsection{Definitonen der Makros}
	\begin{itemize}
		\item  Für $(x,y) \in E$ definieren wir die Relation $(x,y) \in \east$  in $\mso$ als den reflexiven, transitiven Abschluss von $E$
		\item $R(x,Y) = [x \in Y \land \forall z \forall z': z \in Y \land (z,z') \in E \implies z' \in Y]$
		\item $\east(x,y) := [\forall Y (R(x,Y) \implies y \in Y)]$ sowie $\fv(\east) = \{x,y\}$
	\end{itemize}

\subsubsection{$\mso$-Theorie auf $\sigstern$}
	Wir betrachten die Modelle $(V,E,\lambda)$ mit $V= \{1, \ldots, n\}$ und $E=\{(i, i+1) \mid 1 \leq i < n\}$ sowie $$\lambda: \{1, \ldots n\} \to \Sigma$$

\subsubsection{Beispiel}
	Wir betrachten $abcd = a \to b \to c \to d$ mit $\lambda(1) = a$, $\lambda(2) = b$, $\lambda(3) = c$,  und $\lambda(4) = d$. \\ 
	Dann ist $\east = \leq$ auf $\{1, \ldots, n\}$ und $$\phi = \exists x,y: \lambda(x) = b, \lambda(y) = d \land x \leq y$$ und somit $abcd \models \phi$

\subsubsection{Satz}
	Sei $L \subseteq \sigstern$. Dann gilt 
	\begin{align*}
	L \in \reg(\sigma) &\iff \exists \phi \in \mso \text{ mit } \fv(\phi) = \emptyset \text{ und } L=L(\phi) = \{w \mid w \models \phi \}\\
	&\iff \exists \psi \in \mso  \text{ mit } \fv(\psi) \subseteq \{X_1, \ldots, X_n\} 
	\end{align*}
 wobei $\psi$ quantorenfrei ist, und  $L= L(\exists X_1, \ldots, \exists X_n \psi) $
mit 

\subsection{Beweis des Satzes}
\subsubsection{Erste Richtung}
	Sei $L= L(A)$ für einen NFA $A= (Q, \Sigma, \delta, I, F)$. Wir konstruieren jetzt $\phi \in \mso$ mit $$w \in L(A) \iff w \models \phi$$
	Sei dafür $|w| = n$, also $V = \{1, \ldots, n\}$ und $E=\{(i, i+1) \mid 1 \leq i <n\}$ sowie $\lambda(i) = a_i$ für $w = a_1 \ldots a_i \ldots a_n$ mit $a_i \in \Sigma$ und $Q = \{0, \ldots, m\}$. Dann existieren Mengenvariablen $X_0, \ldots, X_m$ und es gilt 
	$$\exists X_q : (1 \in X_q \land q \in I) \land (\exists X_q : n \in X_q \land q \in f) \land  (\bigwedge_{p, q \in Q} (p \not = q \implies x_p \cap x_q = \emptyset))$$
	Dann gilt $$ \forall x \forall y ( \bigwedge_{p, q, a} (x \in x_p, \lambda(x)= a, (x,y \in E),  \bigvee_{(p,q,a) \in \delta} y \in X_q) = \emptyset$$

\subsubsection{Vorbereitungen für zweite Richtung}
	\begin{enumerate}[label= \arabic*.)]
		\item Sei $(x,y) \in E$ dann gilt das Makro $(x,y) \in \east$ genau dann wenn $y = x +1$, $x \leq y \iff (x, y) \in \east$ über $\sigstern$. Insbesondere gilt dann 
				$$ y = x +1 \iff [x \leq y \land \forall z : (x \leq z \leq y \land x \not = z) \implies z = y]$$
		\item Seien hier alle Elemente $\fo$-Variablen. Dann ist $$|x| = 1 \iff [x \not =\emptyset \land \forall y : y \subseteq X \cap Y \not = \emptyset \implies x = y]$$
				Man beachte hierbei dass $\sigstern \not = \emptyset$  ist auch falls $ \Sigma = \emptyset$, da $\emptyset^\ast = \{\varepsilon\}$
		\item Wir erhalten neue atomare Formeln (Achtung DAM Seite 223 Druckfehler): 
			\begin{itemize}
				\item $x < y \iff [\forall x \forall y : x \in X \land y \in Y \implies x < y]$
				\item $a\in X \iff [\exists x \in X \land \lambda(x)=a] $
			\end{itemize}
		\item Boolsche Kombinationen von $\mso$-Formeln sind ''harmless'', da $\reg(\sigstern)$ eine boolsche Algebra ist.
	\end{enumerate}

\subsubsection{Zweite Richtung}
	Man beachte dass ohne Einschränkung gilt: $\varphi : Q_1X_1\ldots Q_nX_n\psi$ und $\psi$ ist quantorenfrei 
	mit $\fv(\varphi) \subseteq \{X_1, \ldots, X_n\}$ und $Q_i \in \{\forall, \exists\}$. 
	\newline
	
	Also können wir davon ausgehen dass sich $\varphi$ in Pränexnormalform befindet und boolsche Kombinationen benutzt sowie atomare Formeln: 
	$x_i \leq x_j$, $a \in x_i$, sowie $x_i < x_j$. 
	Wir definieren nun für $n \geq 0$ ein $\Gamma_n = (\Sigma \times \{0,1\}^n)$. Somit hat $w' \in \Gamma_n$ dann $n+1$ Spuren mit $w_0 \in \Sigma$, 
	mit $w_1', \ldots, w_n' \in \{0,1\}$ und $w \in \Gamma_n^l = (w_0, w_1, \ldots, w_n)$ mit $w_0 \in \Sigma^l$ und $w \in \B^l = \{0,1\}^l$. 
	\newline
	
	Mit Induktion nach $n$ zeigen wir nun $\Gamma_0 = \Sigma$: \\
	Es existiert ein NFA $A_N$ mit $L(A_n) = L(\psi) \subseteq \Gamma_n^\ast$. Hier interpretieren wir $X_i$ durch eine Teilmenge $I(X_i) \subseteq \Gamma^l$ für alle $l$.
	Dann ist $I(X_i) = \{w \in \Gamma_n^l \mid w_i(x)= 1\}$ mit $$w_i(x) = 1 \iff \lambda_i(w_i) = 1$$ für eine Beschriftung $\lambda_i$ der $i$-ten Spur.
	\newline
	
	Sei nun $n= 0$, dann gilt $\fv(\psi) = \emptyset$. Die erzeugte atomare Formel ist $\top = true$ also folgt $L(A_n) = \sigstern$.
	Sei nun $n \geq 1$. Wir konstruieren die atomaren Formeln: $\top, x_i \leq x_j$, $a\in x_i$, und $x_i < x_j$. (Rest verleibt der Übung, siehe DAM).
	\newline 
	
	Dann verbleibt nur noch zu zeigen: $\exists X_n  \psi (x_1, \ldots, x_n)$, denn $\forall X_n \psi = \neg (\exists X_n(\neg\psi))$, da $\reg(\sigstern)$ unter Komplement abgeschlossen ist.
	\newline 
		
	Bis hier hin haben wir $A_n$ mit $L(A_n \subseteq \Gamma_n^\ast)$ und $L(A_n) = L(\psi)$ konstruiert. Sei nun $A_{n+1} = (Q, \Gamma_{n +1}, \delta', I, F)$. 
	Es gilt $\delta \subseteq Q \times \Gamma_n \times Q$. Wir definieren nun $(p, w ', q) \in \delta' \iff \exists b \in \{0,1\}: (p, \binom{w'}{n}, q) \in \delta$.
	Dann gilt $L(A_{n+1}) = L(\exists X_n \psi)$. Nach $n$ Iterationen erhalten wir nun einen NFA $A_0$ mit $L(A_0) \subseteq \Gamma_0^\ast = \sigstern$ und 
	\begin{align*}
		L(A_0) &= L(Q_1X_1\ldots Q_nX_n\psi) \\
		&=L(\phi)
	\end{align*}
	\bewiesen
	
\section{$\fo \implies \starfree $ und das Splitting Lemma - Vorlesung 15}
\subsection{Das Splitting Lemma}
\subsubsection{Allgemeines}
	Für $\sigstern$ gilt: $$\ap = \starfree = \fo[\leq] = \fo^3[\leq] = \ltl = \tl[UX]$$ \\
	
	In dieser Vorlesung: Sei $L \subseteq \sigstern$ in $\fo[\leq]$ definierbar, dann folgt dass $L \in \sf(\sigstern)$. 
	Modelle für $\fo[\leq]$ sind dann Wörter $w \in \sigstern$ mit $$w = (\{1, \ldots, |w|\}, \{(i,j) \mid 1 \leq i \leq j \leq |w|\}, \lambda)$$
	Als atomare Formeln erhalten wir dann ''$x  \leq y$'', ''$\lambda(x) =a$'', und schreiben hierfür $w(x) = a$, sowie $T = true$.
	Unsere Makros sind $y =x+1$, $w(x) \in A$ für $A\subseteq \Sigma$ etc.

\subsubsection{Beispiel}
	Sei $L = b\sigstern bb \sigstern a $ dann gilt $L = b \sigstern \cap \Sigma^+bb\Sigma^+ \cap \sigstern a$

\subsubsection{Aussage des Lemmas}
	Sei $L \in \starfree(\sigstern)$ und es seien $A, B \subseteq \Sigma$ mit $A \cap B = \emptyset$. Dann gilt für $K_i, L_i \in \starfree(\bast)$: $$L \cap \bast A \bast = \bigcup_{\text{endlich}} K_iAL_i$$
	
\subsubsection{Beweis Splitting Lemma - Teil 1\label{Beweis: 1}}
	Ohne Einschränkung: $\bigcup_{\text{endlich}} K_iAL_i = \bigcup_{i = 0}^n K_iAL_i$ mit $K_0 = \sigstern$ und $n \geq 1$, denn $L_0 = \emptyset$ ist erlaubt. 
	\newline
	
	Unter Verwendung von $\ap = \starfree$ ist der Beweis einfach und bleibt zum Selbsttest. Wir geben hier deshalb den direkten Beweis mittels struktureller Induktion an. 
	\newline
	
	Sei $L$ endlich, dann gilt $L \cap \bast A\bast$ ist sternfrei (trivial). Falls $L = L_1 \cup L_2$, $L_i \in \starfree(\Sigma)$, dann $$L_1 \cup L_2 = \bigcup_{j = 1,2} \bigcup K_{i,j}AL_{}i,j$$ Weiter ist $$L' \cdot L'' \cap \bast A \bast = \bigcup_{i \in I} K_i A L_i (L'' \cap \bast) \cup \bigcup_{j \in J}(L' \cap B) K_jAL_j$$ 
	für $L' \cap \bast A \bast = \bigcup_{i \in I} K_iAL_j$ und $L'' \cap \bast A \bast = \bigcup_{j \in J} K_iAL_j$

\subsubsection{Beobachtung}
	Sei $L \cap \bast A \bast = \bigcup_{\text{endlich}} K_iAL_i$. Dann gilt ohne Einschränkung $$K_I \cap K_j = \emptyset$$ für alle $i \not = j$ und $$\bigcup_{i \in I} K_i = \sigstern$$
\subsubsection{Beweis Splitting Lemma - Fortsetzung}
	Ohne Einschränkung gilt $K_i \not = \emptyset$ für alle $i$. 
	\newline
	
	Angenommen $\exists i,j$ mit $i \not = j$ und $K_i \setminus K_j \not = \emptyset \not = K_i \cap K_j$, dann lässt sicht $K_i A L_i$ durch $$(K_i \setminus K_j)AL_i \cup (K_i \cap K_j)AL_i$$ darstellen. Das folgende Bild veranschaulicht dies: 
	
	\begin{figure}[H]
		\centering
		$\sigstern = 
		\begin{array}{c}
				\begin{tikzpicture}[fill=gray]
				\scope
					\clip (2.7,-2.54) rectangle (-1.5,1.5)
							(0,0) circle (1)
							(1.2,0) circle (1);
							(0.6,-1.04) circle (1);
							(2,-2) circle (1);
				\endscope
				\draw (0,0) circle (1) (0,1)  node [text=black,below,shift={(-0.5,-0.5)}] {$K_1$}
						(1.2,0) circle (1) (0,1)  node [text=black,below right,shift={(1.2, -0.5)}] {$K_2$}
						(0.6,-1.04) circle (1) (1.1,-0.6) node [text=black, below right, shift={(-1,-0.5)}] {$K_3$}
						(2,-1.9) circle (1) (1.1,-0.6) node [text=black, shift={(1,-1.1)}] {$K_4$}
						(3.5,-3.5) rectangle (-2,2) node [text=black,shift={(0.5,-0.5)}] {$K_0$};
				\end{tikzpicture}
			\end{array}$
		\end{figure}
	Betrachte $(\sigstern \setminus L) \cap \bast A \bast$ wobei $$L \cap BA\bast = \bigcup_{i = 1}^n K_i A L_i$$ mit $K_i \cap K_j \not = \emptyset$ für alle $i \not = j$ und $\bigcup K_i = \sigstern$ sowie $K_i \not = \emptyset$. Dann ist $\{K_1, \ldots, K_n\}$ eine Partition von $\sigstern$. Somit gilt $$\sigstern \setminus L \cap \bast A \bast = \bigcup_{i = 1}^nK_iA(\bast \setminus L_i)$$ 
	
\subsubsection{Satz}
	Sei $\varphi \in \fo[\leq] = \fo[<]$ und $\Sigma$ ein endliches Alphabet, dann gilt $L(\varphi) \in \starfree(\sigstern)$. Sei $\varphi \in \fo[\leq]$ mit $\fv(\varphi) = \{x_1, \ldots, x_n\} = V$ und $w \in \sigstern$ mit $w = \{1, \ldots, l\}$. Dann ist eine Interpretation der $x_i$ gegeben durch eine Abbildung $$\sigma: V \to \{1, \ldots,l\}$$ wobei $w, \sigma \models \varphi \in \{0,1\} = \B = \{\top, \bot\}$ wohldefiniert ist. 

\subsubsection{Zentrale Idee}
	Kodiere das Paar $(w, \sigma)$ durch Wörter über $\Sigma_V = \Sigma \times \{0,1\}^V$. Dabei haben Wörter über $\Sigma_V$ die Form $$(a_1, \tau_1) \cdots (a_l, \tau_l)$$ mit $a_i \in \Sigma$ und $\tau_i : V \to \{0,1\}$. Konkreter bedeutet das wir kodieren $(w, \sigma)$ durch $$\overline{(w, \sigma)} = (a_1, \tau_1) \cdots (a_{|w|}, \tau_{|w|})$$ wobei $\tau_p(x) =$ ''$\sigma(x) = p$'' $\in \B$ ist. 

\subsubsection{Definition: Menge der Normalformen}
	Es ist $$\overline{(w, \sigma)} = N_V = \bigcap_{x \in V} (\Sigma_V^{x=0})(\Sigma_V^{x=1})(\Sigma_V^{x=0}) \in \starfree(\sigstern)$$

\subsubsection{Definition: Teilalphabet}
	Wir nennen die folgende Menge ein Teilalphabet $$\Sigma_V^{x=b} = \{(a, \tau) \in \Sigma_V \mid \tau(x) = b\}$$ mit $b \in \{0,1\} = \B$

\subsubsection{Semantiken}
	\begin{gather*}
			\rbracedalign
			{}{
				\llbracket \varphi \rrbracket_v &= \{\overline{(w, \sigma)} \in N_V \mid (w, \sigma) \models \varphi \}  \\
				\llbracket x = a \rrbracket_v &= \{\overline{(w, \sigma)} \in N_V \mid w = b_1, \ldots, b_l \in \sigstern \}  \\
				\llbracket x < y\rrbracket_v &= \{\overline{(w, \sigma)} \in N_V \mid \sigma(x) < \sigma(y) \}  \\
				\llbracket \varphi \cup \psi \rrbracket_v &= \llbracket \varphi \rrbracket \cup \llbracket \psi \rrbracket 
			}{ 
				\llbracket \neg \varphi \rrbracket_v &= N_V \setminus \llbracket \varphi \rrbracket \\
				\llbracket \varphi \rrbracket_v &= \{\overline{(w, \sigma)} \in N_V \mid (w, \sigma) \models \varphi \}  
			}{\in \starfree(\sigstern_V)}
	\end{gather*}
	und es gilt $$\llbracket \exists x \varphi \rrbracket_v \in \{ \overline{(w, \sigma)} \in N_V \mid \exists 1 \leq p \leq |w| \text{ und } \overline{(w, \sigma[x_n \to p])} \in \llbracket \varphi \rrbracket_{V \cup [x]} \}$$
	wobei wir $\sigma$ wie folgt definieren: 
	$$\sigma[x_n \to p](y) = 
	\begin{cases}
	\sigma(y) & \text{ falls }  y \not = x_n\\
	\sigma(x_n) = p
	\end{cases}$$
	
\subsubsection{Beweis Fortsetzung (von \ref{Beweis: 1})}
	Ohne Einschränkung können wir annehmen dass $\varphi = Q_1x_1\ldots Q_nx_n \psi$ gilt und $\psi$ ist quantorenfrei sowie $\fv(\psi) = \{x_1, \ldots, x_n\} = V'$ mit 
	$Q_i \in \{\exists, \forall\}$ und $x_i \not = x_j$ für alle $i \not = j$. 
	\newline
	
	Die Behauptung ist richtig falls $\fv(\varphi) = V'$, da dann $\llbracket \varphi \rrbracket_v \in \{\emptyset, \sigstern_V\}$. Es bleibt dann noch zu zeigen dass die Behauptung gilt für 
	$\exists x_n\psi(x_1, \ldots, x_n)$, da $\starfree(\sigstern_V)$ eine boolsche Algebra ist. Weiter gilt:
	\begin{align*}
		&A = \Sigma_V^{x_n = 1} &&B= \Sigma_V^{x_n = 0} &&&N_V \subseteq \bast A \bast &&&&\fv(\exists x_n \psi) = \{x_1\ldots x_{n-1}\}
	\end{align*}
	Mit $A \cup B = \Sigma_V$ und $A \cap B \not = \emptyset$ können wir nun schließen dass 
	\begin{align*}
		\llbracket \exists x \psi \rrbracket_v &= \llbracket \exists x \psi \rrbracket_v \cap N_V \\ 
		&= \bigcup_{\text{endlich}} K_iAL_i \text{ mit } K_i, L_i, \in \starfree(\bast)
	\end{align*}
	Wir erhalten zwei Restriktionen: 
		\begin{align*}
			\pi_0 : B &\iso \Sigma_V \text{ mit } \pi_0(a, \tau) = (a, \tau|_v) \\
			\pi_1 : A &\iso \Sigma_V \text{ mit } \pi_1(a, \tau) = (a_i, \tau|_v) 
		\end{align*}
	Somit kommen wir zu $$\llbracket \exists x \psi \rrbracket = \bigcup_{\text{endlich}} \pi_0(K_i) \pi_1(A)\pi_0(L_i) \subseteq \sigstern_V$$ Denn $\pi_0, \pi_1$ sind nur Umbenennungen von Buchstaben. Wir können daraus schließen dass 
		\begin{align*}
			\pi_0(K_i), \; \pi_1(L_i), \; \pi_1(A) \in \starfree(\Sigma_V)
		\end{align*} gilt. 
		\bewiesen
	
\section{Automatentheoretische Verifikation und Modell Checking- Vorlesung 16}
\subsection{Allgemeines}
\subsubsection{Einführung}
	Wir betrachten diskrete Zeitabstände $0,1,2,3. \ldots$. Nun wollen wir durch Abstraktion eines realen Systems zu endlichen Translationssystemen (= NFA's ohne Endzustände) gelangen. 
	Für ein $t \in \N$ sei $\sX = n\sX t(t) = t+1$ und $\sF t = \mathsf{Future}(t) = \{t' \mid t \leq t'\}$. Unsere Spezifikationssprache ist also ein Fragment von $\mso$ über $\Sigma = 2^P$, wobei $P$ die Menge von Eigenschaften die zur Zeit $t$ wahr sind beschreibt mit $|P| < \infty$
\subsubsection{Anwendung}
	\textbf{Gegeben} sei eine Spezifikation $\varphi$ mit $\varphi \subseteq \sigstern$. Für ein reales System $A$ ist unser Translationssystem dann $L(A) \subseteq \sigstern$.
	\newline
	
		\textbf{Das Problem} ist nun zu entscheiden ob 
	\begin{align*}
	L(A) \subseteq L(\varphi) &\iff L(A) \cap \sigstern \setminus L(\varphi) = \emptyset \\
	&\iff L(A) \cap L(\neg \varphi) = \emptyset
	\end{align*} gilt.
	\newline
	
		\textbf{Die Lösung} dazu bringt die Konstruktion eines NFA $B = (Q, \Sigma, \delta, I, F)$ mit $L(\varphi = L(B))$. Ist dies geschafft, dann ist der Test ob $L(A) \cap L(B) \neq \emptyset$ gilt einfach, und liefert sogar einen Zeugen für den Fehler. 

\subsection{Einführung in $\ltl$}
\subsubsection{Allgemein}
	Es gilt 
	\begin{align*}
		\fo[\leq] \subseteq \starfree(\sigstern) &\implies (aa)^\ast \text{ kann nicht in } \fo[\leq] \text{ akzeptiert werden} \\
		&\implies (ab)^\ast \text{ kann in } \fo[\leq] \text{ akzeptiert werden} 
	\end{align*}
	Beachte: $(ab)^\ast = 1.$ Position ist ein $a$, und die letzte Position ist ein $b$ und $$\mathsf{Globally}(\lambda(t) = a \iff \lambda(\sX t) = b) = \sG(a \iff \sX b)$$ wobei $\sG(\varphi) = \neg \sF \neg \varphi$

\subsubsection{Syntax}
	$\varphi \in \ltl$ ist $\fo[\leq]$-Formel mit genau einer freien Variable.
		\begin{enumerate}[label=\arabic*.)]
			\item Sei $a \in \Sigma$, dann ist $a = a(x) \in \ltl$
			\item Sei $\varphi \in \ltl$, also $\varphi = \varphi(x)$. Dann ist $\sX \varphi(x) \in \ltl$
			\item Seien $\varphi, \psi \in \ltl$, dann ist auch $\varphi \sU \psi \in \ltl$ bzw $\varphi(x), \psi(x) \in \ltl$ auch $(\varphi \sU \psi )(x)\in \ltl$
			\item Seien $\varphi, \psi \in \ltl$, dann ist auch $\varphi \cup \psi \in \ltl$ und $\neg \psi\in \ltl$
		\end{enumerate}
	
\subsubsection{Semantik}
	Sei $\varphi \in \fo[\leq]$ mit $\fv(\varphi) = \{x\}$, dann ist ''$w,x \models \varphi$'' $\in \B$ definiert.
	\begin{itemize}
		\item ''$w,x \models a$'' $\iff$ ''$w(x) = a$'' das heißt $w = a_1 \ldots a_{|w|} \in \Sigma$ und $a_x = a$
		\item ''$w,x \models \sX \varphi$'' $\iff$ ''$w\sX x \models \varphi$'' $\iff$ ''$w, x+1 \models \varphi$''
		\item ''$w, x \models (\varphi \sU \psi)$'' $\iff$ $\exists z : x \leq z \land w, z \models \psi \land \forall y: [x \leq y < z \implies w, y \models \varphi]$
		\item ''$w,x$'' $\models \varphi \#\psi \iff (w,x \models \varphi) \# (w,x \models \psi )$ für $\# \in \{\lor, \neg, \land, \implies, \iff, \ldots\}$
	\end{itemize}

\subsubsection{Beispiele}
		\begin{enumerate}[label=\arabic*.)]
			\item Starte bei einer Sprache $L \in \starfree(\sigstern)$
			\item Konstruiere nun $\varphi \in \ltl$ mit $L(\varphi) = \{(w,1) \mid w \in \sigstern \text{ und } w,1 \models \varphi \text{ und } w \in L\}$
		\end{enumerate}
\subsubsection{Makros}
		\begin{itemize}
		\item $\sF \varphi = \sT \sU \varphi$ mit $\sT = true$
		\item $\sG \varphi = \neg \sF \neg \varphi$
		\item $\neg(\varphi \sU \psi) = \sG \neg \psi \cup (\neg \varphi \sU \neg \varphi)$
		\item $\varphi \sU \psi \iff \psi \lor (\varphi \land \varphi\sX\sU\psi)$
	\end{itemize}
Das unterste Makro ist möglich da $\varphi \sU \psi = (\varphi, \varphi, \ldots,\psi)$ und $$\neg (\varphi \sU \psi) = (\neg \psi, \neg \psi, \ldots, \neg\varphi) \lor (\neg \varphi, \neg \varphi, \ldots, \neg\psi)$$

\subsubsection{Selbsttest und Übung}
	Definiere folgendes: 
		\begin{gather*}
	\rbracedalign
	{}{
		\sY\varphi &= \mathsf{ Yesterday}(\varphi) \\
		\varphi\sS\psi &= \varphi \mathsf{Since}(\varphi)
	}{ 
	}{\in \fo^3[\leq]}
	\end{gather*}
	Beachte das $\fo^k[\leq] \subseteq \fo[\leq]$ und $\varphi \in \fo^k[\leq]$ verwendet nur $k$ viele Namen für die Variablen.
	Ganz allgemein gilt 
	$$\fo^2[\leq] \subsetneqq \fo^3[\leq] = \fo[\leq]$$ und somit auch $$\ltl = \tl[\sX,\sU] = \tl[\sX, \sU, \sY, \sS] \subseteq \fo^3[\leq] \subseteq \fo[\leq] \subseteq \starfree(\sigstern) \subseteq \ap(\sigstern)$$denn es ist $\varphi(\sX\sU)\psi := \sX(\varphi\sU\psi)$ und $\sX\varphi = \bot \sX\sU\varphi$ sowie $\varphi\sU\psi = \psi \lor \varphi \land \varphi \sX \sU \psi$. 

\section{$\ltl = \ap$ - Vorlesung 17}
\subsection{Vorbereitungen zum Beweis des Satzes}
\subsubsection{Satz}
Es gilt $\ltl_\Sigma[\sX \sU] = \ap(\Sigma)$

\subsubsection{Allgemeines}
	Wir wissen dass $$L \in \ap(\Sigma) \iff \exists h : \sigstern \to M$$ mit $h\inv h (L) = L$, $M$ ist endlich und aperiodisch. 
	Dann folgt direkt dass $U(M) = \{1\}$ und $h\inv(1) = \bast$ für $B=\{b \in \Sigma \mid  h(b) = 1\}$. Dann ist 
	\begin{align*}
	L \in \ltl_A[\sX\sU] &\iff L \setminus \{\varepsilon\} \in \ltl_A[\sX\sU] \\
		&\iff  L \setminus \{\varepsilon\} = \{w \in A^+ \mid  w = a_1\ldots a_n \text{ mit } a_i \in A \text{ und } w,1 \models \varphi\}
	\end{align*} wobei $\varphi \in  \ltl_A[\sX\sU]$ ist.


\subsubsection{Strategie}
 Wir zeigen nur $\ap(\Sigma) \subseteq \ltl_\Sigma[\sX\sU]$, da wir wissen dass $$\ltl_\Sigma[\sX\sU] \subseteq \fo_\Sigma^3[\leq] \subseteq \fo_\Sigma[\leq] \subseteq \starfree(\Sigma) \subseteq \ap(\Sigma)\subseteq \ltl[\sX \sU] $$

 Demnach führen wir eine Induktion nach $(|M||\Sigma|)$ durch, denn 
 \begin{align*}
	 (|M'||\Sigma'|) \leq (|M||\Sigma|) \iff  |M'|<|M| \text{ oder } |M'|=|M \text{ und } |\Sigma|\leq |\Sigma|
 \end{align*}
 Weiter gilt ohne Einschränkung: $h:\sigstern \to M$ ist surjektiv. 
 Wir wissen nun dass $\ltl_\Sigma[\sX\sU] \subseteq \ap(\Sigma)$ gilt, falls $M=\{1\}$ oder $\Sigma = \emptyset$ oder $\Sigma = \{a\}$ also $|\Sigma| = 1$: 
 	\begin{align*}
 		L \subseteq A^+ \text{ ist aperiodisch } \iff \exists t \in \N: L \subseteq \{a, a^2, \ldots, a^t\} \cup \{a^m \mid  m \geq t\}
 	\end{align*}
 	Dafür sei $L=\{a^k\}$ mit $k \geq 1$, dann ist $L = L(\underbrace{a \land \sX(a \land \sX(\ldots)}_{\substack{k\text{ -mal}}})\bot)$
 	
\subsubsection{Lemma}
	Sei $A\subseteq \Sigma$ und $L = L(\varphi)$ mit $\varphi \in \ltl_A[\sX\sU]$ und $L \subseteq \aast$. Dann gibt es ein $\varphi_A \in \ltl_\Sigma[\sX\sU]$ mit $L=L(\varphi_A)$

\subsubsection{Beweis des Lemmas}
	Wir zeigen durch strukturelle Induktion
	\begin{align*}
		\varphi = a &\implies\varphi_A \text{ für ein } a \in A \\
		\varphi = \varphi_{1} \lor \varphi_{2} &\implies \varphi_{A} = \varphi_{1A} \lor \varphi_{2A} \\
		(\neg \varphi)_A &= \neg (\varphi_{A}) \land \sG(\bigvee_{a \in A}a)
	\end{align*} Also $L((\neg\varphi)_A) = \ast \setminus L(\varphi_{A})$. Beachte $$(\varphi \sX \sU \psi)_A = (\varphi_{A} \sX \sU \psi_A) \land (\bigvee_{a \in A} a)$$
	\bewiesen
	
\subsubsection{Lemma}
	Es gilt $h\inv h(L) = L$ dann ist $$L = \bigcup_{m \in M}\{h\inv(m) \mid m \in h(L) \}$$
	Somit reicht es also später zu zeigen dass $h^{-1} (m) \in  \ltl_\Sigma[\sX\sU]$. 
	Weiter gilt ab jetzt $\ltl :=  \ltl_\Sigma[\sX\sU]$

\subsubsection{Lemma}
	Es ist $h\inv(1) \in \ltl$
\subsubsection{Beweis des Lemmas}
	Sei $h\inv(1) = \bast$ für $B = \{b \in \Sigma \mid h(b) = 1\}$ und $\bast = L(T)$ in $ \ltl_B[\sX\sU]$. Dann existiert $\varphi_B \in \ltl$ mit $\bast = L(\varphi_B)$.
	Also zeigen wir später nur $$h\inv(m) = \bigcup\{h\inv(m) \cap \sigstern c \sigstern \mid h(c) \neg = 1\} \in \ltl$$ Daher zeigen wir nur für ein festes $c \in \Sigma$, dass $h\inv(m) \cap \sigstern c \sigstern \in \ltl$

\subsubsection{Dirthy Trick}
	Sei $M \cap \sigstern = \{1, c\}$ als Menge definiert und $1 = \varepsilon$ sowie $c \in \Sigma$ und $h(c) = c \neg 1 \in M$. Also ist $M_c = \{cM\cap Mc, \circ, c\}$ ein lokaler Divisor mit $|M_c| < |M|$ und $xc \circ cy = xcy$. 
	\newline
	
	Zeige nun $h\inv(m) \cap \aast c \aast \in \ltl$ mit $A = \Sigma \setminus \{c\}$, und insbesondere $|A| <| \Sigma|$. 
	Es gilt also $$h\inv(m) \cap \aast c \aast  = \bigcup\{(h\inv(p) \cap \aast)c(h\inv(q)\cap \aast) \mid pcq = m\}$$
	Beachte dass $h\inv(p) \cap \aast) \in \ltl_A[\sX\sU] \subseteq \ltl$ und $(h\inv(q)\cap \aast) \in  \ltl_A[\sX\sU]$. Wir zeigen daher nur für $\varphi, \psi \in \ltl$, dass 
	$$(L(\varphi) \cap \aast) c (L(\psi) \cap \aast) \in \ltl$$
	
\subsubsection{Lemma}
	Sei $\varphi \in \ltl$. Dann gilt $\sigstern c (L(\varphi) \cap \aast) \in \ltl$ und $(L(\varphi) \cap \aast)c \sigstern \in \ltl$
	
\subsubsection{Beweis des Lemmas}
	Sei $L(\varphi) \cap \aast = L(\psi)$ für ein $\psi \in \ltl$. Dann gilt $\sigstern c L(\varphi) = L(\sF(c \land \psi))$. 
	Betrachte nun $L(\psi) c \sigstern$. Wir führen eine strukturelle Induktion durch: 
	\newline
	
	Zeige dazu $\forall \psi \in \ltl$ mit $L(\psi) \subseteq \aast$ existiert $\psi^c \in \ltl$, so dass $L(\psi^c) = L(\psi) c \sigstern$. Beachte dass $(\neg \psi )^c = \neg (\psi^c) \land \sF c$, und dann gilt 
	\begin{align*}
		w \in L(\neg \psi) &\iff w \in \aast \lor w = ucv \text{ mit } u \in \aast \text{ und } u \not \in L(\varphi) \\
		&\iff w \in \aast \lor w \in L(\neg(\psi^c))
	\end{align*}

\section{$\ltl \implies \ap$ - Vorlesung 18}
\subsection{Vorbereitungen zum Beweis des Satzes}
\subsubsection{Bezeichnungen}
	Es gilt 
	\begin{itemize}
		\item $\varphi \in \ltl_A[\sX\sU]$ dann ist $L_A(\varphi) = \{w \in A^+ \mid w,1 \models \varphi\}$
		\item $L \in \ltl_A \iff L \setminus \{\varepsilon\} = L_A(\varphi)$ für ein $\varphi \in \ltl_A[\sX\sU]$
		\item $\ltl = \ltl_\Sigma[\sX\sU]$
	\end{itemize}

\subsubsection{Makros}
	\begin{itemize}
		\item $A \in \ltl$ vermöge $A = \bigvee_{a \in A} a$
		\item $A^+ \in \ltl$ vermöge $\neg F(\Sigma \setminus A)$ 
		\item $A \subseteq \Sigma$ und $b \in \Sigma \setminus A$ etwa $b = c \land A = \Sigma \setminus \{c\}$
	\end{itemize}

\subsubsection{Lemma 1}
	 $\forall \varphi \in \ltl_A[\sX\sU]$ existiert ein $\varphi_b \in \ltl$ für das gilt $L(\varphi_b) = \sigstern b L_A(\varphi)$.
	 \newline

	\textit{Hinweis: }Der Beweis dafür wurde bereits in VL17 geführt. Das Lemma wird hier nochmals ausführlicher beschrieben und ist aus Notations Gründen eingefügt. 

\subsubsection{Lemma 2}
	 $\forall \varphi \in \ltl_A[\sX\sU]$ existiert ein  $\varphi_b \in \ltl$ für das gilt $L_A(\varphi)b \sigstern$

\subsubsection{Beweis Lemma 2}
	Mit struktureller Induktion $\varphi = a \in A$. Dann gilt $\varphi_b = a_b = a \land A \sX \sU b$
	\begin{itemize}
		\item $(\varphi \sX \sU \psi)_b = A \land \varphi_b \sX\sU\psi_b \subseteq \{a_1\ldots a_kb\sigstern \mid a_i \in A, k \geq 2\}$
		\item $(\varphi \lor \psi)_b = \varphi_b \lor \psi_b$ 
		\item Weiter ist 
		\begin{align*}
			L((\neg \varphi)_b) &= (\aast \setminus L_A(\varphi))b\sigstern \\
			&= A^+b\sigstern \setminus L_A(\varphi)b\sigstern \\
			&=A^+b\sigstern \cap \sigstern \setminus L_A(\varphi)b\sigstern
		\end{align*} und somit folgt $(\neg \varphi)_b = A \sX\sU B \land A \land \neg (\varphi_b) \in \ltl$
	\end{itemize}
\bewiesen

\subsubsection{Lemma 3}
	Sei $L = L(\varphi)$ mit $\varphi \in \ltl$. Dann existiert ein $\varphi_b \in \ltl$ mit $L(\varphi_b) = Lb\aast$

\subsubsection{Beweis Lemma 3}
	Bleibt der Übung überlassen. \textit{Tipp: } Strukturelle Induktion!

\subsubsection{Satz}
	Es gilt $h\inv h(L) = L \implies L \in \ltl$

\subsubsection{Beweis des Satzes}
	Zeige den folgenden Satz dafür: $\forall m \in M : h\inv (m) \in \ltl$. Für $m = 1$ gilt $h\inv(m) = \bast$ mit $B = \{b \in \Sigma \mid h(b) = a \}$.
	Daher gilt ohne Einschränkung $m \not = 1$ und somit $$h\inv(m) = \bigcup_{c \not \in B } h\inv(m) \cap \sigstern c \sigstern$$ Zeige daher nur den Satz: 
	$$h\inv(m) |çap \sigstern c \sigstern \in \ltl$$

\subsubsection{Lemma}
	Sei $L_A, L'_A \in \ltl_A[\sX\sU]$ dann gilt $L_A c L'_A \in \ltl$

\subsubsection{Beweis des Lemmas}
	Gilt nach Lemma 1 und Lemma 2 und $$L_A c L'_A = L_a c \sigstern \cap \sigstern c L'_A \cap \aast c \aast$$
	\bewiesen

\subsubsection{Lemma}
	Es ist 
	\begin{align*}
	h\inv(m) \cap \sigstern c \sigstern c \sigstern &= h\inv(m) \cap \aast c \sigstern c \aast \in \ltl \\
	&= \bigcup_{p\cdot n = m}(h\inv(p) \cap \aast)(h\inv(n) \cap c \sigstern c \aast)
	\end{align*}

\subsubsection{Beweis des Lemmas}
	Es reicht das nächste Lemma zu zeigen.

\subsubsection{Lemma}
\begin{align*}
	\forall p, n \in M: (h\inv(p) \cap \aast)(h\inv(n) \cap c \sigstern c \aast) &= \\
	(h\inv(p) \cap \aast) c \sigstern \cap \aast(h\inv(n)\cap c \sigstern c \aast) \in \ltl 
	\end{align*}
\subsubsection{Beweis des Lemmas}
Gilt durch Lemma 1, Lemma 2 und den folgenden Satz

\subsubsection{Satz}
	$\forall m \in M : h\inv (n) \cap c \sigstern c \aast \in \ltl$
	
\subsubsection{Beweis des Satzes}
	Ohne Einschränkung gilt $n \not = 1$
	Dann ist $$h\inv(n) \cap c\sigstern c \aast = \bigcup_{qcr = n} L_q c (h\inv(r) \cap \aast) $$ mit $L_q = \{cw \in c \sigstern \mid h(cw)c = qc\}$. Zeige also nur $\forall q : L_q \in \ltl$, dann folgt der Satz mit Lemma 3

\subsubsection{Hauptlemma}
	Sei $L_q \in \ltl$ und $h_A: \aast \to M$ eine Restriktion von $h$ auf $\aast$ und $L_q = \{cw \in c \sigstern \mid h(cw)c = qc\} \in \ltl$. 
	Wir können dann den Beweis für $\ltl = \ap$ mit den lokalen Divisoren führen. 


\subsubsection{Ideen}
\begin{itemize}
	\item Wir wissen das $c\aast$ ein Code ist. Das heißt das $(c\aast)^+$ eine eindeutige Zerlegung in Codewörter hat und $(c\aast)^+ = c\sigstern$ ist, da $A = \Sigma \setminus \{c\}$ gilt. 
	
	\item Falls $h(c\sigstern) \subseteq cM$ ist, dann folgt 
	\begin{align*}
		\rho c h(cw) &= h(cw)c \\ 
		&= ch(w)c \in cMc \subseteq cM \cap Mc
	\end{align*}
	Für $1 \not \in (cM \cap Mc)$ folgt $M_c = (cM \cap Mc, \circ, c)$ ist kleiner als $M$. 
	\item Unseren \textit{Giant-Step} erhalten wir durch $|M_c|<|M|$ und $T= h(\aast) \subseteq M$. Wir interpretieren dabei $T$ als endliches Alphabet, also gilt 
	$$(M_c, T) <(M, \Sigma)$$
\end{itemize}

\subsection{Beweis des Satzes}
\subsubsection{Beweis}
	Wir definieren $g(t) = ctc \in (M_c, \circ, c)$ für $t \in T$. Dann gilt 
	\begin{align*}
		g(t_1\cdots t_k) &= (ct_1c) \circ \ldots \circ (ct_k) \\
		&= ct_1 \ldots ct_kc \in cM \cap Mc \text{ mit } t_i \in T
	\end{align*}
	
	Dann entsteht folgendes Bild:
		\begin{figure}[h]
		\centering
		\begin{tikzpicture}[node distance=6.5cm, auto]
		\node (cs) {$c\sigstern = (c\aast)^+$};
		\node (t) [right of=cs] {$T^+$};
		\node (cm) [below of=cs] {$cM\subseteq \{x \in M \mid xc \in M\}$};
		\node (mc) [below of=t] {$M_c =\{cM\cap Mc, \circ, c\}$};
		
		
		\draw[->] (cs) -- (t) node[ midway,sloped,above,rotate=0] { $\sigma$};
		\draw[ ->] (cs) -- (cm) node[ midway,left=.2,rotate=0] { $h$};
		\draw[ ->] (cm) -- (mc) node[ midway,sloped,below,rotate=0] {$\rho_c = xc$ };
		\draw[ ->] (t) -- (mc) node[midway, right, rotate=0] { $g$};
		\end{tikzpicture}
	\end{figure} 
	
	Es gilt also 
		\begin{align*}
			g\sigma(cw) &= g(t_1\cdots t_k) \\ 
			&= (ct_1)\circ\ldots \circ (ct_kc) \\
			&= h(cw)c \\ 
			&= \rho_ch(cw)
		\end{align*}
		falls $cw = cv_1 \ldots cv_k$ und $t_i = h_A(v_i)$. Also gilt 
		\begin{align*}
				L_q &= \{cw \in c\sigstern \mid \rho_c h(cw) = qc\}\\ 
				&=\{cw \in c\sigstern \mid cw \in h\inv \rho_c\inv(qc)\} \\
				&= \{cw \in c\sigstern \mid cw \in \sigma\inv g\inv(qc)\}
		\end{align*}
		Wir können Induktion verwenden da $(|M_c|,|T|) < (|M|, |\Sigma|)$ und $g\inv(qc) \in \ltl_T[\sX\sU]$. 
		Dann bleibt nun noch das folgende Lemma zu zeigen: 

\subsubsection{Lemma}
	Sei $k \in L_T(\varphi) \subseteq T^+$ mit $\varphi \in \ltl_T[\sX\sU]$, dann folgt $\sigma\inv(k) \in \ltl$		

\subsubsection{Beweis des Lemmas}
	Mittels struktureller Induktion: \\
	Sei $\sigma\inv(t)$ für $t \in T$ dann $\sigma\inv(t) = ch_A\inv(t)$. Nach dem \textit{Baby-Step} gilt: 
	\begin{align*}
		h_A\inv(t) \in \ltl_A[\sX\sU]
	\end{align*}
	somit folgt die Behauptung, denn 
	\begin{align*}
		\sigma(cv_1\cdots cv_k) = h_A(v_1)\ldots h_A(v_k) \in T^+
	\end{align*}

\subsubsection{Beweis Fortsetzung}
	Dann ist $(|M|, |A|) < (|M|,|\Sigma|)$. Genauer gilt: $$\forall \varphi \in \ltl_T[\sX\sU] \exists \varphi_T \in \ltl$$ mit $\sigma\inv(L_t(\varphi)) = L(\varphi_T)$. 
	Es folgt: 
	\begin{align*}
		\varphi &= t \implies \varphi_T = c \land \sX\psi \text{ mit } L(\varphi) = h_A\inv(t) \setminus \{c\}  \\
	\end{align*} bzw. 
	\begin{align*}
		\varphi_T &= (c \land \neg \sX T) \lor (c \land \sX\psi_T) \text{ falls } h_A(\varepsilon) = t = 1
	\end{align*} 
	Somit ist $(\varphi \lor \psi)_T = \varphi_T \lor \psi_T$ sowie $(\neg \varphi)_T = c \land \neg(\varphi_T)$, denn $$\sigma\inv(T\setminus K) = c \sigstern \setminus \sigma\inv(K)$$ wobei das $\sigma$ surjektiv ist. 
	\newline
	
	Man kann sich das dann wie folgt vorstellen:
	%Das wird noch angepasst
		\[ \begin{array}{*{9}{r}}
		cv_1 &cv_2 &\ldots &cv_i &\ldots &cv_j &\ldots &cv_k\\
		t_1 & \rnode{T2}{t_2} &\ldots & \rnode{Ti}{t_i }&\ldots &\rnode{Tj}{t_j} &\ldots &t_k \\[0.5ex]
		& & & & & & & & \makecell{\rnode{Aj}{\psi}\\
			\rnode{Ai}{\varphi}\\
			\rnode{A2}{\varphi}}
		\end{array}
		\psset{linewidth=0.6pt, linejoin=1, arrows=->, arrowinset=0.12, angleA=-90, angleB=180, nodesep=3pt}
		\foreach \s in {2,i, j}{\ncangle{T\s}{A\s}}
		\]
		
	und somit ist $$(\varphi\sX\sU \psi)_T = (x \land (\neg c \lor \varphi_T) \sX\sU(c \land \psi_T)$$
	\bewiesen
	
	



\end{document}


