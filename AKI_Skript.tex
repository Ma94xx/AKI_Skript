\documentclass[12pt, german]{article}
\usepackage[ngerman]{babel}
\usepackage[T1]{fontenc}  
\usepackage[utf8]{inputenc}
\usepackage{amsmath}
\usepackage{dsfont}
\usepackage{array}
\usepackage{amssymb}
\usepackage{enumitem}
\usepackage{upgreek}
\usepackage{graphicx}
\usepackage{pdfpages}
\usepackage{listings}  
\usepackage{mathtools}
\usepackage{listings}
\usepackage{endnotes}
\usepackage{color}
\usepackage{tasks}
\usepackage{mathtools}
\usepackage{forest}
\usetikzlibrary{backgrounds,fit}
\usepackage[outline]{contour}
\usetikzlibrary{shapes.multipart}
\usetikzlibrary{shapes,fit}
\usetikzlibrary{fit}
\usepackage{wrapfig}

%Für Binärbaum
\usepackage{tikz}
\tikzset{
	treenode/.style = {align=center, inner sep=0pt, text centered, font=\upshape},
	arn_b/.style = {treenode, circle, black, font=\upshape, draw=black, fill=okmama, text width=1.5em, thick},
	arn_x/.style = {treenode, rectangle, draw=black, minimum width=0.5em, minimum height=0.5em}
}

\forestset{%
	circle subtree/.style={%
		before typesetting nodes={%
			no edge,
			before computing xy={/pgf/inner ysep/.get=\savedinnerysep, l'=\savedinnerysep},
			replace by/.wrap pgfmath arg={%
				[\phantom{##1}, circle, append, fit=rectangle, no edge,
				before drawing tree={%
					tikz+={%
						\begin{scope}[on background layer]
							\node (n) [draw, fill=gray!25, inner sep=\savedinnerysep, rounded corners, fit=(!1) (!L) (!F)] {} ;
							\path [draw, \forestoption{edge}] (!u.parent anchor) -- (n.north -| !1.child anchor)\forestoption{edge label};
						\end{scope}
					},
				},
				]
			}{content()}
		},
	},
}
%Zum runden
\DeclarePairedDelimiter{\ceil}{\lceil}{\rceil}

%Document Feineinstellungen
\usepackage[a4paper, left=2cm, right=2cm, top=2.5cm]{geometry}

%Deaktiviert Seitenummerierung
\pagenumbering{gobble}

%Farben für Einstellungen der Code Fragmente
\definecolor{mygreen}{rgb}{0,0.6,0}
\definecolor{mygray}{rgb}{0.5,0.5,0.5}
\definecolor{okmama}{RGB}{150,170,120}
\definecolor{character}{RGB}{179, 178, 255}

%Einstellungen für Code Fragmente
\lstset{ 
	backgroundcolor=\color{white},   % choose the background color; you must add \usepackage{color} or \usepackage{xcolor}; should come as last argument
%	basicstyle=\footnotesize,        % the size of the fonts that are used for the code
%	breakatwhitespace=false,         % sets if automatic breaks should only happen at whitespace
%	breaklines=true,                 % sets automatic line breaking
	captionpos=b,                    % sets the caption-position to bottom
	commentstyle=\color{mygreen},    % comment style
	deletekeywords={...},            % if you want to delete keywords from the given language
	escapeinside={(*}{*)},          % if you want to add LaTeX within your code
	extendedchars=true,              % lets you use non-ASCII characters; for 8-bits encodings only, does not work with UTF-8
	frame=single,	                   % adds a frame around the code
	keepspaces=true,                 % keeps spaces in text, useful for keeping indentation of code (possibly needs columns=flexible)
	keywordstyle=\color{blue},       % keyword style
	language=Octave,                 % the language of the code
	morekeywords={*, Algorithm, ...},            % if you want to add more keywords to the set
	numbers=left,                    % where to put the line-numbers; possible values are (none, left, right)
	numbersep=5pt,                   % how far the line-numbers are from the code
	numberstyle=\tiny\color{mygray}, % the style that is used for the line-numbers
	rulecolor=\color{black},         % if not set, the frame-color may be changed on line-breaks within not-black text (e.g. comments (green here))
	showspaces=false,                % show spaces everywhere adding particular underscores; it overrides 'showstringspaces'
	showstringspaces=false,          % underline spaces within strings only
	showtabs=false,                  % show tabs within strings adding particular underscores
	stepnumber=1,                    % the step between two line-numbers. If it's 1, each line will be numbered
	stringstyle=\color{mymauve},     % string literal style
	tabsize=2,	                   % sets default tabsize to 2 spaces
	title=\lstname                   % show the filename of files included with \lstinputlisting; also try caption instead of title
}
%Einstellungen für Tasks
\settasks{
	label-width=4ex,
	label-offset = 0pt,
	item-indent = 1em,
}




%Global Align
\newcommand*{\LongestName}{\ensuremath{h(x)+g(x)}}% function name
\newcommand*{\LongestValue}{\ensuremath{-1}}% function value
\newcommand*{\LongestText}{fermentum fringilla mauris }%

\newlength{\LargestNameSize}%
\newlength{\LargestValueSize}%
\newlength{\LargestTextSize}%

\settowidth{\LargestNameSize}{\LongestName}%
\settowidth{\LargestValueSize}{\LongestValue}%
\settowidth{\LargestTextSize}{\LongestText}%

% Choose alignment of the various elements here: [r], [l] or [c]
\newcommand*{\MakeBoxName}[1]{{\makebox[\LargestNameSize][r]{\ensuremath{#1}}}}%
\newcommand*{\MakeBoxValue}[1]{\ensuremath{\makebox[\LargestValueSize][l]{\ensuremath{#1}}}}%
\newcommand*{\MakeBoxText}[1]{\makebox[\LargestTextSize][l]{#1}}%

%Isomorph zeichen
\newcommand\iso{\xrightarrow{
		\,\smash{\raisebox{-0.65ex}{\ensuremath{\scriptstyle\sim}}}\,}}

%Erzeugt das "bewiesen" Kästen rechts unten
\newcommand{\bewiesen}{

\begin{flushright}
		$\square$  \\
\end{flushright}}


\newcommand\TBox[3][]{%
	\tikz\node[draw,thick,text width=#2,#1] {#3};}


\title{Inoffizielles Skript für Algebra \& Kombinatorik}

\author{Maximilian Kurz \and Amelie Heindl}

\setlength\parindent{0pt}
\begin{document}
\maketitle
\section{Grundlagen - Vorlesung 1}
\subsection{Diedergruppen}
\subsubsection{Definition}
		   Eine Diedergruppe ist die Bewegungsgruppe des regelmäßigen $n$-Ecks mit $n \geq 3$. Dabei ist die Gruppenverknüpfung die Drehung oder Spiegelung des $n$-Ecks. \\ 
		   
		   Das $n$-Eck wird durch Knoten und Kanten beschrieben und eine Drehung/Spiegelung wird durch eine Abbildung wie folgt definiert: 
		   \begin{align*}
		   	\text{Knoten: } &\mathbb{Z}/n\mathbb{Z} \\
		   	\text{Kanten: } &\{ \{i, i+1\} \mid i \in \mathbb{Z}/n\mathbb{Z} \}  \\
		   	\text{Abbildung von Knoten in Knoten: } &f: \mathbb{Z}/n\mathbb{Z} \iso \mathbb{Z}/n\mathbb{Z}\\
		   \end{align*}
		   
		   Zu jeden $n$ gehört eine $D_n$ mit $|D_n| = 2n$

\subsubsection{Eigenschaften von Diedergruppen}
		Wir bezeichnen die Spieglung mit $\sigma$ und die Drehung im Uhrzeiger mit $\delta$. Dann gelten folgende Regeln:
		
		\begin{align*}
			\sigma^2 = id = \delta^n = \delta^0 \\ 	 
		\end{align*}
		In Worten bedeutet dies, dass $n$-faches drehen oder doppeltes Spiegeln eines  $n$-Ecks keine Änderung bewirkt.  \\
		Daraus lässt sich dann folgender Satz ableiten mit $\forall \varepsilon, \varepsilon', m, m' \in \mathbb{Z}$: 
		
		\begin{align*}
			\sigma^\varepsilon\delta^m = \sigma^{\varepsilon'}\delta^{m'} \iff \varepsilon \equiv \varepsilon' (2) \text{ und } m \equiv m' (n)
		\end{align*}
		Weiter gilt, dass $\sigma\delta = \delta^{-1}\sigma$, also dass erst spiegeln und dann drehen den selben Effekt hat wie in die andere Richtung drehen und dann spiegeln. Im folgenden wird gezeigt weshalb dies gilt
		
		\begin{align*}
			\sigma \delta(j) = \sigma(j+1) = -(1 + j) = -j -1 = \delta^{-1}(-j)=\delta^{-1}\sigma(j)
		\end{align*}
		\bewiesen
		Dieser Satz gibt uns also die Möglichkeit das $\sigma$ nach rechts zu schieben.
		
\subsubsection{Fixpunkte in Diedergruppen}
		Ein Fixpunkt $x$ einer Abbildung ist ein Punkt bei dem $f(x) = x$ gilt. Für Diedergruppen fragen wir uns also ab wann 
		\begin{align*}
			\sigma\delta^i(j) = j = \sigma(i+j) = -i -j
		\end{align*} gilt. 
		Dies ist der Fall, wenn $-i = 2j \in \mathbb Z /n \mathbb Z$ mit ungeradem $n$ gilt. Bei Spiegelungen existiert also genau ein Knoten $j$ für alle $i$. Falls $n$ gerade ist, existieren entweder genau $2$ (das wäre dann $-i = 2j$ und $ i + \frac{n}{2}$) oder gar kein Fixpunkt. Für Drehungen ist nur $n$ bzw. $id$ eine Fixpunkt. 
		
		
\section{Gruppentheorie 1 - Vorlesung 2}
\subsection{Untergruppen}
\subsubsection{Definition}
		Sei im folgenden $G=(G, \cdot, 1)$ eine Gruppe. Dann gilt 
		\begin{align*}
			H \subseteq G \text{ ist eine Untergruppe} \iff H \not = \emptyset \text{ und } \forall g, h \in H : gh^{-1} \in H
		\end{align*}
		$H$ ist also eine Untergruppe von $G$, genau dann wenn $H$ nicht leer ist und für alle Elemente Inverse vorhanden sind.  


\subsection{Nebeklassen}
\subsubsection{Definition}
		Für eine Untergruppe $H \leq G$ sind die folgenden Nebenklassen definiert
		\begin{align*}
			 G/H &= \{gH \subseteq G \mid g \in G\} \quad \text{Rechtsnebenklasse } \\
			 H/G &= \{Hg \subseteq G \mid g \in G\} \quad \text{Linksnebenklasse } 
		\end{align*}
		wobei $gH$ die Multiplikation von  g mit allen Elementen aus $H$ ist. 

\subsubsection{Größe von Nebenklassen}
		Wir definieren die folgende Abbildung
		\begin{align*}
			\forall f,g \in G \text{ ist } (f^{-1}g \cdot): gH \iso fH \text{ bijektiv mit der Umkehrabbildung } (gf^{-1}\cdot)
		\end{align*}
		Wir können dann folgern dass 
		\begin{align*}
			gH \iso H \iso Hf \quad \text{ für alle } f,g \in G
		\end{align*}
		Es gilt also $|gH| = |Hf| = |H|$

\subsubsection{Gleichheit von Nebenklassen}
		Im Folgenden wird gezeigt dass $gH \cap fH \not = \emptyset \iff gH = fH$ gilt.
		Die Rückrichtung ist hierbei trivial, da der Schnitt von $gH$ und $fH$ bei Gleichheit offensichtlich nicht leer ist. \\
		
		Für die Hinrichtung seinen $h_1, h_2 \in H$ mit $gh_1 = fh_2$
		\begin{align*}
			&\implies g=fh_2h_1^{-1} \in fH \\
			&\implies gH \subseteq fH. 
		\end{align*}
		Analog dazu gilt $fH \subseteq gH \implies gH = fH$
		\bewiesen
		
\subsubsection{Graphische Interpretation von Nebenklassen}
Sei $G= (G, \cdot, 1)$ eine Gruppe und $H\leq G$
\begin{figure}[h!]
	\centering
	$G=\begin{array}{c}
	
	\begin{tikzpicture}[every fit/.style={inner sep=0pt, outer sep=0pt, draw}]
	\begin{scope}[yshift=1.5cm,y=1cm]
	\node [fit={(0,0) (1,3)}, label=center:{$H$}] {};
	\end{scope}
	\begin{scope}[yshift=1.5cm,y=1cm]
	\node [fit={(1,0) (5,1)}, label=center:{$fH$}] {};
	\end{scope}
	\begin{scope}[yshift=2.5cm,y=1cm]
	\node [fit={(1,0) (5,1)}, label=center:{$\vdots$}] {};
	\end{scope}
	\begin{scope}[yshift=3.5cm,y=1cm]
	\node [fit={(1,0) (5,1)}, label=center:{$gH$}] {};
	\end{scope}
	
	\end{tikzpicture} 
\end{array}
= 
\begin{array}{c}
	\begin{tikzpicture}[every fit/.style={inner sep=0pt, outer sep=0pt, draw}]
	\begin{scope}[yshift=1.5cm,y=1cm]
	\node [fit={(0,0) (1,3)}, label=center:{$H$}] {};
	\end{scope}
	\begin{scope}[yshift=1.5cm,y=1cm]
	\node [fit={(1,3) (2,0)}, label=center:{$Hg$}] {};
	\end{scope}
	\begin{scope}[yshift=1.5cm,y=1cm]
	\node [fit={(2,0) (3,3)}, label=center:{$\ldots$}] {};
	\end{scope}
	\begin{scope}[yshift=1.5cm,y=1cm]
	\node [fit={(3,3) (4,0)}, label=center:{$Hf$}] {};
	\end{scope}
	
	\end{tikzpicture} 
\end{array}$
\caption{Zerlegung von $G$ in Links bzw. Rechtsnebenklassen}
\end{figure}

Wir sehen dass $|H|=|gH| = |Hf|$ gilt


\subsection{Representantensysteme}		
\subsubsection{Definition}
	Sei $R_H := \{r(gH) \mid gH \in G/H\} \subseteq G$ mit der folgenden bijektiven  Abbildung
	\begin{align*}
		R_H \iso G/H \\
		r(gH) \mapsto gH 
	\end{align*}
\subsection{Satz von Lagrange}		
\subsubsection{Definition und Beweis}
	Wir definieren die folgende Abbildung
	\begin{align*}
		\lambda: (G/H) \times H &\iso G \text{ ist bijektiv} \\
		(gH, h) &\mapsto r(gH) \cdot h
	\end{align*}
	 
	Nun bezeichnen wir mit $[G:H] = |G/H|$ den Index von $H$ in $G$. \\ 
	Wenn jetzt $R = \{r(gH) \mid g \in G \}$ ein Linksrepresentantensystem (LRS) ist, \\ 
	dann ist $R^{-1} = \{r(gH)^{-1} \mid g \in G \}$ ein RRS. 
	Denn es gilt $(rH)^{-1} = H^{-1}r^{-1} = Hr^{-1}$, da $H=H^{-1}$.
	Also gilt $|G/H| = |H/G|$ \\
	
	Wenn für $|G| < \infty $ gilt, ist $|H| < \infty $ und $|G| = |G/H| \cdot |H|$ und 
	außerdem sind $\{gH \mid g \in G \}$ und $\{Hg \mid g \in G \}$ Partitionen. Weiter gilt $|gH| = |Hf| \;\;\forall g,f \in G$. \\
	Also gilt $|G| = [G:H] \cdot |H|$ 


\subsection{Homomorphiesatz}		
\subsubsection{Grundlagen Homomorphismus}
	Eine Abbildung $\varphi: G \to F$ heißt Homomorphismus\footnote{abgekürzt ab jetzt Hom.} falls $\forall g,h \in G : \varphi(gh) = \varphi(g)\varphi(h)$ \\
	\begin{enumerate}[label=\roman*)]
		\item $\varphi(1_G) = 1_H$ ~\par
			Denn $g=1\cdot g \implies \varphi(g)=\varphi(1)\varphi(g) \implies 1_H = \varphi(g) \varphi(g)^{-1} = \varphi(1)$
		
		\item $\varphi(g^{-1})=\varphi(g)^{-1}$ ~\par
			Denn $\varphi(1)=\varphi(gg^{-1}) \implies 1_h = \varphi(g)\varphi(g^{-1}) \implies \varphi(g)^{-1} = \varphi(g^{-1})$
	\end{enumerate}

\subsubsection{Lemma}
	Ein Hom. $\varphi: G \to F$ ist injektiv $\iff$ $ker(\varphi) = \{g \in G \mid \varphi(g)= 1\} = \{1\}$ 
	
	Im Folgenden geben hierfür einen Beweis an in dem wir $\varphi(g) = \varphi(h) \iff \varphi(gh^{-1}) = \{1\} $ zeigen:
	\begin{align*}
		\varphi \text{ ist injektiv } &\implies ker(\varphi) = \{1\} \\
		&\implies [\varphi(gh^{-1}) = \{1\} \implies gh^{-1} = 1 \implies g = h] \\ 
		&\implies \varphi \text{ ist injektiv}
	\end{align*}

\subsubsection{Definition Homomorphiesatz}
	Sei $\varphi: G \to F$ ein Hom.  und $N = ker(\varphi)$. \\
	Dann induziert $\varphi$ ein Hom. $\overline{\varphi} : G/N \to F$ der injektiv ist durch $\overline{\varphi}(gH) = \varphi(g)$ 
	\newline
	
	Also induziert $\varphi$ einen Isomorphismus $\overline{\varphi}: G/N \iso im(G) = \{\varphi(g) \mid g \in G\} \leq F$

\subsubsection{Beweis Homomorphiesatz}
	\begin{enumerate}[label=\arabic*)]
		\item $\overline{\varphi}$ ist wohldefiniert ~\par
				Sei $f\in gH \implies \varphi(f) \in \varphi(g)N = \{\varphi(g)\}$
		
		\item $\overline{\varphi}$ ist injektiv ~\par
			$ker(\overline{\varphi}) = ker(\varphi) = N = 1_{G/N}$
	\end{enumerate}

\subsubsection{Beispiel für den Homomorphiesatz}
	Wir definieren 
		\begin{align*}
			exp = \begin{cases}
			\mathbb{R} \to \mathbb{C}^\ast\\
			x \mapsto e^{2\pi i x}
			\end{cases}
		\end{align*}
	Dann ist $ker(exp) = \mathbb Z $.\\ 
	 Also ist $\mathbb R / \mathbb Z \cong \{z \in \mathbb C \mid |z| = 1\}$	



\subsection{Normalteiler}		
\subsubsection{Definition}
	$H \subseteq G$ heißt Normalteiler ($H\trianglelefteq G$) falls $\forall g \in G : gH = Hg$. 

\subsubsection{Satz zu Normalteilern}
	Sei $H \subseteq G$, dann sind die folgenden Aussagen äquivalent: 
	\begin{enumerate}[label=\arabic*)]
		\item $H\trianglelefteq G$
		\item $H \leq G$ und $gH \cdot fH = gfH$ definiert eine Gruppenstruktur auf $G/H$
		\item Es existiert ein Hom. $\varphi: G \to F$ mit $ker(\varphi) = \{g \in G \mid \varphi(g)= 1\} = H$
		\item $\forall g \in G : gHg^{-1} \subseteq H$ und $H \leq G$
	\end{enumerate}

	Im Folgenden wird die Äquivalenz der folgenden Aussagen gezeigt: 
	\begin{itemize}
		\item $1)\rightarrow 2)$ ~\par
			$H\trianglelefteq G \implies gHfH = g(Hf)H = g(fH)H = gfH$
			\begin{align*}
				&\text{ Neutrales Element: } &&1_{G/H} = H \\
				&\text{ Inverse: } &&(gH)^{-1} = H^{-1}g^{-1}= Hg^{-1}=g^{-1}H \\
				&\text{ Assoziativität: } &&(fH)(gH)(hH)= fghH\\
			\end{align*}
			
		\item $2)\rightarrow 3)$ ~\par
		Sei $\varphi : G \to G/H $ und $\varphi(g) = gH$, dann
			\begin{align*}
				ker(\varphi) = \{g \in G \mid gH = H\} = H = 1_{G/H}
			\end{align*}
		
		\item $3)\rightarrow 4)$~\par
		Sei $H = ker(\varphi)$, dann
			\begin{align*}
				\varphi(gHg^{-1}) = \varphi(gg^{-1}) = \varphi(1) \subseteq H 
			\end{align*}
		
		\item $4)\rightarrow 1)$~\par
		$\forall g \in G : gHg^{-1} \subseteq H$
			\begin{align*}
			&\implies  \forall g \in G : H \subseteq g^{-1}Hg \\
			&\implies \forall g \in G : H \subseteq gHg^{-1} \\
			&\implies gHg^{-1} = H \\
			&\implies gH = Hg
			\end{align*}
	\end{itemize}



\subsection{G-Mengen}		
\subsubsection{Definition}
	Sei $G = (G, \cdot, 1)$ eine Gruppe und $X$ eine Menge.
	Dann operiert $G$ auf $X$ falls folgende Abbildung existiert 
	\begin{align*}
		\cdot : G \times X \to X \\ 
		(g,x) \mapsto gx
	\end{align*}
	Mit der Eigenschaft $g(h(x)) = gh(x)$ und $g(1) = id_x$

\subsubsection{Homomorphismus auf G-Mengen}
	Ein Hom. von $G$-Mengen $X,Y$ ist eine Abbildung 
	\begin{align*}
		f: X \to Y \text{ mit } f(gx) = gf(x)
	\end{align*}

\subsubsection{Fixpunkte in G-Mengen}
	Sei $X$ eine  $G$-Menge. Dann heißt $x$ ein Fixpunkt von $g \in G$ falls $gx=x$. \\
	Weiter ist die Menge der Fixpunkte eine Untergruppe von $G$
	\begin{align*}
		Fix(x) = \{g \in G \mid  gx = x\} \leq G 
	\end{align*}


\subsection{Bahnenlemma}		
\subsubsection{Definition einer Bahn}
	Wir definieren eine Bahn wie folgt $$Bahn(x) = Orbit(x) = G\cdot x = \{gx \mid  g \in G \}$$
	
	
	Ebenfalls gilt das $Gx \cong G/Fix(x)$ ein Isomorphismus von $G$-Mengen ist, wobei $Gx$ ist eine $G$-Menge ist. 
	Dies ist wie folgt zu begründen. Wir betrachten die folgende Abbildung
	\begin{align*}
		G/Fix(x) \to Gx \\ 
		gFix(x) \mapsto gx
	 \end{align*}
	 Diese ist wohldefiniert, da $h \in gFix(x) \implies h=gf$ mit $f \in Fix(x)$. \\ 
	 Also $h(x) = gf(x) = g(x)$ 


\subsubsection{Definition Bahnenlemma}
	Das Bahnenlemma sagt folgendes aus $$Gx \cap Gy \not = \emptyset \iff Gx = Gy$$
	Im Allgemeinen also $$ X = \dot{\bigcup} \{Gx \mid  x \in X \}$$. Wobei hier die disjunkte Vereinigung gemeint ist. 
\subsubsection{Beweis Bahnenlemma}
	Die Rückrichtung ist hier ebenfalls trivial (wie bei dem sehr ähnlichen Resultat über Nebenklassen). Wir zeigen deshalb die andere Richtung. \\ 
	
	Sei $gx = hy$ für $g,h \in G$
	\begin{align*}
		&\implies h^{-1}g(x) = y \\
		&\implies y \in Gx \\
		&\implies Gy \subseteq Gx
	\end{align*}
	Analog dazu $Gx \subseteq Gy \implies Gx = Gy$

\subsubsection{Graphische Interpretation einer Bahn}

Sei $X$ eine $G$-Menge und 

\begin{figure}[h!]
	\centering
	$X=\begin{array}{c}
	
	\begin{tikzpicture}[every fit/.style={inner sep=0pt, outer sep=0pt, draw}]
	\begin{scope}[y=1.5cm]
	\node [fit={(4,0) (7,1)}, label=center:{$Gx_1$}] {};
	\end{scope}		
	\begin{scope}[yshift=1.5cm,y=1cm]
	\node [fit={(0,-1.5) (4,1)}, label=center:{$Gx_2$}] {};
	\node [fit={(4,0) (7,1)}, label=center:{$Gx_3$}] {};
	\end{scope}
	\begin{scope}[yshift=2.5cm,y=1.2cm]
	\node [fit={(0,0) (5,1)}, label=center:{$Gx_4$}] {};
	\node [fit={(5,0) (7,1)}, label=center:{$Gx_5$}] {};
	\end{scope}
	\end{tikzpicture}
\end{array}$
\caption{Disjunkte Zerlegung von $X$ in Bahnen}
\end{figure}
Wir wissen, dass $|Gx|=|G/Fix(x)|$ ein Teiler von $|G|$ ist. Also gilt $|G| = p$ wobei $p$ eine Primzahl ist, und daraus folgt $\forall x \in X : |Gx|=1$ oder $|Gx|=p$

\subsection{Satz von Cauchy}		
\subsubsection{Definition}
	Sei $G$ eine endliche Gruppe und $p \in \mathbb P$ mit $p \mid |G|$. \\
	Dann existiert $g \in G$ mit $g \not = 1$ und $ g^p = 1$. Also $ord(g)= p$
\subsubsection{Beweis}
	Wir betrachten die Menge der Tupel $(g_1, g_p) \in G^p$ mit $g_1 \cdot \ldots \cdot g_p = 1$. 
	Sei nun 
	\begin{align*}
		T &= |\{(g_1, \ldots, g_p) \mid  g_1 \cdot \ldots \cdot g_p = 1 \}| \\
		&= |\{(g_1, \ldots, g_{p-1}) \in G^{p-1} \}| \\
		&= |G^{p-1}|
	\end{align*}
	Nun soll $\mathbb Z /p\mathbb Z$ auf $T$ operieren durch
	 \begin{align*}
		T = (\underbrace{g_1, \ldots, g_m}_{\substack{u}}, \underbrace{g_{m+1}, \ldots, g_p}_{\substack{v}}) \implies m\cdot T = (\underbrace{g_{m+1}, \ldots, g_p}_{\substack{v}}, \underbrace{g_1, \ldots, g_m}_{\substack{u}})
	\end{align*}
	Es gilt also $uv = 1 \iff v(uv)v^{-1} = 1 = u = 1$, da $|G^{-1}| = n^{p-1}$ und $p \mid n^{p-1} $ da $p \mid n$.
	Weiter haben Bahnen entweder die Länge $1$ oder $p$, da $\mathbb Z /p\mathbb Z$ nur zwei Untergruppen hat.

	\begin{align*}
		\mathbb Z /p\mathbb Z \cdot (1, \ldots, 1) = (1, \ldots, 1) \\
		\{(g, \ldots, g) \mid gP= 1 \} \implies \exists g \text{ mit } gP = 1
	\end{align*}
\bewiesen

\section{Gruppentheorie 2 - Vorlesung 3}
\subsection{p-Gruppen}
\subsubsection{Definition p-Gruppe}
	$G$ heißt $p$-Gruppe falls $p$ eine Primzahl ist und $|G| \in p^{\mathbb N}$

\subsubsection{Definition Zentrum}
	Wir definieren das Zentrum einer Gruppe wie folgt $$Zentrum(G)= Z(G) = \{g \in G \mid  \forall x \in G: xg = gx\}$$

\subsubsection{Satz über das Zentrum}
	\begin{align*}
		G \text{ ist eine $p$-Gruppe} \implies Z(G) \not = \{1\}
	\end{align*}

\subsubsection{Beweis des Satzes}
	$G$ operiert auf sich selbst durch $x \mapsto gxg^{-1}$. Dann ist $G$ disjunkt zerlegt in Bahnen. Für jede Bahn gilt $|G/Fix(x)|$. 
	Also $Fix(x) \not= G \implies p \mid |G/Fix(x)|$. Dann gilt 
	\begin{align*}
		|G| &\equiv \sum_{x \in G} |G/Fix(x)|  \mod p\\
		&\equiv |Z(G) |\mod p
	\end{align*}
	Daraus folgt $|Z(G)| \geq 1$ und durch die Kongruenz zu $p$ muss $|Z(G)|\geq p$ gelten. \\
	Also $p \mid |Z(G)|$. 
	\newline
	
	Wir können dann daraus schließen dass $G/Z(G)$ eine $p$-Gruppe mit $Z(G/Z(G)) \not = \{1\}$ oder $G=Z(G)$ ist.
	\bewiesen
	
\subsection{Zyklische Gruppen}
\subsubsection{Definition Nilpotent}
	Eine Gruppe $G$ heißt nilpotent falls es eine Kette von Gruppen gibt, sodass
	\begin{align*}
		\langle 1 \rangle &= G_0, G_1, \ldots, G_M = G  \text{ und } G_K/Z(G) = G_{k-1} \text{ für } 1 \leq k \leq k
	\end{align*}

\subsubsection{Beispiele Nilpotent}
	\begin{itemize}
		\item $|D_3| = 6$ ist die kleinste nicht nilpotente Gruppe, da $Z(D_3) = \{1\}$ und somit nicht kommutativ.
		\item $|D_4| = 8$ ist nilpotent und kommutativ, da $D_4/Z(D_4) = \mathbb Z/2 \mathbb Z \times \mathbb Z /2\mathbb Z$
	\end{itemize}

\subsubsection{Definition zyklische Gruppe}
	Eine Gruppe $G$ heißt zyklisch falls $\exists g \in G$ mit $G=g^{\mathbb Z}$

\subsubsection{Satz zu zyklischen Gruppen}
	$|G| < \infty$ und $G$ zyklisch $\implies \exists g \in G$ mit $G =\{g^0, g^1, \ldots, g^{|G| -1} \}$  

\subsubsection{Beweis des Satzes}
 Betrachte $g, g^2, g^3, \ldots$ für $g^{\mathbb Z}$

	




\end{document}