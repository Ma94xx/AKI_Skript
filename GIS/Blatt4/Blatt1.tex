\documentclass[12pt, german]{article}

\usepackage[ngerman]{babel}
\usepackage[T1]{fontenc}  
\usepackage{enumitem}
\usepackage[utf8]{inputenc}
\usepackage{amsmath}
\usepackage{marvosym}
\usepackage{tasks}
\usepackage{stmaryrd}

\usepackage{amssymb}
\usepackage{pgf}
\usepackage{graphicx}
\usepackage{listings}

\usepackage{algpseudocode}% http://ctan.org/pkg/algorithmicx
\usepackage{algorithm}% http://ctan.org/pkg/algorithm

\usepackage{pgf}
\usepackage{tikz}
\usetikzlibrary{arrows,automata}

%Entfernt Seitenummerierung
\pagenumbering{gobble}

\usepackage[a4paper,lmargin={2cm},rmargin={2cm},tmargin={3.5cm},bmargin = {2.5cm},headheight = {4cm}]{geometry}
 \newcommand{\bewiesen}{\begin{flushright}$\square$ \end{flushright} } 
 \newcommand{\teilsbewiesen}{\begin{flushright}$\triangle$ \end{flushright} } 


\lstset{ 
	backgroundcolor=\color{white},   % choose the background color; you must add \usepackage{color} or \usepackage{xcolor}; should come as last argument
	%	basicstyle=\footnotesize,        % the size of the fonts that are used for the code
	%	breakatwhitespace=false,         % sets if automatic breaks should only happen at whitespace
	%	breaklines=true,                 % sets automatic line breaking
	captionpos=b,                    % sets the caption-position to bottom
	commentstyle=\color{mygreen},    % comment style
	deletekeywords={...},            % if you want to delete keywords from the given language
	escapeinside={\%*}{*)},          % if you want to add LaTeX within your code
	extendedchars=true,              % lets you use non-ASCII characters; for 8-bits encodings only, does not work with UTF-8
	frame=single,	                   % adds a frame around the code
	keepspaces=true,                 % keeps spaces in text, useful for keeping indentation of code (possibly needs columns=flexible)
	keywordstyle=\color{blue},       % keyword style
	language=Octave,                 % the language of the code
	morekeywords={*,...},            % if you want to add more keywords to the set
	numbers=left,                    % where to put the line-numbers; possible values are (none, left, right)
	numbersep=5pt,                   % how far the line-numbers are from the code
	numberstyle=\tiny\color{mygray}, % the style that is used for the line-numbers
	rulecolor=\color{black},         % if not set, the frame-color may be changed on line-breaks within not-black text (e.g. comments (green here))
	showspaces=false,                % show spaces everywhere adding particular underscores; it overrides 'showstringspaces'
	showstringspaces=false,          % underline spaces within strings only
	showtabs=false,                  % show tabs within strings adding particular underscores
	stepnumber=1,                    % the step between two line-numbers. If it's 1, each line will be numbered
	stringstyle=\color{mymauve},     % string literal style
	tabsize=2,	                   % sets default tabsize to 2 spaces
	title=\lstname                   % show the filename of files included with \lstinputlisting; also try caption instead of title
}


\title{Grundlagen Informations Sicherheit \\ Übungsblatt 04}

\author{Max Kurz (3265240)  \and Mohamed Barbouchi (3233706) \and Daniel Kurtz (3332911)}
\date{}
\pagenumbering{gobble}
\begin{document}
	\maketitle
    \subsection*{Problem 1}
    Sei $r_2 = (r_1 \mod N)^c$, dann gilt
    \begin{align*}
    E^{r_1}(x,N)^c \mod N^2 &= ((1+N)^x \cdot r_1^N \mod N^2)^c \mod N^2  \\
    &=((1+N)^x \cdot r_1^N \mod N)^c \mod N^2  \\
    &=((1+N)^{x \cdot c} \cdot (r_1^N \mod N)^c) \mod N^2  \\
    &=((1+N)^{x \cdot c \mod N} \cdot (r_1^N \mod N)^c) \mod N^2  \\ \\
    E^{r_2}(cx \mod N,N) &= ((1+N)^{x\cdot c \mod N} \cdot r_2^N) \mod N^2 \\
    &= ((1+N)^{x\cdot c \mod N} \cdot (r_1^N \mod N)^{c} \mod N^2     
    \end{align*}
    Wir sehen also dass $E^{r_1}(x,N)^c \mod N^2 = E^{r_2}(cx \mod N,N)$ gilt.
    \bewiesen
    \newpage
    \subsection*{Problem 2}
   \begin{enumerate}[label=\arabic*.]
   	\item $\phi(35) = (7-1)(5-1) = 6\cdot 4 = 24 $
   	\item  Mit $\phi^{-1}(35) = 19$ berrechnen wir nun $D(1031, 24)$ und $D(776, 24)$
   	\begin{align*}	
   	D(1031, 24) &= \left(\frac{(1031^{24} \mod 35^2) -1}{35} \cdot 19\right) \mod 35 \\
   				&=\left(\frac{596 -1}{35} \cdot 19\right) \mod 35 \\
   				&=(17 \cdot 19) \mod 35 \\
   				&= 24
   	\end{align*}
   	\begin{align*}	
   	D(776, 24) &= \left(\frac{(776^{24} \mod 35^2) -1}{35} \cdot 19\right) \mod 35 \\
   	&=\left(\frac{1051 -1}{35} \cdot 19\right) \mod 35 \\
   	&=(30 \cdot 19) \mod 35 \\
   	&= 10
   	\end{align*}
   	Wir erhalten somit $x_3 = 10 + 8 = 18$
   	\item $y_3 = (1031 \cdot 776) \mod N^2= 131$
   	Berechne $D(131, 24)$
   	\begin{align*}
   		D(131, 24) &= \left(\frac{(131^{24} \mod 35^2) -1}{35} \cdot 19\right) \mod 35 \\
   		&=\left(\frac{421 -1}{35} \cdot 19\right) \mod 35 \\
   		&=(12 \cdot 19) \mod 35 \\
   		&= 18
   	\end{align*}
   	Es gilt also $x_3 == x_3'$
   \end{enumerate}
    \subsection*{Problem 3}
    \begin{enumerate}[label=\alph*)]
 \item Nope. Man wähle $x_1 = 100$ und $x_2 = 010$ dann gilt $h(100) = h(010) = 2$
  \item Nope. Für beliebigen Input $x$, lässt sich $x_1$ und $x_2$ vertauschen um eine Kollision zu erzeugen, da $x_1 \oplus x_2 = x_2 \oplus x_1$ gilt.
\end{enumerate}
    \subsection*{Problem 4}
    Sei $h'(x) = h'(y)$ dann gilt $x(0)h(x) = y(0)h(y)$ \\
    Da $|x(0)| = |y(0)|$ gilt, folgt somit dass $h(x) = h(y)$ gilt. 
 \bewiesen
 \newpage
    \subsection*{Problem 5}
    \begin{algorithm} 
    	\centering
    	\caption{ }
    	\label{Alg:1}
    	\begin{algorithmic}[1]
    		\State Let $x$, and $x'$ be random
    		\State \texttt{send}($x$)
    		\State \texttt{receive}($t$)
    		\State \texttt{send}($x'$)
    		\State \texttt{receive}($t'$)
    		\State $x'' = x \| (x_1' \oplus t)\|x_2'\|\ldots\|x'_{n-1}$
    		\State \texttt{return } $x'', t$
    	\end{algorithmic}
    \end{algorithm}

   	
    

\end{document}