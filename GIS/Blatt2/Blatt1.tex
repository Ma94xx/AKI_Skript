\documentclass[12pt, german]{article}

\usepackage[ngerman]{babel}
\usepackage[T1]{fontenc}  
\usepackage{enumitem}
\usepackage[utf8]{inputenc}
\usepackage{amsmath}
\usepackage{marvosym}
\usepackage{tasks}
\usepackage{stmaryrd}

\usepackage{amssymb}
\usepackage{pgf}
\usepackage{graphicx}
\usepackage{listings}

\usepackage{algpseudocode}% http://ctan.org/pkg/algorithmicx
\usepackage{algorithm}% http://ctan.org/pkg/algorithm

\usepackage{pgf}
\usepackage{tikz}
\usetikzlibrary{arrows,automata}

%Entfernt Seitenummerierung
\pagenumbering{gobble}

\usepackage[a4paper,lmargin={2cm},rmargin={2cm},tmargin={3.5cm},bmargin = {2.5cm},headheight = {4cm}]{geometry}
 \newcommand{\bewiesen}{\begin{flushright}$\square$ \end{flushright} } 
 \newcommand{\teilsbewiesen}{\begin{flushright}$\triangle$ \end{flushright} } 


\lstset{ 
	backgroundcolor=\color{white},   % choose the background color; you must add \usepackage{color} or \usepackage{xcolor}; should come as last argument
	%	basicstyle=\footnotesize,        % the size of the fonts that are used for the code
	%	breakatwhitespace=false,         % sets if automatic breaks should only happen at whitespace
	%	breaklines=true,                 % sets automatic line breaking
	captionpos=b,                    % sets the caption-position to bottom
	commentstyle=\color{mygreen},    % comment style
	deletekeywords={...},            % if you want to delete keywords from the given language
	escapeinside={\%*}{*)},          % if you want to add LaTeX within your code
	extendedchars=true,              % lets you use non-ASCII characters; for 8-bits encodings only, does not work with UTF-8
	frame=single,	                   % adds a frame around the code
	keepspaces=true,                 % keeps spaces in text, useful for keeping indentation of code (possibly needs columns=flexible)
	keywordstyle=\color{blue},       % keyword style
	language=Octave,                 % the language of the code
	morekeywords={*,...},            % if you want to add more keywords to the set
	numbers=left,                    % where to put the line-numbers; possible values are (none, left, right)
	numbersep=5pt,                   % how far the line-numbers are from the code
	numberstyle=\tiny\color{mygray}, % the style that is used for the line-numbers
	rulecolor=\color{black},         % if not set, the frame-color may be changed on line-breaks within not-black text (e.g. comments (green here))
	showspaces=false,                % show spaces everywhere adding particular underscores; it overrides 'showstringspaces'
	showstringspaces=false,          % underline spaces within strings only
	showtabs=false,                  % show tabs within strings adding particular underscores
	stepnumber=1,                    % the step between two line-numbers. If it's 1, each line will be numbered
	stringstyle=\color{mymauve},     % string literal style
	tabsize=2,	                   % sets default tabsize to 2 spaces
	title=\lstname                   % show the filename of files included with \lstinputlisting; also try caption instead of title
}


\title{Grundlagen Informations Sicherheit \\ Übungsblatt 02}

\author{Max Kurz (3265240)  \and Mohamed Barbouchi (3233706) \and Daniel Kurtz (123456)}
\date{}
\pagenumbering{gobble}
\begin{document}
	\maketitle
    \subsection*{Problem 1}
   	\begin{algorithm} 
    		\centering
    		\caption{$D(Y)$}
    		\label{Alg:1}
    		\begin{algorithmic}[1]
    			\State \texttt{send($x_1$)}  
    			\State \texttt{$y_1$ = receive($y_1$)}
    			\State \texttt{send($x_2$)}  
    			\State \texttt{$y_2$ = receive($y_2$)}
    			\State  $a= (y_1 -_n y_2)/(x_1 -_n x_2)$ 
    			\State $b=y_2 -_n ax_2$
    			\State $x_{d1}= D(y_1, (a,b))$
    			\State $x_{d2}= D(y_2, (a,b))$
    			\State  \texttt{if($x_{d1}$ == $x_1$)}{	$b = 1$ \texttt{else} $b = 0$ }
    			\State \texttt{return } $b$
    		\end{algorithmic}
    	\end{algorithm} 		
    Die Wahrscheinlichkeit für den Angreifer das richtige $b$ zurrückzugeben und liegt bei $1$, da mit dem oben beschrieben Algorithmus das Tupel $(a,b)$ rekonstruiert werden kann.
    
    %TODO anpassen
    \subsection*{Problem 2}
    
    \subsection*{Problem 3}
    	Sei  $x, y \in \mathbb{Z}_n$ und $l,k \in \mathbb{Z}$
    	\begin{align*}
    			a &\equiv x \mod n 	\implies	a = ln +_n x \\
    			b &\equiv y \mod n \implies 	b = kn +_n y
    	\end{align*}
    	\begin{align*}
    	a\cdot b &= (ln +x)(kn +y) \\
    	&=lkn^2 +lny +kxn + xy \\
    	&= lkn^2 +n(ly +kx) +xy \equiv x\cdot y \mod n
    	\end{align*}
    	\bewiesen
    
    
       \subsection*{Problem 4}
    	Assoziativität ist gegeben\footnote{$\forall a,b,c \in R^\ast : a,b,c \in R $} da $(R, \cdot)$ bereits assozitiv ist. Wir zeigen Abgeschlossenheit, die Existenz von Inversen, und neutralem Element. 
    	\subsubsection*{Abgeschlossenheit}
    	 $\forall  x,y \in R^\ast \implies \exists x^{-1}, y^{-1} \in R^\ast : xx^{-1} = 1_R = x^{-1}x, \; yy^{-1} = 1_R = y^{-1}y $
    	\begin{align*}
    		x \cdot y = z &\implies xyy^{-1} = x1_R=x=zy^{-1} \\
    		&\implies xx^{-1} = 1 = zy^{-1}x^{-1} \\
    		&\implies \exists z^{-1} \in R^\ast : y^{-1}x^{-1} = z^{-1} : zz^{-1} = 1_R  \quad \text{ Assoziativität }\\
    			&\implies z \in R^\ast
    	\end{align*} \teilsbewiesen
    	\subsubsection*{Inverse}
    	Sei $x \in R$ 
    		\begin{align*} 
    		 \forall  x \in R \; \exists x^{-1} \in R^\ast : x\cdot x^{-1} = 1_R  \\
    		 \implies \forall x^{-1} \in R^\ast \; \exists x \in R : x^{-1} \cdot x = 1_R
    	\end{align*} \teilsbewiesen
    	
    	\subsubsection*{Neutrales}
    		Sei $1_R \in R$ 
    		\begin{align*}
    			1_R \cdot 1_R = 1_R \implies 1_R \in R^\ast
    	\end{align*} \bewiesen
   
    \subsection*{Problem 5}
    Zu zeigen: 	$a | b \implies (x \mod b) \mod a = x \mod a $. Sei $z =(x \mod b)$ 
    \begin{align*}
    	a | b \implies b \mod a = 0 \\
    	(x \mod b) \mod a &= x \mod a  \\
    	(x \mod b) &\equiv x \mod a \\ 
		z &\equiv x \mod a     
    \end{align*}
    Da $(x \mod b) = z$  gilt, und wir $b \mod a = 0$ vorrausetzen können, wissen wir dass für $z = lb + x$ mit $l \in \mathbb{Z}$ gilt. 
    \begin{align*}
    lb + x &\equiv x \mod a     \\
     x &= x
    \end{align*} \bewiesen
    \newpage
    %TODO anpassen
    \subsection*{Problem 6}
    		
    
    
    

\end{document}