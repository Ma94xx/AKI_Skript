\documentclass[12pt, german]{article}

\usepackage[ngerman]{babel}
\usepackage[T1]{fontenc}  
\usepackage{enumitem}
\usepackage[utf8]{inputenc}
\usepackage{amsmath}
\usepackage{marvosym}
\usepackage{tasks}
\usepackage{stmaryrd}

\usepackage{amssymb}
\usepackage{pgf}
\usepackage{graphicx}
\usepackage{listings}

\usepackage{algpseudocode}% http://ctan.org/pkg/algorithmicx
\usepackage{algorithm}% http://ctan.org/pkg/algorithm

\usepackage{pgf}
\usepackage{tikz}
\usetikzlibrary{arrows,automata}

%Entfernt Seitenummerierung
\pagenumbering{gobble}

\usepackage[a4paper,lmargin={2cm},rmargin={2cm},tmargin={3.5cm},bmargin = {2.5cm},headheight = {4cm}]{geometry}



\lstset{ 
	backgroundcolor=\color{white},   % choose the background color; you must add \usepackage{color} or \usepackage{xcolor}; should come as last argument
	%	basicstyle=\footnotesize,        % the size of the fonts that are used for the code
	%	breakatwhitespace=false,         % sets if automatic breaks should only happen at whitespace
	%	breaklines=true,                 % sets automatic line breaking
	captionpos=b,                    % sets the caption-position to bottom
	commentstyle=\color{mygreen},    % comment style
	deletekeywords={...},            % if you want to delete keywords from the given language
	escapeinside={\%*}{*)},          % if you want to add LaTeX within your code
	extendedchars=true,              % lets you use non-ASCII characters; for 8-bits encodings only, does not work with UTF-8
	frame=single,	                   % adds a frame around the code
	keepspaces=true,                 % keeps spaces in text, useful for keeping indentation of code (possibly needs columns=flexible)
	keywordstyle=\color{blue},       % keyword style
	language=Octave,                 % the language of the code
	morekeywords={*,...},            % if you want to add more keywords to the set
	numbers=left,                    % where to put the line-numbers; possible values are (none, left, right)
	numbersep=5pt,                   % how far the line-numbers are from the code
	numberstyle=\tiny\color{mygray}, % the style that is used for the line-numbers
	rulecolor=\color{black},         % if not set, the frame-color may be changed on line-breaks within not-black text (e.g. comments (green here))
	showspaces=false,                % show spaces everywhere adding particular underscores; it overrides 'showstringspaces'
	showstringspaces=false,          % underline spaces within strings only
	showtabs=false,                  % show tabs within strings adding particular underscores
	stepnumber=1,                    % the step between two line-numbers. If it's 1, each line will be numbered
	stringstyle=\color{mymauve},     % string literal style
	tabsize=2,	                   % sets default tabsize to 2 spaces
	title=\lstname                   % show the filename of files included with \lstinputlisting; also try caption instead of title
}


\title{Grundlagen Informations Sicherheit \\ Übungsblatt 01}

\author{Max Kurz (3265240)  \and Mohamed Barbouchi (3233706) \and Daniel Kurtz {123456}}
\date{}
\pagenumbering{gobble}
\begin{document}
	\maketitle

    \subsection*{Problem 1}
    \begin{enumerate}[label=\alph*)]
    	\item  ~\par
    		\begin{table}[h!]
    		\centering
    		\begin{tabular}{c | c | c | c }
    			$D()$ & A & B & C \\ \hline
    		 	$k_1$ & a & b & c \\ \hline
    		 	$k_2$ & c & a & b \\ \hline
    		 	$k_3$ & b & c & a 
    		\end{tabular}
    	\end{table}
    		
    	\item  ~\par
    	\begin{table}[h!]
    		\centering
    		\begin{tabular}{c | c | c | c }
    			$D()$ & A & B & C \\ \hline
    			$k_1$ & a & b & c \\ \hline
    			$k_2$ & b & a & c \\ \hline
    			$k_3$ & a & b & c 
    		\end{tabular}
    	\end{table}
    	
    	\item Klappt nicht, da $\forall x \in X, k \in K: D(E(x, k), k) = x$ gelten muss und wir hier aber $D(A, k_1) = a$ und  $D(A, k_1) = c$ bekommen. Die Entschlüsslelung ist also nicht eindeutig.
    	
    	\subsection*{Problem 2}
    	Das Wort lautet: Hitchhiker
    	Man wandelt die Zahlen in Binär um, und xort die beiden Strings. Das Ergebnis wird wieder in Dezimal umgewandelt. 
    	
    	\newpage
    	\subsection*{Problem 3}
    	\begin{enumerate}[label=\alph*)]
    	
    	\item	Sei $x \in \{0, 1 \}^l$, dann gilt $x^l \oplus x^l \oplus x^l = x^l \oplus 0^l $ wobei $0^l$ hier die Identität ist. \\ Für die Identität gilt: $x \oplus 1_m = x$.
    
    	\begin{algorithm} 
    		\centering
    		\caption{$D(Y)$}
    		\label{Alg:1}
    		\begin{algorithmic}[1]
    			\State $z_{1}.\texttt{concat}(z_{2}) := Y$ mit $|z_1| = |z_2|$
    			\State $l := |z_2|$
    			\State $r := z_2 \oplus 1^l$
    			\State \texttt{return } $r$
    		\end{algorithmic}
    	\end{algorithm} 		
    		
    		\item
		\begin{algorithm} 
				\centering
				\caption{Attack}
				\label{Alg:2}
			\begin{algorithmic}[1]
					\State Let $x \in \{0^l\}$ and $y \in \{1^l\}$
					\State \texttt{send}($x$)
					\State \texttt{send}($y$)
					\State \texttt{receive}($w$)
					\State \texttt{if}($w == 0^l$) $b=1$ \texttt{else}  $b=0$
					\State \texttt{return } $b$
			\end{algorithmic}
		\end{algorithm} 		
		Die Wahrscheinlichkeit für den Angreifer so die richtige Nachricht zu erkennen und das passende $b$ zurückzugeben liegt bei $1$. Siehe Aufgabe a). 

    	\end{enumerate}
    
    
    
    
    
    
    
    
    \end{enumerate}
\end{document}