\documentclass[12pt, german]{article}

\usepackage[ngerman]{babel}
\usepackage[T1]{fontenc}  
\usepackage{enumitem}
\usepackage[utf8]{inputenc}
\usepackage{amsmath}
\usepackage{marvosym}
\usepackage{tasks}
\usepackage{stmaryrd}

\usepackage{amssymb}
\usepackage{pgf}
\usepackage{graphicx}
\usepackage{listings}

\usepackage{algpseudocode}% http://ctan.org/pkg/algorithmicx
\usepackage{algorithm}% http://ctan.org/pkg/algorithm

\usepackage{pgf}
\usepackage{tikz}
\usetikzlibrary{arrows,automata}

%Entfernt Seitenummerierung
\pagenumbering{gobble}

\usepackage[a4paper,lmargin={2cm},rmargin={2cm},tmargin={3.5cm},bmargin = {2.5cm},headheight = {4cm}]{geometry}
 \newcommand{\bewiesen}{\begin{flushright}$\square$ \end{flushright} } 
 \newcommand{\teilsbewiesen}{\begin{flushright}$\triangle$ \end{flushright} } 


\lstset{ 
	backgroundcolor=\color{white},   % choose the background color; you must add \usepackage{color} or \usepackage{xcolor}; should come as last argument
	%	basicstyle=\footnotesize,        % the size of the fonts that are used for the code
	%	breakatwhitespace=false,         % sets if automatic breaks should only happen at whitespace
	%	breaklines=true,                 % sets automatic line breaking
	captionpos=b,                    % sets the caption-position to bottom
	commentstyle=\color{mygreen},    % comment style
	deletekeywords={...},            % if you want to delete keywords from the given language
	escapeinside={\%*}{*)},          % if you want to add LaTeX within your code
	extendedchars=true,              % lets you use non-ASCII characters; for 8-bits encodings only, does not work with UTF-8
	frame=single,	                   % adds a frame around the code
	keepspaces=true,                 % keeps spaces in text, useful for keeping indentation of code (possibly needs columns=flexible)
	keywordstyle=\color{blue},       % keyword style
	language=Octave,                 % the language of the code
	morekeywords={*,...},            % if you want to add more keywords to the set
	numbers=left,                    % where to put the line-numbers; possible values are (none, left, right)
	numbersep=5pt,                   % how far the line-numbers are from the code
	numberstyle=\tiny\color{mygray}, % the style that is used for the line-numbers
	rulecolor=\color{black},         % if not set, the frame-color may be changed on line-breaks within not-black text (e.g. comments (green here))
	showspaces=false,                % show spaces everywhere adding particular underscores; it overrides 'showstringspaces'
	showstringspaces=false,          % underline spaces within strings only
	showtabs=false,                  % show tabs within strings adding particular underscores
	stepnumber=1,                    % the step between two line-numbers. If it's 1, each line will be numbered
	stringstyle=\color{mymauve},     % string literal style
	tabsize=2,	                   % sets default tabsize to 2 spaces
	title=\lstname                   % show the filename of files included with \lstinputlisting; also try caption instead of title
}


\title{Grundlagen Informations Sicherheit \\ Übungsblatt 04}

\author{Max Kurz (3265240)  \and Mohamed Barbouchi (3233706) \and Daniel Kurtz (3332911)}
\date{}
\pagenumbering{gobble}
\begin{document}
	\maketitle
    \subsection*{Problem 1}
    \begin{align*}
    f_1 \cdot_{\mathbb{F}_{2^{8}}} f_2 = f_1 \cdot_{\mathbb{Z}_{2}[x]} f_2 \mod g &= \\
    x^7 + x^5 + x^4 +x^2 + x \cdot_{\mathbb{Z}_{2}[x]} x^6 + x^4 +x + 1 \mod g &= \\
    x^{13}+x^{10} +x^9 + x^8 + x^5 +x^4 + x^3 +x \mod g &=
    \end{align*}
    $= x^5 + x^4 +x^2 +x$
    \subsection*{Problem 2}
    Seien $f$ und $p$ beliebige Polynome. Dann gilt:
    \begin{align*}
    	\deg(f \cdot p) &= \deg(\sum a_ix^i \cdot \sum a_j x^i) \\
    	&= \deg(\max\{x^i\} \cdot \max\{x^j\}) \quad \forall a_i, a_j \neq 0 \\
    	&= \deg(x_{\text{max}}^{i} \cdot x_{\text{max}}^{j})  \\
    	&= \deg(\underbrace{(x_{\text{max}} \cdot \ldots \cdot x_{\text{max}})}_\text{i-mal}\cdot\underbrace{(x_{\text{max}} \cdot \ldots \cdot x_{\text{max}})}_\text{j-mal}) 
    	=\deg(\underbrace{x_{\text{max}}\cdot \ldots \cdot x_{\text{max}}}_\text{i+j -mal})  \\
    	&=\deg(x^{i+j}) \\
    	&= \deg(f) + \deg(g)
    \end{align*}
    \bewiesen
    \newpage
     \subsection*{Problem 3}
     Sei $x$ und $x'$ unterschiedlich aber es gilt $h_n(x) = h_n(x')$. 
      \begin{algorithm} 
     	\centering
     	\caption{ }
     	\label{Alg:1}
     	\begin{algorithmic}[1]
     		\State Generate Keypair $((n,e),(n,d))$
     		\State \texttt{send}($x$)
     		\State \texttt{receive}($s$) und  $s$ = \texttt{PKCS-sig($x, (n,d)$)} = $h_n(x)^d\mod n$
     		\State \texttt{output}($x', s$)
     	\end{algorithmic}
       \end{algorithm}
   	
   	\noindent
   	Der Angreifer hat einen Vorteil von $1$. \\
   	$V'(x', s,(n,e)) = V(h_n(x'), s, (n,e)) = $ \texttt{ vaild}, da: 
   	\begin{align*}
   		h_n(x') &= h_n(x) \text{ und }\\
   		s^e &= h_n(x)^{d^{e}} \mod n = h_n(x)
   	\end{align*}
   	Das bedeutet also dass der Angreifer ohne das Orakel mit dieser Message zubefragen, einen \texttt{valid}-Tag gefunden hat und somit immer das Game gewinnt.
    
     \subsection*{Problem 4}
      \subsection*{Problem 5}
   
  

   	
    

\end{document}