\documentclass[12pt, german]{article}

\usepackage[ngerman]{babel}
\usepackage[T1]{fontenc}  
\usepackage{enumitem}
\usepackage[utf8]{inputenc}
\usepackage{amsmath}
\usepackage{marvosym}
\usepackage{tasks}
\usepackage{stmaryrd}

\usepackage{amssymb}
\usepackage{pgf}
\usepackage{graphicx}
\usepackage{listings}

\usepackage{algpseudocode}% http://ctan.org/pkg/algorithmicx
\usepackage{algorithm}% http://ctan.org/pkg/algorithm

\usepackage{pgf}
\usepackage{tikz}
\usepackage{polynom}
\usetikzlibrary{arrows,automata}

%Entfernt Seitenummerierung
\pagenumbering{gobble}

\usepackage[a4paper,lmargin={2cm},rmargin={2cm},tmargin={3.5cm},bmargin = {2.5cm},headheight = {4cm}]{geometry}
 \newcommand{\bewiesen}{\begin{flushright}$\square$ \end{flushright} } 
 \newcommand\scalemath[2]{\scalebox{#1}{\mbox{\ensuremath{\displaystyle #2}}}}
 \newcommand{\teilsbewiesen}{\begin{flushright}$\triangle$ \end{flushright} } 


\lstset{ 
	backgroundcolor=\color{white},   % choose the background color; you must add \usepackage{color} or \usepackage{xcolor}; should come as last argument
	%	basicstyle=\footnotesize,        % the size of the fonts that are used for the code
	%	breakatwhitespace=false,         % sets if automatic breaks should only happen at whitespace
	%	breaklines=true,                 % sets automatic line breaking
	captionpos=b,                    % sets the caption-position to bottom
	commentstyle=\color{mygreen},    % comment style
	deletekeywords={...},            % if you want to delete keywords from the given language
	escapeinside={\%*}{*)},          % if you want to add LaTeX within your code
	extendedchars=true,              % lets you use non-ASCII characters; for 8-bits encodings only, does not work with UTF-8
	frame=single,	                   % adds a frame around the code
	keepspaces=true,                 % keeps spaces in text, useful for keeping indentation of code (possibly needs columns=flexible)
	keywordstyle=\color{blue},       % keyword style
	language=Octave,                 % the language of the code
	morekeywords={*,...},            % if you want to add more keywords to the set
	numbers=left,                    % where to put the line-numbers; possible values are (none, left, right)
	numbersep=5pt,                   % how far the line-numbers are from the code
	numberstyle=\tiny\color{mygray}, % the style that is used for the line-numbers
	rulecolor=\color{black},         % if not set, the frame-color may be changed on line-breaks within not-black text (e.g. comments (green here))
	showspaces=false,                % show spaces everywhere adding particular underscores; it overrides 'showstringspaces'
	showstringspaces=false,          % underline spaces within strings only
	showtabs=false,                  % show tabs within strings adding particular underscores
	stepnumber=1,                    % the step between two line-numbers. If it's 1, each line will be numbered
	stringstyle=\color{mymauve},     % string literal style
	tabsize=2,	                   % sets default tabsize to 2 spaces
	title=\lstname                   % show the filename of files included with \lstinputlisting; also try caption instead of title
}


\title{Grundlagen Informations Sicherheit \\ Übungsblatt 05}

\author{Max Kurz (3265240)  \and Mohamed Barbouchi (3233706) \and Daniel Kurtz (3332911)}
\date{}
\pagenumbering{gobble}
\begin{document}
	\maketitle
    \subsection*{Problem 1}
    
    \begin{align*}
    f_1 \cdot_{\mathbb{F}_{2^{8}}} f_2 = f_1 \cdot_{\mathbb{Z}_{2}[x]} f_2 \mod g &= \\
    x^7 + x^5 + x^4 +x^2 + x \cdot_{\mathbb{Z}_{2}[x]} x^6 + x^4 +x + 1 \mod g &= \\
    x^{13}+x^{10} +x^9 + x^8 + x^5 +x^4 + x^3 +x \mod g &=
    \end{align*}
   Mit der Polynomdivision bildet der entstandene Rest das Ergebnis übertragen nach $\mathbb{Z}_2$: \\
   $$x^5 +x^4 +x^2+x$$. 
   
 $\scalemath{0.8}{\polylongdiv[style=C]{x^{13}+x^{10} +x^9 + x^8 + x^5 +x^4 + x^3 +x}{x^8+x^4+x^3+x+1}}$


 \subsection*{Problem 2}
    \begin{enumerate}[label=\alph*)]
    	
   
    \item 
    Seien $f$ und $p$ beliebige Polynome. Dann gilt:
    \begin{align*}
    	\deg(f \cdot p) &= \deg(\sum a_ix^i \cdot \sum a_j x^i) \\
    	&= \deg(\max\{x^i\} \cdot \max\{x^j\}) \quad \forall a_i, a_j \neq 0 \\
    	&= \deg(x_{\text{max}}^{i} \cdot x_{\text{max}}^{j})  \\
    	&= \deg(\underbrace{(x_{\text{max}} \cdot \ldots \cdot x_{\text{max}})}_\text{i-mal}\cdot\underbrace{(x_{\text{max}} \cdot \ldots \cdot x_{\text{max}})}_\text{j-mal}) 
    	=\deg(\underbrace{x_{\text{max}}\cdot \ldots \cdot x_{\text{max}}}_\text{i+j -mal})  \\
    	&=\deg(x^{i+j}) = \deg(x^i) + \deg(x^j) \\
    	&= \deg(f) + \deg(g)
    \end{align*}
    \bewiesen
   \newpage
   \item ~\par
  
  	\subsubsection*{Abgeschlossenheit unter Addition}
     Es gilt $\forall i \, a_i+b_i \in F $ und damit $(a_0+b_0,\ldots) \in F[x]$, da  F ein Körper ist und unter Addition abgeschlossen ist.
    $$f +_{F[x]} g = (a_0, a_1,\ldots) +_{F[x]} (b_0, b_1,\ldots) = (a_0+b_0, a_1+b_1,\ldots)$$
  
	\subsubsection*{Kommutativität der Addition} 
    \begin{align*} 
    f +_{F[x]} g = (a_0,a_1,\ldots) +_{F[x]} (b_0,b_1,\ldots)  &= (a_0+b_0,a_1+b_1,\ldots) = (b_0+a_0,b_1+a_1,\ldots) \\
    &= (b_0,b_1,\ldots) +_{F[x]} (a_0,a_1,\ldots) = g +_{F[x]} f
    \end{align*}
    
    \subsubsection*{Assoziativität der Addition}
    \begin{align*}
    (f +_{F[x]} g)+_{F[x]} h &=(a_0+b_0,a_1+b_1,\ldots)+(c_0,c_1,\ldots) \\
    &=((a_0+b_0)+c_0,(a_1+b_1)+c_1,\ldots) \\
    &=(a_0+(b_0+c_0),a_1+(b_1+c_1),\ldots)=f +_{F[x]}(g +_{F[x]})
    \end{align*}
    
    \subsubsection*{Neutrales Element der Addition}
    Sei $e_+=(0,0,\ldots,0)$, wobei 0 das neutrale Element von F ist. Dann gilt: \\
    \begin{align*}
    	f +_{F[x]} e_+ &= (a_0+0,a_1+0,\ldots)=(a_0,a_1,\ldots)=f\\
    	e_+ +_{F[x]} f  &= (0+a_0,0+a_1,\ldots)=(a_0,a_1,\ldots)=f
    \end{align*}
   	
   	\subsubsection*{Nichtleerheit}
    Es gilt $e_+ \in F[x]$, damit folgt $F[x]\neq\emptyset$.
	
	\subsubsection*{Existenz des inversen Elemente}
    Sei $a_i^{-1}$ das inverse Element zu $a_i$ in F für alle $a_i \in F$ und $f^{-1}=(a_0^{-1},a_1^{-1},\ldots)$.\\
    $$f +_{F[x]} f^{-1} = (a_0+a_0^{-1},a_1+a_1^{-1},\ldots)=(0,0,0,\ldots)=(a_0^{-1}+a_0,\ldots)=f^{-1} +_{F[x]} f$$
  	
  	\subsubsection*{Abgeschlossenheit unter Multiplikation}
  	Da F ein Körper ist, ist F abgeschlossen unter Multiplikation und Addition. Damit ist $(d_0,d_1,\ldots)\in F[x]$ und 
    $f \cdot_{F[x]} g = (d_0,d_1,\ldots)$ mit $d_i=\sum_{j\leq i}{a_j\cdot b_{i-j}}$ und $d_i \in F$
   	
   	\subsubsection*{Kommutativität der Multiplikation}
    \begin{align*}
    	f \cdot_{F[x]} g &= (\sum_{j\leq 0}{a_j\cdot b_{0-j}},\sum_{j\leq 1}{a_j\cdot b_{1-j}},\ldots) \\
    	&= (\sum_{j\leq 0}{b_j\cdot a_{0-j}},\sum_{j\leq 1}{b_j\cdot a_{1-j}},\ldots) \\
    	&=g \cdot_{F[x]} f
    	 \end{align*}
    
   	\subsubsection*{Distributivgesetze}
    \begin{align*}
  (f +_{F[x]} g) \cdot_{F[x]} h = (a_0+b_0,a_1+b_1,\ldots) \cdot_{F[x]} (c_0,c_1,\ldots) 
  &= (\sum_{j\leq 0}{(a_0+b_0)\cdot c_{0-j}},\ldots) \\
  &=(\sum_{j\leq 0}{a_0 \cdot c_{0-j} +b_0\cdot c_{0-j}},\ldots) \\ 
  &= (\sum_{j\leq 0}{a_0 \cdot c_{0-j}} +\sum_{j\leq 0}{b_0\cdot c_{0-j}},\ldots) \\ 
  &= (f \cdot_{F[x]} h +_{F[x]}(g \cdot_{F[x]} h)
    \\
    \\
    	f \cdot_{F[x]} (g +_{F[x]}h) = f \cdot_{F[x]}(b_0+c_0,b_1+c_1,\ldots)
    	&=(\sum_{j\leq 0}{a_0 \cdot (b_{0-j}+c_{0-j}}),\ldots) \\
    	&=(\sum_{j\leq 0}{a_0 \cdot b_{0-j}+a_0 \cdot c_{0-j}}, \ldots) \\
    	&= (\sum_{j\leq 0}{a_0 \cdot b_{0-j}+ \sum_{j\leq 0}a_0 \cdot c_{0-j}},\ldots) \\
    	&=(f \cdot_{F[x]} g)+_{F[x]}(f \cdot_{F[x]}h)
    \end{align*}
   \bewiesen

  \end{enumerate}
    \newpage
     \subsection*{Problem 3}
     Sei $x$ und $x'$ unterschiedlich aber es gilt $h_n(x) = h_n(x')$. 
      \begin{algorithm} 
     	\centering
     	\caption{ }
     	\label{Alg:1}
     	\begin{algorithmic}[1]
     		\State Generate Keypair $((n,e),(n,d))$
     		\State \texttt{send}($x$)
     		\State \texttt{receive}($s$) und  $s$ = \texttt{PKCS-sig($x, (n,d)$)} = $h_n(x)^d\mod n$
     		\State \texttt{output}($x', s$)
     	\end{algorithmic}
       \end{algorithm}
   	
   	\noindent
   	Der Angreifer hat einen Vorteil von $1$. \\
   	$V'(x', s,(n,e)) = V(h_n(x'), s, (n,e)) = $ \texttt{ vaild}, da: 
   	\begin{align*}
   		h_n(x') &= h_n(x) \text{ und }\\
   		s^e &= h_n(x)^{d^{e}} \mod n = h_n(x)
   	\end{align*}
   	Das bedeutet also dass der Angreifer ohne das Orakel mit dieser Message zubefragen, einen \texttt{valid}-Tag gefunden hat und somit immer das Game gewinnt.
    \newpage
     \subsection*{Problem 4}
     \begin{enumerate}[label=\alph*)]
     	\item ~\par
     			\begin{enumerate}[label=\arabic*.]
     		\item Data link (Source MAC = aa:aa:aa:aa:aa:aa, Dest. MAC = ff:ff:ff:ff:ff:ff) / \\ ARP (Who has 10.0.0.1 ?)
     		\item Data link (Source MAC = 01:01:01:01:01:01, Dest. MAC = aa:aa:aa:aa:aa:aa) /\\ ARP (I have 10.0.0.1)
     		\item Data link (Source MAC = aa:aa:aa:aa:aa:aa, Dest. MAC = 01:01:01:01:01:01) /\\  IP (Source IP = 10.0.0.2, Dest. IP = 10.0.0.1) /\\ UDP (Source Port = 16000 , Dest. Port = 53) /\\ 	DNS (Transaction ID = 0x0000, query = "www.example.com, type A, class IN")
     		\item IP (Source IP = 19.19.19.19, Dest. IP = 4.0.0.1) /\\ UDP (Source Port = 16001 , Dest. Port = 53) /\\ 	DNS (Transaction ID = 0x0001, query = "www.example.com, type A, class IN" )
     		\item IP (Source IP = 4.0.0.1, Dest. IP = 19.19.19.19) /\\ UDP (Source Port = 53 , Dest. Port = 16001) /\\ 	DNS (answer = "dont know, ask 4.0.0.2", Transaction ID = 0x0001, response for ="www.example.com, type A, class IN")
     		\item IP (Source IP = 19.19.19.19, Dest. IP = 4.0.0.2) /\\ UDP (Source Port = 16002 , Dest. Port = 53) /\\ 	DNS (Transaction ID = 0x0002, query = "www.example.com, type A, class IN" )
     		\item IP (Source IP = 4.0.0.2, Dest. IP = 19.19.19.19) /\\ UDP (Source Port = 53 , Dest. Port = 16002) /\\ 	DNS (answer = "dont know, ask 4.0.0.3", Transaction ID = 0x0002,  response for ="www.example.com, type A, class IN")
     		\item IP (Source IP = 19.19.19.19, Dest. IP = 4.0.0.3) /\\ UDP (Source Port = 16003 , Dest. Port = 53) /\\ 	DNS (Transaction ID = 0x0003, query = "www.example.com, type A, class IN" )
     		\item IP (Source IP = 4.0.0.3, Dest. IP = 19.19.19.19) /\\ UDP (Source Port = 53 , Dest. Port = 16003) /\\ 	DNS (answer = "www.example.com, type A, class IN, 1.2.3.4", Transaction ID = 0x0003, response for ="www.example.com, type A, class IN" )
     		\item Data link (Source MAC = 01:01:01:01:01:01, Dest. MAC = aa:aa:aa:aa:aa:aa) /\\ IP (Source IP = 10.0.0.1 , Dest. IP = 10.0.0.2) /\\ UDP (Source Port = 16000 , Dest. Port = 53) /\\ 	DNS (answer = "www.example.com, type A, class IN, 1.2.3.4", Transaction ID = 0x0000, response for ="www.example.com, type A, class IN")
     		%tcp verbindungsaufbau
     		\item Data link (Source MAC = aa:aa:aa:aa:aa:aa, Dest. MAC = 01:01:01:01:01:01) /\\  IP (Source IP = 10.0.0.2, Dest. IP = 1.2.3.4) /\\ TCP (SYN, Source Port = 17000 , Dest. Port = 80, Seq = 1)
     		\item IP (Source IP = 19.19.19.19, Dest. IP = 1.2.3.4) /\\ TCP (SYN, Source Port = 17001 , Dest. Port = 80, Seq = 1 ) 
     		\item IP (Source IP = 1.2.3.4, Dest. IP = 19.19.19.19) /\\ TCP (SYN-ACK, Source Port = 80 , Dest. Port = 17001, Seq = 10, ACK = 2 ) 
     		\item Data link (Source MAC = 01:01:01:01:01:01, Dest. MAC =  aa:aa:aa:aa:aa:aa) /\\  IP (Source IP = 1.2.3.4, Dest. IP = 10.0.0.2) /\\ TCP (SYN-ACK, Source Port = 80 , Dest. Port = 17000, Seq = 10, ACK = 2 )
     		\item Data link (Source MAC = aa:aa:aa:aa:aa:aa, Dest. MAC = 01:01:01:01:01:01) /\\  IP (Source IP = 10.0.0.2, Dest. IP = 1.2.3.4) /\\ TCP (ACK, Source Port = 17000 , Dest. Port = 80, Seq = 2, ACK = 11)
     		\item IP (Source IP = 19.19.19.19, Dest. IP = 1.2.3.4) /\\ TCP (ACK, Source Port = 17001 , Dest. Port = 80, Seq = 2, ACK = 11 ) 
     		%verbindung steht
     		\item Data link (Source MAC = aa:aa:aa:aa:aa:aa, Dest. MAC = 01:01:01:01:01:01) /\\  IP (Source IP = 10.0.0.2, Dest. IP = 1.2.3.4) /\\ TCP (Message, Source Port = 17000 , Dest. Port = 80, Seq = 3) /\\ HTTP (GET / HTTP1.1, HOST = www.example.com)
     		\item IP (Source IP = 19.19.19.19, Dest. IP = 1.2.3.4) /\\ TCP (ACK, Source Port = 17001 , Dest. Port = 80, Seq = 3)  /\\ HTTP (GET / HTTP/1.1, HOST = www.example.com)
     		
     		\item IP (Source IP = 1.2.3.4, Dest. IP = 19.19.19.19) /\\ TCP (Message, Source Port = 80 , Dest. Port = 17001, Seq = 11, ACK = 4 ) /\\ HTTP (HTTP/1.1 200 OK, Content-Length: ..., Content = <html>...</html>)
     		\item Data link (Source MAC = 01:01:01:01:01:01, Dest. MAC =  aa:aa:aa:aa:aa:aa) /\\  IP (Source IP = 1.2.3.4, Dest. IP = 10.0.0.2) /\\ TCP (Message, Source Port = 80 , Dest. Port = 17000, Seq = 11, ACK = 4 ) /\\ HTTP (HTTP/1.1 200 OK, Content-Length: ..., Content = <html>...</html>)
     		
     		%tcp verbindungsabbau
     		\item Data link (Source MAC = aa:aa:aa:aa:aa:aa, Dest. MAC = 01:01:01:01:01:01) /\\  IP (Source IP = 10.0.0.2, Dest. IP = 1.2.3.4) /\\ TCP (FIN, Source Port = 17000 , Dest. Port = 80, Seq = 4, ACK = 12)
     		\item IP (Source IP = 19.19.19.19, Dest. IP = 1.2.3.4) /\\ TCP (FIN, Source Port = 17001 , Dest. Port = 80, Seq = 4, ACK = 12)
     		
     		\item IP (Source IP = 1.2.3.4, Dest. IP = 19.19.19.19) /\\ TCP (FIN, Source Port = 80 , Dest. Port = 17001, Seq = 12, ACK = 5 )
     		\item Data link (Source MAC = 01:01:01:01:01:01, Dest. MAC =  aa:aa:aa:aa:aa:aa) /\\  IP (Source IP = 1.2.3.4, Dest. IP = 10.0.0.2) /\\ TCP (FIN, Source Port = 80 , Dest. Port = 17000, Seq = 12, ACK = 5 )
     		
     		\item Data link (Source MAC = aa:aa:aa:aa:aa:aa, Dest. MAC = 01:01:01:01:01:01) /\\  IP (Source IP = 10.0.0.2, Dest. IP = 1.2.3.4) /\\ TCP (ACK, Source Port = 17000 , Dest. Port = 80, Seq = 5, ACK = 13)
     		\item IP (Source IP = 19.19.19.19, Dest. IP = 1.2.3.4) /\\ TCP (ACK, Source Port = 17001 , Dest. Port = 80, Seq = 5, ACK = 13)
     		
     	\end{enumerate}
  \end{enumerate}
\newpage
     
\subsection*{Problem 5}
      \begin{enumerate}[label=\alph*)]
      	\item Wir müssen die TXID und den UDP Source Port raten. Dies sind $2^{16}$ Werte jeweils, also insgesamt $2^{32}$ Werte. Die Wahrscheinlichkeit beim ersten Versuch richtig zu raten liegt also bei $P=2^{-32}$
		
		\item 	Wie bereits in oben beschrieben gibt es $2^{32}$ mögliche Werte, also muss eine Angreifer durchschnittlich $\frac{2^{32}}{2}$ verschiendene Werte raten um erfolgreich zu sein.
		
	Geht man von 2000 Versuchen pro Runde und 0,1 Sekunden pro Runde aus, kann man die Durchschnittliche Zeit in Sekunden berechnen: 
	\begin{align*}
		&=\frac{2^{31}\text{ value}}{\dfrac{2000 \frac{\text{ value}}{\text{ round}}}{0,1\frac{\text{ second}}{\text{ round}}}}  \\ &=\dfrac{2^{31}\text{value}}{20000\frac{\text{value}}{second}}  \\ 
		&= 107374,182 \text{ seconds  }  \, \, \widehat{=}\,\, 1,243 \,\,\text{ days} \\
   \end{align*}
      \end{enumerate}
      
   
   
  

   	
    

\end{document}